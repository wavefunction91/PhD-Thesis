\chapter{Pauli Algebra on Even Rank Spinors}
\label{apx:SpinorOp}

In this section, we develop the necessary spin--algebra on even ranked tensors over the spin basis (spinor) to supplement the development 
of non--collinear formulations of quantum mechanics.
Let $\spinor{2}{X}$ be a rank-2 spinor, such that
\begin{equation}
\label{eq:XSpinBlock}
\spinor{2}{X} = 
  \begin{bmatrix}
    \spinor{2}{X}^{\alpha\alpha} & \spinor{2}{X}^{\alpha\beta} \\
    \spinor{2}{X}^{\beta\alpha}  & \spinor{2}{X}^{\beta\beta} \\
  \end{bmatrix}, 
\end{equation}
where $\left\lbrace\spinor{2}{X}^{\sigma\sigma'} \in S
\text{ }\vert\text{ } \sigma,\sigma' \in \{\alpha,\beta\} \right\rbrace$ is a
set of tensor components over some vector space $S$. In this work, the state of affairs will
typically dictate $S = \GL{C}{N}$: the  of complex linear operators of dimension $N$. However, 
for the pruposes of this appendix, a general space $S$ will suffice. From \cref{eq:XSpinBlock}, 
it is clear $\spinor{2}{X} \in S \times \GL{C}{2}$.
Choosing the standard basis of $\GL{C}{2}$ as $U^2 \times U^2$,
we may recast \cref{eq:XSpinBlock} as
\begin{equation}
\label{eq:XSpinComp}
  \spinor{2}{X} = \sum_{\sigma\sigma'} \spinor{2}{X}^{\sigma\sigma'} \otimes \vc{e}_\sigma \otimes \vc{e}_{\sigma'} 
  \qquad 
  \vc{e}_\alpha = \begin{bmatrix} 1 \\ 0 \end{bmatrix}, \quad 
  \vc{e}_\beta  = \begin{bmatrix} 0 \\ 1 \end{bmatrix}.
\end{equation}

In this work, we rely on a change of basis to simplify the subsequent derivations and arithmetic in the development
of non--collinear electronic structure. We choose the basis of the Pauli matricies (\cref{eq:PauliMats}), for which it may
be shown that 
%\begin{equation}
%\mathrm{span}\left[ \{ \vc{I}_2, \vc{\sigma}_z, \vc{\sigma}_y, \vc{\sigma}_x \} \right] = \GL{C}{2},
%\end{equation}
i.e. $\left\lbrace \pauli{K} \text{ }\vert\text{ } K \in \{0,1,2,3\} \right\rbrace$ forms a basis of  $\GL{C}{2}$.
Because the Pauli matricies form a basis for $\GL{C}{2}$, there must exist $\{ \spinor{2}{X}^0, \spinor{2}{X}^1, \spinor{2}{X}^2, \spinor{2}{X}^3 \}$
such that
\begin{equation}
\label{eq:XPauliBasis}
\spinor{2}{X} = \sum_{K = 0 }^3 \spinor{2}{X}^K \otimes \pauli{K} \quad.
\end{equation}
From the definitions in \cref{eq:PauliMats}, we may define linear transformations ($\mathscr{T}$ and $\mathscr{T}^{-1}$) between the two bases and their
components (the other being that of \cref{eq:XSpinComp}),
\begin{subequations}
\begin{align}
% Basis transformation
\label{eq:BasisTrans}
&\begin{bmatrix}
  \pauli{0} \\
  \pauli{3} \\
  \pauli{2} \\
  \pauli{1} 
\end{bmatrix} = \mathscr{T} 
\begin{bmatrix}
  \vc{e}_\alpha \otimes \vc{e}_\alpha  \\
  \vc{e}_\alpha \otimes \vc{e}_\beta  \\
  \vc{e}_\beta  \otimes \vc{e}_\alpha   \\
  \vc{e}_\beta  \otimes \vc{e}_\beta
\end{bmatrix} \qquad
%
\mathscr{T} = 
\begin{bmatrix}
1 &  0 & 0 &  1 \\
1 &  0 & 0 & -1 \\
0 & -i & i &  0 \\
0 &  1 & 1 &  0 \\
\end{bmatrix} \quad, \\
\nonumber\\
% Component transformation
\label{eq:CompTrans}
&\begin{bmatrix}
  \spinor{2}{X}^0 \\
  \spinor{2}{X}^3 \\
  \spinor{2}{X}^2 \\
  \spinor{2}{X}^1
\end{bmatrix} = \mathscr{T}^{-T} 
\begin{bmatrix}
  \spinor{2}{X}^{\alpha \alpha}  \\
  \spinor{2}{X}^{\alpha \beta} \\
  \spinor{2}{X}^{\beta  \alpha}   \\
  \spinor{2}{X}^{\beta  \beta}
\end{bmatrix} \qquad
%
\mathscr{T}^{-1} = \frac{1}{2} \mathscr{T}^\dagger \quad.
%\begin{bmatrix}
%1 &  1 &  0 & 0 \\
%0 &  0 &  i & 1 \\
%0 &  0 & -i & 1 \\
%1 & -1 &  0 & 0 \\
%\end{bmatrix}
\end{align}
\end{subequations}
By resolving the identity with $\mathscr{T}$ in \cref{eq:XSpinComp}, we arrive at \cref{eq:XPauliBasis}.

As a consequence of \cref{eq:XPauliBasis,eq:BasisTrans,eq:CompTrans}, a number of properties are immediately evident. Firstly,
suppose there is another rank-2 spinor $\spinor{2}{Y}$ described as in \cref{eq:XPauliBasis}, the product of $\spinor{2}{X}$ and 
$\spinor{2}{Y}$ takes on a component form
\begin{align}
\label{eq:XYSpinorProd}
\spinor{2}{XY} &= \left(\sum_{K = 0 }^3 \spinor{2}{X}^K \spinor{2}{Y}^K\right) \otimes \pauli{0}   \nonumber \\
&\quad  + \sum_{k = 1}^3 \left( \spinor{2}{X}^0 \spinor{2}{Y}^k + \spinor{2}{X}^k \spinor{2}{Y}^0 + \sum_{j,l = 1}^3 i\epsilon_{kjl}\spinor{2}{X}^j \spinor{2}{Y}^l \right) 
    \otimes \pauli{k}
    \quad.
\end{align}
This form is convenient for a number of reasons, however in the context of electronic structure, \cref{eq:XYSpinorProd} exhibits particular utility in the
context of operator traces, i.e. property evaluation. Using the product ansatz of \cref{eq:XYSpinorProd}, we may write the trace of $\spinor{2}{X}$ and
$\spinor{2}{Y}$ (denoted $\mathrm{Tr}[\spinor{2}{X}\spinor{2}{Y}]$) simply as
\begin{equation}
\mathrm{Tr}[\spinor{2}{XY}] = 2\sum_{K = 0}^3 \mathrm{Tr}\left[\spinor{2}{X}^K\spinor{2}{Y}^K\right]
\end{equation}
This simplicity of this expression is due to the fact that the trace operation over the Kronecker product is given by
\begin{equation}
\mathrm{Tr}[\vc{A} \otimes \vc{B}] = \mathrm{Tr}[\vc{A}]\mathrm{Tr}[\vc{B}]
\end{equation}
and that the Pauli matricies are traceless with the exception of $\pauli{0}$ which has a trace of 2.

This notion of Pauli representation may be extended to any aribtrary even--ranked spinor such that
\begin{equation}
\spinor{2N}{X} = \sum_{\sigma_1\sigma_1'\cdots \sigma_N\sigma_N'} \spinor{2N}{X}^{\sigma_1\sigma_1'\cdots \sigma_N\sigma_N'} \otimes 
  \bigotimes_{i = 1}^N  \vc e_{\sigma_i} \otimes \vc e_{\sigma_i'}.
\end{equation}
By resolving the $\mathscr{T}$ identity $N$ times, we obtain
\begin{equation}
\spinor{2N}{X} = \sum_{K_1\cdots K_N} \spinor{2N}{X}^{K_1\cdots K_N} \otimes
  \bigotimes_{i=1}^N \pauli{K_i},
\end{equation}
where
\begin{equation}
\spinor{2N}{X}^{K_1\cdots K_N} = \frac{1}{2N} \sum_{\sigma_1\sigma_1'\cdots \sigma_N\sigma_N'} 
  \mathscr{T}^*_{\sigma_1\sigma'_1}\cdots\mathscr{T}^*_{\sigma_N\sigma'_N}\spinor{2N}{X}^{\sigma_1\sigma_1'\cdots \sigma_N\sigma_N'}  
\end{equation}
%The utility of such a transformation becomes apparent when we examing tensor contractions involving two even-rank spinors. For example,
%consider the tensor contraction of a rank-4 spinor $\spinor{4}{X}$ with a rank-2 spinor $\spinor{2}{Y}$ to form a rank-2 spinor $\spinor{2}{Z}$.
%\begin{equation}
%\spinor{2}{Z}^{\sigma\sigma'} = \sum_{\tau\tau'}\spinor{4}{X}^{\sigma\sigma'\tau\tau'} \spinor{2}{Y}^{\tau\tau'}
%\end{equation}
