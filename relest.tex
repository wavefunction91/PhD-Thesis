\chapter{Two--Component Relativistic Electronic Structure Theory}


\section{Noncollinear Kohn--Sham Density Functional Theory}
\label{sec:NCDFT}

\subsection{Assembly of $\mathbf{V}^{xc}$ for Spinor Densities}

In this section, we examine the integration and assembly of the $E^{GGA}_{xc}$ dependent terms of
the Kohn--Sham Fock matrix using a spinor density. In general, the exchange correlation contribution to the
electronic energy may be written as an integral over an exchange correlation (xc) integration kernel, $f$,
\begin{equation}
  \label{eq:EXCGGA}
E_{xc}^{GGA}[\rho, \nabla \rho] = \int \mathrm{d}^3 \vc r \text{  }f(\{ U(\vc{r}) \}) ,
\end{equation}
where we have introduced a set of auxiliary ``U"--variables, $\{U(\vc{r})\}$, upon which
the xc kernel depends. These need not be the density variables ($\rho, \nabla \rho$) directly,
which we will refer to as ``V"--variables, $\{V(\vc{r})\}$, but rather have complete flexibility in
functional form. In non--relativistic, spin--polarized density functional theory, these sets of variables
may be defined as
\begin{equation}
  \label{eq:Vcol}
  \{ V^{col}(\vc{r}) \} = \{ \rho^\alpha(\vc{r}), \rho^\beta(\vc{r}), \nabla \rho^\alpha(\vc{r}), \nabla \rho^\beta(\vc{r}) \} 
\end{equation}
and
\begin{equation}
  \label{eq:Ucol}
  \{ U^{col}(\vc{r}) \} = \{ \rho^\alpha(\vc{r}), \rho^\beta(\vc{r}), \gamma^{\alpha\alpha}(\vc{r}), \gamma^{\alpha\beta}(\vc{r}), \gamma^{\beta\beta}(\vc{r}) \}
\end{equation}
where
\begin{equation}
  \gamma^{\sigma\sigma'}(\vc{r}) = \nabla \rho^\sigma(\vc{r}) \cdot \nabla \rho^{\sigma'}(\vc{r})
\end{equation}
The practical utility for the use of these two separate sets of variables 
is especially apparent in the context of two--component density functional theory
as it allows for a simple retrofitting of standard xc functionals for relativistic calculations
by simply redefining the transformations from the V variables of relativistic theory. 

In the following, we will make use of the transformation
of the spin blocked density of \cref{eq:2CDen} to the variables $\{\rho^S,\rho^z,\rho^y,\rho^x\}$
via \cref{eq:CompTrans}.  Defining
the magnetization density as $\mathbf{m}(\vc{r}) = \{\rho^z(\vc{r}), \rho^y(\vc{r}), \rho^x(\vc{r})\}$, non--collinear
analogues for \cref{eq:Vcol,eq:Ucol} may be defined as
\begin{align}
  \label{eq:NCAuxVar}
  &\{ V^{NC}(\vc{r}) \} = \{ \rho^S(\vc{r}), \mathbf{m}(\vc{r}), \nabla \rho^S, \nabla \mathbf{m}(\vc{r}) \} \\
  &\{ U^{NC}(\vc{r}) \} = \nonumber \\ &\quad\{ n^+(\vc{r}), n^-(\vc{r}), \gamma^{++}(\vc{r}), \gamma^{+-}(\vc{r}), \gamma^{--}(\vc{r}) \}
\end{align}
where we have used $NC$ to denote non--collinearity. The connection between $\{U^{col}\}$ and $\{U^{NC}\}$
is clear by making the substitution $\alpha \leftrightarrow +$ and $\beta \leftrightarrow -$. 
$\{V^{col}\}$ and $\{V^{NC}\}$ may be related by recognizing $\{\rho^\alpha,\rho^\beta\}$ as the diagonal contributions
of the spinor density, and thus
\begin{equation}
  \{\rho^\alpha(\vc{r}), \rho^\beta(\vc{r})\} \mapsto \{\rho^S(\vc{r}), \mathbf{m}^{col}(\vc{r})\}
\end{equation}
where $\mathbf{m}^{col}(\vc{r}) = \{\rho^z(\vc{r}),0,0\}$. Given the components of the spinor density matrix,
spatial evaluation of the V--variables is given by
\begin{align}
  \label{eq:VVarEval}
  \rho^I(\vc{r}) &= 
%    \sum_{\mu\nu} P^I_{\mu\nu} \chi_\mu(\vc{r}) \chi_\nu(\vc{r}) = 
    \sum_{\mu\nu} \mathrm{Re}[P^I_{\mu\nu}] \chi_\mu(\vc{r}) \chi_\nu(\vc{r})\\
  \nabla\rho^I(\vc{r}) &= 
%    \sum_{\mu\nu} P^I_{\mu\nu} \nabla \left(\chi_\mu(\vc{r}) \chi_\nu(\vc{r})\right) = 
    2\sum_{\mu\nu} \mathrm{Re}[P^I_{\mu\nu}] \chi_\mu(\vc{r}) \nabla\chi_\nu(\vc{r})
\end{align}
where $\mathrm{Re}[x]$ denotes the real part of $x$. Remark that the evaluation of $\{V(\vc{r})\}$ may then be practically 
evaluated using strictly real arithmetic.

Given a transformation, $\{ V^{NC}(\vc{r}) \} \mapsto \{U^{NC}_I(\vc{r})\}$, it is possible to perform practical
density functional calculations using standard implementations of collinear xc functionals, such
as those provided by \texttt{libxc} \cite{Burnus12_2272,Marques18_1}. However, defining such a transformation is not 
a trivial task, as the added spin degrees of freedom in the non--collinear spinor density and
Fock matrix must obey to stricter conditions than their collinear counterparts, such as orientation
invariance and adhering to the zero--torque theorem for the xc potential.\cite{Gyorffy01_206403}
Several definitions of the generalized density variables have been proposed.\cite{Wullen02_779,Ullrich05_073102,Frisch07_125119,Frisch12_2193,Scuseria13_035117}
In this work, we utilize the transformation method which meets such conditions from Ref.\cite{Li17_2591} 
\begin{subequations}
  \label{eq:NCtrans}
  \begin{align}    
    n^\pm(\vc{r})        & = \frac{1}{2}\rho^S(\vc{r}) \pm \frac{1}{2} \vert\mathbf{m}(\vc{r})\vert\\
    \gamma^{\pm\pm}(\vc{r}) & = \frac{1}{4}\nabla\rho^S (\vc{r})\cdot \nabla\rho^S (\vc{r}) \\  
                         & + \frac{1}{4}\sum_k \nabla\rho^k (\vc{r}) \cdot \nabla\rho^k (\vc{r}) \nonumber \\
                         & \pm \frac{f_{\nabla}(\vc{r})}{2}\sqrt{ \sum_k D^{Sk} (\vc{r})^2} \nonumber \\
    \gamma^{+-}(\vc{r})      & = \frac{1}{4}\nabla\rho^S (\vc{r})\cdot \nabla\rho^S (\vc{r}) \\ 
                         & - \frac{1}{4}\sum_k \nabla\rho^k (\vc{r}) \cdot \nabla\rho^k (\vc{r}) \nonumber
  \end{align}
\end{subequations}
where
\begin{align}  
  f_{\nabla}(\vc{r}) & = sgn\left(\nabla\rho^S (\vc{r}) \cdot \left(\sum_k \nabla\rho^k (\vc{r}) \cdot \rho^k (\vc{r}) \right)\right) \\
  D^{IJ} (\vc{r})      &  = \nabla\rho^I (\vc{r}) \cdot  \nabla \rho^J (\vc{r})  
\end{align} 
Using this set of transformations, we may define an electronic energy and Fock matrix which has no dependence
on the global orientation of $\mathbf{m}(\vc{r})$ and which satisfies the zero torque theorem (see Appendix) for
the exchange correlation potential also in the GGA framework.

In the limit of small $\mathbf{m}(\vc{r})$ ($\lvert \mathbf{m}(\vc{r}) \rvert < 10^{-12}$, in this work), the transformations outlined in 
\cref{eq:NCtrans} yield numerically unstable
expressions for the exchange correlation potential.\cite{Li17_2591} Thus one must define another set of transformations for practical
implementations of non--collinear DFT to ensure proper convergence in the limit of small $\mathbf{m}(\vc{r})$. In summary,
this change for small $\mathbf{m}(\vc{r})$ may be described by the following substitutions in \cref{eq:NCtrans},
\begin{subequations}
  \label{eq:smallTrans}
%  \begin{align}  
%    n^\pm(\vc{r})        & = \frac{1}{2}\rho^S(\vc{r}) \pm \frac{1}{2} \left( \frac{1}{3}\sum_k \rho^k(\vc{r}) \right) \\ 
%    \gamma^{\pm\pm}(\vc{r}) & = \frac{1}{4}\nabla\rho^S (\vc{r})\cdot \nabla\rho^S (\vc{r})  \\
%                         & + \frac{1}{4}\sum_k \nabla\rho^k (\vc{r}) \cdot \nabla\rho^k (\vc{r}) \nonumber \\
%                         & \pm \frac{f_{\nabla}(\vc{r})}{2} \left( \frac{1}{3}\sum_k D^{Sk}  (\vc{r}) \right) \nonumber \\
%    \gamma^\pm(\vc{r})      & = \frac{1}{4}\nabla\rho^S (\vc{r})\cdot \nabla\rho^S (\vc{r}) \\ 
%                         & - \frac{1}{4}\sum_k \nabla\rho^k (\vc{r}) \cdot \nabla\rho^k (\vc{r}) \nonumber
%  \end{align}
\begin{align}
  \vert \mathbf{m}(\vc{r}) \vert   &\mapsto m_s(\vc r) = \frac{1}{3}\sum_k \rho^k(\vc{r}),\\
  \sqrt{ \sum_k \left( D^{Sk} (\vc{r}) \right)^2} &\mapsto \nabla \rho^S(\vc r) \cdot \nabla m_s(\vc r).
\end{align}
\end{subequations}
Using these mappings ensures no orientation dependence of $\mathbf{m}$ while maintaining numerical stability
in the resulting expression for the xc potential. While \cref{eq:smallTrans} formally violates the zero
torque theorem, its influence on the over all nature of the electronic density has been shown to be 
negligible\cite{Li17_2591}.

Differentiating \cref{eq:EXCGGA} with respect to the elements of the density matrix, we obtain for \cref{eq:VXCdef}
\begin{align}
  \label{eq:VXCGGA_eval}
V^{GGA,I}_{\mu\nu} &= \sum_{\Gamma \Gamma'} \int \mathrm{d}^3 \vc r \text{  }
  \frac{\partial f }{\partial U^\Gamma(\mathbf{r})}
  \frac{\partial U^\Gamma(\mathbf{r})}{\partial V^{\Gamma'}(\mathbf{r})}
  \frac{\partial V^{\Gamma'}(\mathbf{r})}{\partial P^{I}_{\mu\nu}} \\ &= \nonumber \int \mathrm{d}^3 \vc r \text{  } V^{GGA,I}_{\mu\nu}(\vc{r})
\end{align}  
where the partial derivatives of $f$ are the same as in collinear DFT, and the partial derivatives of $\{V(\vc{r})\}$ may be
identified through differentiating \cref{eq:VVarEval} for a particular spin component. We refer the reader to the Appendix
of Ref.\cite{Li17_2591} for explicit expressions for the partial derivatives of $\{U(\vc{r})\}$.

In practice, \cref{eq:EXCGGA,eq:VXCGGA_eval} are evaluated numerically using a molecular quadrature scheme,
\cite{Becke88_2547,Jackson90_7453,Laming93_997,Johnson95_169,Frisch96_213,Reveles04_681,Sierka11_3097}
\begin{align}
  &E_{xc}^{GGA}[\rho,\nabla\rho] \approx \sum_i w(\vc{r}_i) f(\{U(\vc{r}_i)\}) \label{eq:EXCGGA_num}\\
  &V^{GGA,I}_{\mu\nu} \approx \sum_i w(\vc{r}_i) V^{GGA,I}_{\mu\nu}(\vc{r}_i) \label{eq:VXCGGA_num}
\end{align}
where $\{w(\vc{r}_i)\}$ is a set of quadrature weights. 
In this work, we utilize the Becke multi-center numerical integration scheme,\cite{Becke88_2547}
where the integral is evaluated on series of overlapping atomic centered grids, transformed, through their weights, into ``fuzzy'', overlapping, and analytically continuous cells instead. 
We also refer the reader to a more thorough discussion regarding specific details.\cite{Becke88_2547,Frisch96_213}
%%%%% We can put this back if we need to
%Here, we simply remark that the hallmark of the Becke integration scheme is that the the molecular integrand is decomposed
%into a sum over overlapping atomic integrands,
%\begin{equation}
%  \mathcal{I} = \sum_A \mathcal{I}_A \qquad \mathcal{I} \in \{ E_{xc}^{GGA}, V^{GGA,I}_{\mu\nu} \}
%\end{equation}
%where the atomic integrands, $\mathcal{I}_A$ can be performed using standard spherical integration techniques.
%%%%%%
The evaluation of \cref{eq:EXCGGA_num} is straight forward as it is a scalar function. In the spirit of the intermediates used
in Ref.\cite{Sierka11_3097}, by substituting the definitions of the partial derivatives of the U and V variables, we arrive
at a concise expressions for assembly of \cref{eq:VXCGGA_num} 
\begin{align}
&V^{GGA,I}_{\mu\nu} = \sum_i Z^I_\mu(\vc{r}_i) \chi_\nu(\vc{r}_i) + Z^I_\nu(\vc{r}_i) \chi_\mu(\vc{r}_i) \label{eq:VXC_point_update}\\
  &Z^I_\mu(\vc{r}) = w(\vc{r})
    \left(
      \frac{1}{2} \mathcal{Z}^I_{\rho}(\vc{r}) \chi_\mu(\vc{r}) + \sum_\xi \mathcal{Z}^I_{\nabla,\xi}(\vc{r}) \nabla_\xi \chi_\mu(\vc{r})
    \right) \label{eq:ZImu}
\end{align}
where $\xi \in \{x,y,z\}$ and
\begin{subequations}
\begin{align}
&\mathcal{Z}^I_{\rho}=     \frac{\partial f}{\partial \rho^I} =
                                     \dfrac{1}{2}
                                     \begin{cases}
  \left( \dfrac{\partial f}{\partial n^+} + \dfrac{\partial f}{\partial n^-} \right) & I = S \\[12pt]
  \left( \dfrac{\partial f}{\partial n^+} - \dfrac{\partial f}{\partial n^-} \right) K^{I} & I \neq S \\
                                     \end{cases}
\end{align}
\label{eq:ZrhoVar}
\end{subequations}

\begin{subequations}
\label{eq:ZgammaVar}
\begin{align}
&\mathcal{Z}^I_{\nabla,\xi} = \frac{\partial f}{\partial \nabla_\xi \rho^I} = \dfrac{1}{2} \begin{cases}
                                                          \nabla_\xi\rho^S    \left(  \dfrac{\partial f}{\partial \gamma^{++}} + \dfrac{\partial f}{\partial \gamma^{+-}} + \dfrac{\partial f}{\partial \gamma^{--}} \right)  
                                                          + \sum_k \nabla_\xi \rho^k H^k 
                                                          \left( \dfrac{\partial f}{\partial \gamma^{++}} - \dfrac{\partial f}{\partial \gamma^{--}} 
                                                          \right)                                                          
                                                          & I= S \\[8pt]                                                                                                        
                                                          \nabla_\xi\rho^S H^I  \left( \dfrac{\partial f}{\partial \gamma^{++}} - \dfrac{\partial f}{\partial \gamma^{--}} \right) 
                                                          + \nabla_\xi\rho^I
                                                          \left(\dfrac{\partial f}{\partial \gamma^{++}} - \dfrac{\partial f}{\partial \gamma^{+-}} + \dfrac{\partial f}{\partial \gamma^{--}}
                                                           \right)
                                                          & I \neq S
                                                        \end{cases}                                                                                                                                     
\end{align}
\end{subequations}
where we have dropped the explicit dependence on $\vc r$ for brevity. To consolidate the transformation rules of 
\cref{eq:NCtrans,eq:smallTrans}, we now define
\begin{subequations}
\begin{align}
&K^I = 
\begin{cases}
  \dfrac{\rho^I}{\vert \mathbf{m} \vert} & \text{(Significant }\mathbf{m}\text{)} \\[20pt]
  \dfrac{1}{3} & \text{ (Small }\mathbf{m}\text{)} \\[12pt]
\end{cases}\\
&H^I = 
\begin{cases}
  \dfrac{f_\nabla D^{SI} }{\sqrt{\sum_k \left(D^{Sk}\right)^2   }} & \text{(Significant }\mathbf{m}\text{)} \\[20pt]
  \dfrac{f_\nabla}{3} & \text{(Small }\mathbf{m}\text{)} \\[12pt]
\end{cases}
\end{align}
\end{subequations}


\subsection{Implementation}
\label{sec:NCDFT_IMPL}

On modern computing architectures, there are three primary facets one must consider when developing high--performance
scientific software: parallelism, cache utilization, and exploitation of micro-architecture specific floating point
operations ($\mu$-ops) such as single instruction--multiple data (SIMD) and fused multiply--add (FMA) operations. 
We refer the reader to the work of Goto, \emph{et al}\cite{Goto08_TOMS12} for an excellent discussion of these considerations in the context of matrix operations.
In the context of density functional theory, maximal exploitation of computational cache
and $\mu$-ops is achieved through batching groups of integration points together to maximize screening capability and
memory contingency. 
There exist many batching schemes for various molecular integration quadratures in the 
literature\cite{Becke88_2547,Jackson90_7453,Laming93_997,Johnson95_169,Frisch96_213,Reveles04_681,Sierka11_3097}.
We provide the following discussion without loss of generality.

As the point--wise
function evaluations required for numerical integration are completely independent, it constitutes what is called an
\emph{embarrassingly parallel task}, i.e. no communication is required between the independent operations and thus
one should expect near linear speedup with the number of processors used. 
In the context of electronic structure theory,
the final two facets can usually be addressed through the use of highly optimized linear algebra software, such as
the optimized BLAS (basic linear algebra subroutines) implementations offered OpenBLAS\cite{Qing13_ICHPC25,Yunquan12_PDS684} BLIS\cite{vanDeGeijn15_TOMS14} and Intel--MKL\cite{MKL}. 
However, blind application of such software without careful consideration will often yield sub-optimal results,
thus it is often the case that one must perform some level of algorithmic rearrangement to maximally utilize
such capability. 
To demonstrate this point, we examine the assembly of the exchange correlation potential
in \cref{eq:VXC_point_update}.
One may immediately recognize that the operation on the left of \cref{eq:VXC_point_update}
is a sum over symmetric rank--2 updates (SYR2) of column vectors, $\mathbf{z}$ and $\boldsymbol{\chi}$,
i.e.
\begin{equation}
\label{eq:VXC_SYR2}
\mathbf{V}^{GGA,I} = \sum_i \mathbf{z}^I_i \boldsymbol{\chi}_i^T + \boldsymbol{\chi}_i \left(\mathbf{z}^I_i\right)^T ,
\end{equation}
where $\mathbf{z}_i^I$ and $\boldsymbol{\chi}_i$ are of length of the number of basis functions, $N_b$. While \cref{eq:VXC_SYR2}
is a valid scheme for the assembly of \cref{eq:VXC_point_update}, for large $N_b$ this scheme will be drastically sub-optimal.
This is due to the fact that, for large $N_b$, $\mathbf{V}^{GGA,I}$, $\mathbf{z}_i^I$ and $\boldsymbol{\chi}_i$ occupy a significant
portion of the computational caches for each point. This means that the probability of the program attempting to access
a memory address, i.e. an element of $\mathbf{z}_i^I$ or $\boldsymbol{\chi}_i$, between integration points and finding that it does not
currently reside in the cache, i.e. a ``cache miss", is rather high relative to other memory access patterns. This yields a
large degradation in performance as whenever a cache miss occurs, the program must then move that address in some manner to
the cache from main memory before it can perform any operations on it. Moving data to and from main memory is disproportional
more expensive than floating point operations, thus it must be kept to a minimum to obtain
optimal efficiency.

We may instead factor out a portion of the sum in \cref{eq:VXC_SYR2} such that we may partition it into sum over batches of points.
In detail, we use the macrobatch approach,\cite{Frisch96_213} where the grid points of each atoms are arranged in Lebedev spheres\cite{Lebedev77_99}  of several radii, for the numerical integration of the exchange correlation potential,

\begin{align}
\label{eq:VXC_SYR2K}
\mathbf{V}^{GGA,I} &= \sum_{S_j \in S} \left(\sum_{i\in S_j} \mathbf{z}^I_i \boldsymbol{\chi}_i^T + \boldsymbol{\chi}_i \left(\mathbf{z}^I_i\right)^T \right) \nonumber \\
&= \sum_{S_j \in S} \mathbf{Z}^{I(j)} \left(\mathbf{X}^{(j)}\right)^T + \mathbf{X}^{(j)} \left(\mathbf{Z}^{I(j)}\right)^T
\end{align}
where
\begin{equation}
\mathbf{Y}^{(j)} = \begin{bmatrix} \mathbf{y}_1 & \mathbf{y}_{2} & \cdots & \mathbf{y}_i & \cdots &\mathbf{y}_{\vert S_j \vert} \end{bmatrix} \quad \forall i\in S_j,
\end{equation}
and $\mathbf{y}$ is either $\mathbf{z}$ or $\boldsymbol{\chi}$. \Cref{eq:VXC_SYR2K} is a sum over symmetric rank--2$k$ updates (SYR2K), where $k = \vert S_j \vert$. By tuning
$k$, one improves caching behavior dramatically. This is due to the fact that optimized implementations of SYR2K operations utilized efficient block
operations to optimize the flow of data to and from the computational caches. 
%  \todo{make a table / plot of cache misses + wall time for SYR2 and SYR2K}.}
 Similar schemes may
developed for the evaluation of the V variables (\cref{eq:VVarEval}) over batches using optimized matrix--matrix multiplication routines. However, while the caching behavior
is improved with increasing $k$, this is not the only consideration one needs take into account when partitioning the integration grid into batches.

The scheme in \cref{eq:VXC_SYR2K} may be further improved by recognizing the fact  that the basis functions typically used for molecular calculations carry 
a degree of spatial locality. 
A pictorial representation of the screening and updating scheme is given in \cref{fig:Integration}.
For each batch, we create a list of basis functions
that effectively overlaps it (colored subset of basis functions in \cref{fig:Integration}). 
This list of significant basis functions will be different for
each batch, but the number of basis
functions in each list becomes independent of size
for sufficiently large molecules, given the  spatial localization nature of Gaussian atomic centered basis sets.
%As such, not every basis function need be evaluated for every batch depending on the distance of the basis function to the points in the
%batch. Deciding which basis functions need to be evaluated are dependent on the characteristics of the basis set. \todo{explain / cite how we screen}. 
This reduced list of basis, evaluated for all points in the batch, is stored in contiguous blocks of memory, and used (along with the corresponding submatrices of the density-matrix, when required) for the evaluation of the potential, $\mathbf{Z}$ (see \cref{eq:ZImu}), by exploiting a sub-sequential series of vectorized operations.
An important note here is that the maximum values for the batch of basis functions, $\chi_{max}(batch)$, and potential, $Z_{max}(batch)$, can be used to screen the entire contribution of the points in the batch to the integration. In this case, the integration can move to the next batch, avoiding the rank-2k update, that is the computationally most expensive part of the process,  without loosing accuracy.
Otherwise, recognizing that $\mathbf{Z}$ and $\mathbf{X}$ exhibit the same sparsity pattern, we may define
\begin{equation}
\tilde{\mathbf{V}}_{(j)}^{GGA,I} = \tilde{\mathbf{Z}}^{I(j)} \left(\tilde{\mathbf{X}}^{(j)}\right)^T + 
\tilde{\mathbf{X}}^{(j)} \left(\tilde{\mathbf{Z}}^{I(j)}\right)^T
\end{equation}
where moieties denoted with a tilde are the packed quantities where only the basis functions which have been chosen to be evaluated for the batch are represented.
The packed $\tilde{\mathbf{V}}_{(j)}^{GGA,I}$ may then be used to update the full $\mathbf{V}^{GGA,I}$ by mapping its elements to those in the full basis dimension.

\subsection{Discussion}

In this section, we will provide validation and computational performance
results for the proposed non--collinear KS--DFT method. The proposed method was
implemented in a locally modified version of the open source
\texttt{ChronusQ}\cite{chronusq_beta2} electronic structure software package.
All calculations were performed using Intel Haswell compute nodes (14$\times$2
Intel\textregistered Xeon E5--2680 v4 CPUs @ 2.40 GHz, 32k L1 cache, 256k L2
cache, 35840k L3 cache) without the exploitation of molecular point group
symmetry. All numerical integrations were carried out with the Becke molecular
integration scheme using 100 Euler--Maclaurin~\cite{Laming93_997} quadrature points for the
radial integration and 302 Lebedev~\cite{Lebedev77_99} points for the angular integration around
each atom.

\subsubsection{Validation}
\label{sec:NCDFT_VALID}

A series of geometrically frustrated hydrogen rings were used to gauge the validity of the 
proposed two--component Kohn--Sham implementation. Geometrically frustrated systems
provide an excellect test case for the validation of non--collinear electronic structure
methods as their lowest energy mean--field solutions break $\op{S}_z$ symmetry to minimize
Pauli repulsion \cite{Gross07_196405,Frisch07_125119,Frisch12_2193,Scuseria13_035117,Yamaguchi01_670,Blochl03_15772,Truhlar11_2629,Truhlar13_5349,Li15_154109} .
Unlike their collinear counter parts (such as restricted and unrestricted Kohn--Sham), non--collinear
Kohn--Sham is able to support this broken symmetry due to its explicit treatment of the full
spinor nature of the electronic density. To this end, a series of six Hydrogen rings ranging from 3 to
8 hydrogens were constructed such that each hydrogen was placed at 1 \AA~ spacing around 
an equidistant circle (see \cref{fig:rings}). Thus only the odd membered rings may be considered geometrically
frustrated. All calculations involving Hydrogen rings were performed using the X2C-B3LYP/6-311+G(d,p)
level of theory\cite{Becke93_5648,Parr88_785,Preuss89_200}. 
The scalar and spin densities of the Hydrogen rings solutions have been examined in \cref{fig:rings}. 


In comparison to the thorough discussion of geometric frustration of Hydrogren rings in Ref \cite{Li15_154109}, 
we can see that for the even numbered Hydrogren rings,
the symmetrical distribution of the spin and scalar densities indicates an anti--ferromagnetic spin
alignment; the same as one would get in a collinear solutions. This is further confirmed by the fact that
the expectation value of $\op{S}$ for these solutions was found to be zero. thus all spins must be antialligned.
While is this perhaps not the most
intetesting result in the context of non--collinear calculations, it does indicate that our implementation
collapses to the expected collinear behaviour if needed. Further in the case that the spin density is small
in all space (the six membered Hydrogen ring), we can see that the the choice for the generalized auxillary
variables in \todo{reference eq} is a robust and accurate choice even in the worst case scenario.
The primary result of \cref{fig:rings} is that both the spin and scalar densities adopt a symmetrical
distribution even for the odd membered rings; a property which would be impossible in a collinear solution
due to the geometrical frustration of the system. An anti--ferromagnetic solution (the lowest energy
solution within the unrestricted formalism for the Slater determinant ground state) would yield an
asymmetric spin density. Thus, our implementation of X2C-KS is able to reproduce the expected symmetries of
the non--collinear soltions for both even and odd membered Hydrogen rings, ensuring the robustness and 
reilability of the proposed method.




%Molecular systems characterized by geometrically frustrated conformations have
%been used to gauge the reliability of the presented implementation, since
%generalized-KS  and X2C-KS, unlike U-KS, methods are capable to support the
%broken $\op{S}_z$ symmetry that minimizes Pauli
%repulsion\cite{Gross07_196405,Frisch07_125119,Frisch12_2193,Scuseria13_035117,Yamaguchi01_670,Blochl03_15772,Truhlar11_2629,Truhlar13_5349,Li15_154109} .
%Thus, we examined a series of neutral hydrogen rings, ranging from 3 to 8
%hydrogen atoms, that have already been proved to show broken symmetry solutions
%within generalized Hartree-Fock theory.\cite{Li15_154109} The used
%conformations have 1 \AA~ spaced hydrogen atoms and only the  odd-membered
%rings will be geometrically frustrated.  All  hydrogen rings electronic
%structure were obtained by solving the presented X2C-KS equation  in
%open-source \texttt{ChronusQ}\cite{chronusq_beta2} electronic structure
%software using the hybrid Becke, 3-parameter, Lee-Yang-Parr (B3LYP) density
%functional 
%\cite{Becke93_5648,Parr88_785,Preuss89_200} with a 6-311+g** basis, since it is required to include the polarization of hydrogenic s-functions.\cite{Li15_154109}
%Qualitatively,  X2C-KS solutions are given in \cref{fig:rings}, where both the spin and the charge densities are reported, respectively. The spin density in the non-collinear
%approach is a mapping of the 2-norm of the spin magnetization vector onto the total density \cite{Wullen02_779}, ensuring the spin density to be positive and always real, and the charge density is a mapping of the electrostatic potential to the total density.
%For even numbered rings, the anti-ferromagnetic
%spin distribution can be observed, aside from the six rings (where the closed shell solution is the most stable one according to the $2n + 1$ rule for M\"obius-like periodicity).
%For odd membered rings, the M\"obius-like periodicity of magnetization ordering
%in X2C-KS results can be inspected by the totally symmetric distribution of the spin density throughout the structure in \cref{fig:rings}. This symmetric distribution can not be supported by collinear solutions\cite{Li15_154109} which are constrained to be eigenfunctions of $\op{S}_z$. Thus, X2C-B3LYP  is able to reproduce the expected symmetry of non-collinear results for odd membered rings \cite{Li15_154109}, ensuring the reliability of the  current implementation.

\begin{figure}
\begin{center}
\resizebox{6.5in}{!}{%
  \includegraphics{rings.pdf}
}
\end{center}
\caption{X2C-B3LYP  6-311+g(D,P) spin ($\vert \vc{m}(\vc{r})\vert$, left) and scalar ($\rho^{1,0}(\vc{r})$, right) densities for a series of Hydrogen rings.
Blue and Red represent regions of greater density magnitude respectively.}
\label{fig:rings}       
\end{figure}

\subsubsection{Computational Performance}

\begin{figure}
\begin{center}
\resizebox{6.5in}{!}{%
  \includegraphics{radicals.pdf}
}
\end{center}
\caption{X2C-B3LYP 6-311+G(2D,P) spin densities ($\vert\vc{m}(\vc{r})\vert$) for phenoxy radicals with an increasing number of fused benzene rings 
($n$=1-10, where $n$ represents the number of fused benzene rings on each side).}
\label{fig:radicals}       
\end{figure}

After we have checked the reliability and presented the schematics of the integration scheme, 
we focus on the performances of the method.
All the following tests presented were performed without exploiting symmetry using B3LYP/6-311+G(2d,p) theory level
All times refer to the combined wall time for 
the numerical evaluation of $E_{XC}$ and the matrix elements  $V^{GGA,I}_{\mu\nu}$, (\cref{eq:EXCGGA_num} and \cref{eq:VXCGGA_num}) for
each  self consistent field (SCF) step (as average over 5 steps).
The scaling of the numerical integration
with respect to increasing system sizes is investigated first by measuring wall timings on a series of phenoxy radicals with increasing number of symmetrical fused benzene rings ($n$=1-10). 
These systems have been chosen, since they present a delocalized spin-density over the entire molecule (see \cref{fig:radicals}) for all values of $n$. This is very important, since the screening based on the different components of the density can be very large in regions showing zero magnetization density.
Wall times are shown in \cref{fig:timing} as function of the total number of atoms in the systems for both the U-KS and X2C-KS implementations.
The scaling factors are similar for both U-KS and X2C-KS integration (see the parallel fitted lines in the figure), showing how the  X2C-KS integration does not show any consistent computational overhead in the   $V^{GGA,I}_{\mu\nu}$ integration with
respect to the common U-KS integration. 
Finally, by employing the largest phenoxy radical of the series ($n$=10), the same wall times are recorded as function of increasing MPI processes across different computational nodes (for 1, 2, 4, 8, 12 nodes, for a total 
of 28, 56, 112, 224, 336 CPUs) and reported in \cref{fig:nodes}. 
The implemented algorithm presents linear ($-1.01$) with increasing number of CPUs, as can be expected since the integration over different atomic centers can be easily independently performed across different computing nodes.
\begin{figure}
\begin{center}
\resizebox{3.25in}{!}{%
  \includegraphics{timings_natoms.pdf}
}
\end{center}
\caption{Wall timings in seconds as a function of the total number of atoms in the systems for the evaluation of the $E_{XC}$ and the matrix elements $X_{\mu \nu}$ in one SCF step (as average over 5 steps, 6-311+G(2d,p), 100$\times$302 grid.
  Both the U-KS, empty diamonds, and X2C-KS, filled circles, are reported for comparison. A linear fit in this log-log scale is performed to show the similar scaling for the two different integration. 
}
\label{fig:timing}     
\end{figure}

\begin{figure}
\begin{center}
\resizebox{3.25in}{!}{%
  \includegraphics{cpus.pdf}
}
\end{center}
\caption{
Wall timings in seconds as function of increasing computational CPUs of the $E_{XC}$ and the matrix 
elements $X_{\mu \nu}$ for each SCF step (as average over 5 steps, 6-311+G(2d,p), 100$\times$302 grid, 
2619 basis functions). A linear fit, -1.01 slope, is also represented as full line.
}
\label{fig:nodes}       
\end{figure}

\subsection{Conclusions}
In this work, we presented an integration and assembly strategy, 
using a two-component spinor density, for efficient evaluation
of the exchange correlation term using localized basis
functions. 
The presented formulation is suitable to exploit parallelism 
along with optimal cache utilization and micro-architecture specific floating point operations.
This leads to the evaluation of exchange correlation
contributions with matrices of optimal sizes. 
We also show that the proper choice of auxiliary variables 
correctly give rise to nonzero local torque of the xc magnetic field on the magnetization
while maintaining net zero global torque, as is expected from
the exact functional. 
Several tests were used to validate 
the reliability and the performance of the proposed strategy.
This approach can help to extend 
the applicability of relativistic two-component DFT 
to systems of large size ($>$ 100 atoms). 







\section{The Relativistic Particle--Particle Random Phase Approximation}

\subsection{Motivation}

The ability to accurately predict and characterize the electronically excited
states of molecular systems is paramount to a complete understanding of many
chemical phenomena. As such, excited states are the central focus of many
fields of physical chemistry, most prominent being that of spectroscopy. Due to
this centralized importance and the need to efficiently and accurately predict
excited states properties, much  effort has been devoted over the years towards
the modeling of excitation energies and oscillator strengths.


Recently, the particle-particle random phase approximation (pp-RPA) and 
Tamm-Dancoff approximation (pp-TDA), which have been standard trade tools of 
the nuclear physics community in the treatment of the many--body correlation energy for low matter density 
systems for some time \cite{SchuckBook_04}, 
have been extended to the treatment the correlation energy and excitation energies of quantum 
molecular systems within a finite basis set \cite{Yang13_224105,Yang13_18A522,
Yang13_174110,Yang13_104112,Yang13_030501,Yang09_066403,Bulik13_104113}.  Although this 
introduction into the quantum chemistry community in relatively recent, a wealth 
of effort has been afforded to the rigorous investigation of these methods in a 
variety of different contexts, involving the evaluation of excitation 
energies \cite{Yang13_224105,Yang13_18A522,Yang13_174110}, excited state properties 
and geometry optimizations \cite{Yang15_1025}, and the treatment of non-adiabatic 
phenomena such as non-adiabatic dynamics \cite{Liu14_244105} and the description 
of conical intersections \cite{Yang16_2407}.  So far in their development, however, 
these methods have only seen application in molecular systems using strictly 
spin-collinear references, disallowing extension to systems with 
spin-noncollinear states, such as those that arise in spin-frustrated systems 
or whenever spin-orbit effects are included in the treatment of the electronic structure.


Recent years have also seen new developments in the realm of relativistic quantum 
chemistry.  Relativistic effects, while often neglected in most standard treatment
of electronic structure, can have profound consequences in chemical 
systems.\cite{Pyykko12_45} Scalar relativistic effects cause the contraction of 
the core electron shells of heavy atoms, but perhaps of even more consequence is 
the introduction of spin couplings in the Hamiltonian.  Spin-spin and spin-orbit 
interactions can affect the electronic spin dynamics even in light atoms, and a 
direct consequence of these couplings on excited states is the loss of degeneracies
of spin-eigenstates, giving rise to fine structure splittings (FSS) in atoms and 
molecules with symmetry-induced degeneracies. It is therefore desirable to develop accurate and 
cost-effective relativistic electronic structure methods able to model such effects. 


Thus, in this work, we extend the pp-TDA formalism for use with two--component
relatvistic reference determinants, specifically the X2C-HF reference of \cref{sec:X2C,sec:MF2C}. 
However, the presented formalism may be employed in any case where the GSCF method must be
employed, such as the spin-frustrated systems explored in \cref{sec:NCDFT_VALID},
even in the absence of relativistic effects. 




\subsection{The Particle-Particle Tamm-Dancoff Approximation}
\label{subsec:ppTDA}
The pp-TDA approximation is a non-particle-number conserving method, i.e. the excitation operator of \cref{eq:ExcitationOp} does not commute with the number operator. 
To model a system with $N$ electrons, the pp-TDA starts from a reference determinant for a system with $N-2$  electrons, and adds two electrons back using 
an appropriate excitation operator.
The ground and excited states of the $N$-particle system are thus obtained as ``excited states'' of the $N-2$ electron reference, and the desired excitation energies can be written as simple energy differences.
A general formalism for the treatment of the pp-TDA within a finite basis of spin--collinear MOs has described rigorously elsewhere \cite{Yang13_224105,Yang13_18A522,Yang13_104112}.
Here, we review this formalism for completeness and present the working expressions for the relativistic X2C pp-TDA (X2C-pp-TDA) within the basis of MO spinors 
as well as describe some caveats in the practical application of this method within the context of a two-component reference.

From an (exact) $M$-particle ground state, $\ket{\Psi_0^M}$, all ground and excited 
$N$-particle states ($\ket{\Psi_0^N}$ and $\ket{\Psi_n^N}$ respectively) may be regarded as ``excitations" through the 
excitation operator $\mathcal{O}^\dagger_n = \ket{\Psi_n^N}\bra{\Psi_0^M}$,
\begin{equation}
 \ket{\Psi_n^N} = \mathcal{O}^\dagger_n \ket{\Psi_0^M} \quad.
\end{equation}
Given such an ansatz, it is possible to construct an equation-of-motion (EOM)\cite{SchuckBook_04} for $\mathcal{O}_n^\dagger$, affording some corresponding
probing de-excitation operator $\delta \mathcal{O}$ , to obtain the eigenenergies of $\ket{\Psi_n^N}$,
\begin{equation}
\label{eq:EOM}
 \left[ \delta \mathcal{O},\left[ \mathcal{H}, \mathcal{O}_n^\dagger \right]\right] = (E^N_n - E^M_0)\left[\delta \mathcal{O}, \mathcal{O}^\dagger_n \right]
 \quad.
\end{equation}

\Cref{eq:EOM} is formally exact and completely independent of the chosen reference, provided one has access to the exact $N$- and $M$-particle states.
In practice, one may take the expectation value of \cref{eq:EOM} using some approximate $M$-particle ground state, $\ket{\Phi_0^M}$, to obtain \emph{approximate} energy differences between $N$- and $M$-particle states given some explicit form of $\mathcal{O}^\dagger_n$.
As has been previously discussed, if the $N$ and $M$ systems differ by exactly two particles ($M=N- 2$), the X2C-pp-TDA equation may be obtained by
postulating that the excitation operator take the form of all (unique) 2-particle additions to obtain $N$-particle states from an $(N-2)$-particle reference,
\begin{equation}
 \mathcal{O}_n^\dagger = \sum_{a < b} X^n_{ab} c_a^\dagger c_b^\dagger \quad, \label{eq:ppTDAExcitation}
\end{equation}
while the de-excitation operator takes the form
\begin{equation}
 \delta\mathcal{O} = c_b c_a \quad.
\end{equation}
%Here, $\lbrace c^\dagger_p \rbrace$ takes the form of the field operators from \Cref{eq:SQSpinor} and the distinction between ``particle" and 
%``(\emph{p})article" has been made apparent by the fact that ``(\emph{p})articles" are quasi-particles that exist only above the Fermi-level while
%``particles" may exist above or below the Fermi-level.
Here, $X^n_{ab}$ is an expansion coefficient that describes the contribution to the $n$-th $N$-particle state of the addition of 2 particles into single-particle virtual states, $a$ and $b$, of the $(N-2)$-particle ground state reference. By taking the expectation value of \cref{eq:EOM} in a single X2C-HF Slater determinant
given \cref{eq:ppTDAExcitation}, one obtains the Hermitian eigenvalue problem of the X2C-pp-TDA
\begin{align}
\sum_{c < d} A_{ab,cd} X^n_{cd} &= \Omega_n X^n_{ab} \quad (a < b), \label{eq:ppTDA}\\
\Omega_n &= (E_n^N - E_0^{N-2}) \\
A_{ab,cd} &= \delta_{ac}\delta_{bd} (\epsilon_a + \epsilon_b) + \innerop{ab}{}{cd} \quad \label{eq:AMat},
\end{align}
while $\innerop{\cdot}{}{\cdot}$ is the antisymmetrized two-electron integral tensor in Dirac integral notation and $\epsilon_p$ is a spinor-orbital
eigenenergy obtained from solving the X2C-HF equation in \cref{eq:RHX2C}. One may obtain neutral $N$-particle excitations from the eigen spectrum of \cref{eq:ppTDA} via a reference shift to the
$(N-2)$-particle system. By variationally optimizing the wave function of the $(N-2)$-particle system via X2C-HF, both the $N$-particle ground and excited state energies are obtained via
\begin{align}
E_n^N = E_0^{N-2} + \Omega_n \quad . \label{eq:ESEnergy}
\end{align}
Thus the excitation energy between $N$-particle ground and excited states, $\omega_n^N$, described via the X2C-pp-TDA may be written as differences of the eigenenergies,
\begin{equation}
\omega_n^N = \Omega_n - \Omega_0 \quad (n > 0). \label{eq:EXEnergy}
\end{equation}

These working expressions in \cref{eq:ppTDA,eq:AMat,eq:ESEnergy,eq:EXEnergy} are similar to those previously expressed for the spin-collinear reference.\cite{Yang13_224105,Yang13_18A522,Yang13_104112}
The key difference is that all of the above equations are expressed in the spinor basis within the X2C picture frame rather than a spin-collinear spin-orbital basis and that the orbitals have been optimized in the presence of relativistic spin-orbit effects. 
While \cref{eq:ppTDA} is a completely valid expression for the spin-noncollinear  as well as the spin-collinear case,
the spin-collinear case allows for significant simplification due to the spin-orthogonality of the spin-aligned reference\cite{Yang13_224105,Yang13_174110}.
%\Cref{eq:ppTDA} provides the maximally simplified expression for the X2C-pp-TDA  as such spin orthogonality cannot be exploited. 

\subsection{Results and Discussion}
\label{sec:ppX2CResults}

\co{This section needs heavy revision to cite internal sections as opposed to papers}
All calculations were performed with a locally modified version of the Gaussian quantum chemistry suite of programs,\cite{GDVI06}
and employed the taug-cc-pVTZ-DK basis set\cite{Dixon01_48} with the diffuse \emph{f}-functions removed.
Relativistic effects were accounted for by means of the variational X2C method \co{Specific HF}.\cite{Reiher13_184105,Saue11_3077,Liu09_219,Liu10_532,Liu09_1945}
In order to partially account for two-electron spin-orbit interaction in the Hamiltonian, we employed a scheme based on the scaling of the nuclear charge according to the angular momenta.\cite{Boettger00_7809}
The atomic nuclei, rather than being treated as point charges, were described using $s$-type Gaussian charge distribution.\cite{Dyall97_207,Saue98_920}
The stability of the two-component ground state wave function was also tested before X2C-pp-TDA calculations were performed.\cite{Li15_154109}

\subsubsection{Single Excitations}
\label{subsec:SingleEx}
In order to highlight the capability of the pp-TDA method to describe excited
states within a relativistic framework, in this section we examine the fine
structure splittings of some atomic systems.  The presence of spin-orbit
couplings 
%causes the total spin, $\vec{S}$, and orbital, $\vec{L}$, angular
%momentum operators to no longer commute with the Hamiltonian, therefore they no
%longer generate good quantum numbers for the system.  Instead, the total
%angular momentum $\vec{J}=\vec{L}+\vec{S}$ is the fundamental quantity that
%should be considered when classifying the electronic states of the system.  A
%direct consequence of spin-orbit couplings on the spectra of atoms is the
%lifting of some of 
lifts some of the energetic degeneracies that would be expected in the ground or
excited electronic states.  We therefore calculate the \ch{spectrum}{excitation energies} of selected
atomic systems and compare the obtained fine structure splittings with
experimental reference values\cite{NIST_ASD} to asses the accuracy of the
method.  In this section we restrict ourselves to states describable by single
excitations (with respect to the $N$-electron system) which allows us to also
compare our results with the results obtained using the two-component
particle-hole Random-Phase Approximation (X2C-ph-RPA), also known as
Time-Dependent Hartree-Fock method (X2C-TDHF), as well as with results obtained
using the X2C-ph-TDA method.\cite{Li16_3711} Results for these fine-structure
splittings are collected in \cref{tb:SingleEx}.

It can be seen that, in general, the three methods perform similarly with
respect to the reference values insofar as the order of magnitude of the error
is concerned. A general trend may be observed in that the X2C-pp-TDA
consistently overestimates the splittings as the atomic charge of the
underlying nucleus increases. This effect is magnified in the low energy
transitions while it is less apparent in the higher energy transitions. This is
due to the fact that the frontier orbitals of the ($N-2$)-reference being used
become sub-optimal in the proper description of the $N$-electron system due to
a contraction in the presence of higher nuclear charge. This leads to an
unphysically small energetic separation between the frontier orbitals of the
$N$-electron system which causes increasing errors due to an unphysical
increase in mixing. This problem is less obvious in higher energy excitation
because the higher lying orbitals are not as affected. These orbitals are
properly optimized in the X2C-ph-RPA/TDA due to orbital occupancy of the
resulting wave function. The general out-performance of the X2C-ph-TDA over the
X2C-ph-RPA may be attributed to an over estimation of electron correlation in
the ground and excited states via the RPA\cite{Dreuw05_4009}. The presence of
the de-excitation amplitudes in the X2C-ph-RPA allow for an over-mixing for the
low-lying excited states with the ground state which give rise to an
overestimate of the FSS, much the same as the case for the X2C-pp-TDA.
                                                                                                                                                 
 
                                                                                                                                                                                                                                                              
 
The main advantage of the particle-particle over the particle-hole formalism is
that, in the former, both the ground and electronically excited $N$-particle
states are described on equal footing with respect to correlation, being a
linear combination of several Slater determinants.  Conversely, in the ph-TDA
or ph-RPA method, the ground state is described as a single determinant, while
excited states are described as linear combinations of single excitations (and
possibly de-excitations).  That being said, the excitation space spanned by the
X2C-pp-TDA solutions does not include all chemically relevant excitations, many
of which can be found using the more traditional ph-RPA or ph-TDA methods.
This is due to the fact that the X2C-pp-TDA is, in its traditional form,
incapable of accessing excitations that involve contributions from below the
Fermi level.  Some work has been done in attempts to resolve this
problem\cite{Yang13_224105}, but these alterations to the pp-TDA method have
not been explored in this work.

\begin{table}[htbp]
  \caption{Calculated and reference\cite{NIST_ASD} excited-state fine structure splittings (in meV) of some atomic systems. The presence of a superscript ``$\circ$" in the term symbol denotes an odd state with respect to space inversion.}
 \label{tb:SingleEx}
 \centering
 \begin{tabular}{llrrr}
  \hline
  Method     & Level                                              &  Mg   & Al$^+$ & Si$^{2+}$ \\ \hline
  X2C-ph-RPA & \multirow{3}{*}{$^3$P$^\circ_{1}-^3$P$^\circ_{0}$} &  4.89 & 10.49  & 19.85 \\
  X2C-ph-TDA &                                                    &  2.41 &  7.94  & 16.62 \\
  X2C-pp-TDA &                                                    &  2.77 &  9.13  & 18.97 \\
  Ref        &                                                    &  2.49 &  7.55  & 15.94 \\
  \hline
  X2C-ph-RPA & \multirow{3}{*}{$^3$P$^\circ_{2}-^3$P$^\circ_{1}$} &  9.75 & 21.02  & 39.96 \\
  X2C-ph-TDA &                                                    &  4.82 & 15.96  & 33.50 \\
  X2C-pp-TDA &                                                    &  5.55 & 18.40  & 38.40 \\
  Ref        &                                                    &  5.05 & 15.36  & 32.45 \\
  \hline
  X2C-ph-RPA & \multirow{3}{*}{$^3$P$^\circ_{1}-^3$P$^\circ_{0}$} &  0.33 &  1.82  &  4.43 \\
  X2C-ph-TDA &                                                    &  0.33 &  1.82  &  4.31 \\
  X2C-pp-TDA &                                                    &  0.42 &  2.17  &  5.08 \\
  Ref        &                                                    &  0.41 &  1.73  &  4.10 \\
  \hline
  X2C-ph-RPA & \multirow{3}{*}{$^3$P$^\circ_{2}-^3$P$^\circ_{1}$} &  0.67 &  3.67  &  9.09 \\
  X2C-ph-TDA &                                                    &  0.67 &  3.68  &  8.86 \\
  X2C-pp-TDA &                                                    &  0.84 &  4.50  & 11.04 \\
  Ref        &                                                    &  0.84 &  3.65  &  9.07 \\ 
  \hline
  \\
  \\
             &  &  MSE  &  MAE   & \\
  \hline
  X2C-ph-RPA &  &  2.32 &  2.28  & \\
  X2C-ph-TDA &  &  0.32 &  0.19  & \\
  X2C-pp-TDA &  &  1.55 &  1.55  & \\
  \hline
 \end{tabular}
\end{table}

\subsubsection{Triplet References and Double Excitations}
\label{subsec:DoubleEx}
In this section we wish to highlight other advantages of X2C-pp-TDA over conventional X2C-ph-RPA.
In the previous section we presented results for atomic systems that are characterized by being closed shell in both the $N$ and $N-2$ systems.
This is important because if the reference state has unpaired electrons then, as a consequence of the single reference nature of the Hartree-Fock wave function,
excited states will in general be spin-contaminated, affecting one's ability to extract meaningful fine structure splittings from the results, though adaptations to the general case exist.\cite{Liu10_064106,Suo11_134101,Liu11_194106,Liu12_024107,Liu13_3741}
By using X2C-pp-TDA it is however possible to treat systems with $N$ electrons with any odd spin multiplicity, provided they become closed shell upon the addition or removal of two electrons.
Molecules which possess triplet ground states as well as diradical moieties may be taken as examples.
To demonstrate this feature we compare the FSS of one set of excited states of molecular oxygen with experimental data in \cref{tb:DoubleEx}.
The difference between the calculated and measured value is just 2.5~meV, notwithstanding the approximations intrinsic in our method (e.g.~the approximate treatment of electron correlation and the two-electron spin-orbit contributions, or the finite basis set).
Of course, the same reasoning can also be applied in reverse: X2C-ph-RPA theory can be readily used to find excited-state FSS of systems with a singlet ground state, however if the addition or removal of two electrons produces an open-shell molecule, then X2C-pp-TDA will present some spin-contamination in the computed excited-states.



One advantage that X2C-pp-TDA always has over X2C-ph-RPA theory, however, is its ability to describe double excitations.
\Cref{tb:DoubleEx} compares calculated and reference excited-state FSS of doubly-excited states of some atomic moieties.
The performance of the method is similar as in the case of single excitations presented in the previous section.
Such states cannot be found among the excited states computed via X2C-ph-RPA.

\begin{table}[htbp]
 \caption{Excited-state fine structure splittings (in meV).}
 \label{tb:DoubleEx}
 \centering
 \begin{tabular}{llrr}
  \hline
  System & Level & X2C-pp-TDA & Ref \cite{NIST_ASD,Krupenie72_423} \\ \hline
  O$_2$ & $^3\Delta_3-{^3\Delta_2}$ & 20.58 & 18.09 \\ \hline
  \multirow{2}{*}{Al$^+$} & $^3$P$_1-^3$P$_0$ & 9.20 & 7.75 \\ 
  & $^3$P$_2-^3$P$_1$ & 17.93 & 15.03 \\  \hline
  \multirow{2}{*}{Si$^{2+}$} & $^3$P$_1-^3$P$_0$ & 19.46 & 16.55 \\ 
  & $^3$P$_2-^3$P$_1$ & 37.88 & 32.06 \\    \hline
 \end{tabular}
\end{table}

%-------------------------------------------------------------------------------
% CONCLUSION
%-------------------------------------------------------------------------------
\subsection{Conclusion}

In this work, a scheme for the extension of the pp-TDA method to relativistic two-component wave functions has been presented.
This scheme involves the approximate decoupling of the large and small components of the relativistic wave function by means of the X2C method, followed by an Hartree-Fock SCF calculation on the system obtained by removing two electrons, in order to obtain a set of complex spinor molecular orbitals.
The two-component reference system is then used in the X2C-pp-TDA calculation that yields the ground and excited states for the $N$-electron system.
The extension of the pp-TDA to a two-component reference comes at the cost of the employing complex spinor orbitals, and not being able to separate the problem into smaller sub-problems as is done in the case of RHF or UHF references via spin integration.
The increased computational cost highlights the ever pressing need for direct and parallel implementations of post-SCF electronic structure methods, which is exaggerated in the case of relativistic electronic structure calculations.

It has been shown that the X2C-pp-TDA method exhibits excellent results in the prediction of the fine-structure splittings of the atomic and molecular species considered here.
The results are comparable and at times better than those obtained using X2C-ph-RPA.\cite{Li16_3711}
In addition, the X2C-pp-TDA is able to capture electronic excitations traditionally inaccessible by the X2C-ph-RPA/TDA thanks to the 2-particle reference shift, such as double excitations and those that would be described as spin-contaminated in particle-number conserving methods.
While these results are promising, the general applicability of the X2C-pp-TDA method, as with the spin-collinear variant, is limited as it is traditionally unable to capture excitations that involve contributions of orbitals from below the Fermi level.
That being said, there are many systems, such as triplet and diradical systems, that the X2C-pp-TDA provides a suitable method for the accurate description of the
electronic manifold.

