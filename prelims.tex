\prelimpages


\Title{Towards Scalable Relativistic Electronic\\
  Structure Methods for the Treatment of Molecular
  Response}
\Author{David Williams--Young}
\Year{2018}
\Program{Chemistry}

\Chair{Xiaosong Li}{Professor}{Chemistry}
\Signature{Stefan Stoll}
\Signature{David Masiello}

\copyrightpage

\titlepage  




\setcounter{page}{-1}
\abstract{%
Abstract to go here
}
 
%
% ----- contents & etc.
%
\tableofcontents
\listoffigures
\listoftables  % I have no tables

 
%
% ----- glossary 
%
\chapter*{Glossary}      % starred form omits the `chapter x'
\addcontentsline{toc}{chapter}{Glossary}
\thispagestyle{plain}
%
\begin{glossary}
  \item[$\mu$-op]    Microarchitechture specific operations
  \item[1PDM]        One--particle density matrix
  \item[1RDM]        One--particle reduced density matrix
  \item[2C]          Two--component
  \item[2RDM]        Two--particle reduced density matrix
  \item[4C]          Four--component
  \item[ADC]         Algebraic diagrammatic construction
  \item[B3LYP]       Becke 3--parameter Lee--Yang--Parr exchange correlation functional
  \item[BLAS]        Basic linear algebra subroutines
  \item[BO]          Born--Oppenheimer
  \item[CC]          Coupled cluster
  \item[CPP-CC]      Complex polarization propagator coupled cluster
  \item[CPP-SCF]     Complex polarization propagator self--consistent field
  \item[CPP]         Complex polarization propagator
  \item[CPU]         Central processing unit
  \item[DC]          Dirac--Coulomb
  \item[DFT]         Density functional theory
  \item[EOM]         Equation of motion
  \item[FMA]         Fused multiply--add
  \item[FOPPA]       First order polarization propagator
  \item[FSS]         Fine structure splitting
  \item[FW]          Foldy--Wouthuysen
  \item[GEMM]        Matrix--matrix product
  \item[GEMV]        Matrix--vector product
  \item[GGA]         Generalized gradient approximation
  \item[GHF]         Generalized Hartree--Fock
  \item[GKS]         Generalized Kohn--Sham
  \item[GMRES]       Generalized minimum resudual method
  \item[GSCF]        Generalized self--consistent field
  \item[HF-MO]       Hartree--Fock   molecular orbital
  \item[HF]          Hartree--Fock
  \item[HK]          Hohenberg--Kohn
  \item[KS-DFT]      Kohn--Sham density functional theory
  \item[KS-MO]       Kohn--Sham   molecular orbital
  \item[KS]          Kohn--Sham
  \item[MD]          Modified Dirac equation
  \item[MIMO]        Multiple--input multiple--output
  \item[MKL]         Math kernel library
  \item[MOR]         Model order reduction
  \item[MO]          Molecular orbital
  \item[MPI]         Message passing interface
  \item[NC]          Non--collinear
  \item[NEXAFS]      Near edge X--ray absorption fine structure spectroscopy
  \item[NI]          Non--interacting
  \item[NR]          Non--relativistic
  \item[ph-RPA]      The exact two--component particle--hole random phase approximation
  \item[ph-RPA]      The particle--hole random phase approximation
  \item[ph-TDA]      The particle--hole Tamm--Dancoff approximation
  \item[pp-RPA]      The particle--particle random phase approximation
  \item[pp-TDA]      The particle--particle Tamm--Dancoff approximation
  \item[QFT]         Quantum field theory
  \item[RHF]         Restricted Hartree--Fock
  \item[RKS]         Restricted Kohn--Sham
  \item[RPA]         The random phase approximation
  \item[RSCF]        Restricted self--consistent field
  \item[SCF]         Self--consistent field
  \item[SIMD]        Single--instruction multiple--data
  \item[SYR2K]       Symmetrix rank-2k update
  \item[SYR2]        Symmetric rank-2 update
  \item[TDA]         The Tamm--Dancoff approximation
  \item[TDDFT]       Time--dependent density functional theory
  \item[TDHF]        Time--dependent Hartree--Fock
  \item[UHF]         Unestricted Hartree--Fock
  \item[UKS]         Unestricted Kohn--Sham
  \item[USCF]        Unrestricted self--consistent field
  \item[X2C-HF]      Exact two--component Hartree--Fock
  \item[X2C-KS]      Exact two--component Kohn--Sham
  \item[X2C-ph-TDA]  The exact two--component particle--hole Tamm--Dancoff approximation
  \item[X2C-pp-TDA]  The exact two--component particle--particle Tamm--Dancoff approximation
  \item[X2C]         Exact two--component
  \item[XAS]         X--Ray absorption spectroscopy.
  \item[XC]          Exchange--correlation
\end{glossary}




 
%
% ----- acknowledgments
%
\acknowledgments{% \vskip2pc
  % {\narrower\noindent
  I express my sincere appreciation to
  University of Washington (UW), where I have had the opportunity
  to pursue meaningful research in the field of electronic
  structure theory. Further, I would like to thank Professor
  Xiaosong Li, my research adviser, for his invaluable guidance
  and mentorship over the past five years. The encouragement
  and criticism I have received from Xiaosong during my time
  in his group has shaped the scientist and I am and will
  guide my development in my future scientific endeavors. Xiaosong,
  thank you for allowing me to pursue my own research directions
  and make mistakes an realizations in my own time; it's been
  great working with you.

  I would further like to thank my undergraduate adviser, Professor
  Jaeju Ko, for encouraging my interest in chemical physics and computer
  science at an early point in my career. I would have likely not pursued
  higher education were it not for my research experience in her group.

  At UW, I've had the pleasure to work with, or adjacent to a number of
  outstanding faculty who have helped shape my views of science and 
  its interplay with theory. In particular, Professor David Masiello
  and Professor Stefan Stoll have played an integral part in my development
  as a scientist. Although we may have not always seen eye-to-eye on
  certain aspects of scientific inquiry, your opinions are among those which
  I hold in highest regard.  Thank you for keeping me grounded and for 
  never allowing me to forget that theory must always exist parallel to
  physics. 

  The Li research group has also played an important role in my research career.
  In particular, Dr. Alessio Petrone, Dr. Joshua Goings, Dr. Patrick Lestrange,
  Dr. Feizhi Ding, Dr. Franco Egidi, Dr. Greta Donati, and Dr. David Lingerfelt have been invaluable
  assets in all of my research projects in one way or another. Their expertise has
  complemented mine in fantastic ways as to produce the best possible science
  that we could. The next academic generation of students in the Li group has 
  kept me honest and has aided my ability to convey scientific ideas to
  further the knowledge to others (although it may not always seem like it). 
  Thanks to Joseph Kasper, Joseph Radler, Shichao Sun, Hongbin Liu, Ryan Beck,
  Torin Stetina, Andrew Wildman and Lauren Koulias.

  I'd like to thank my friends and family, without who's support I would
  not be where I am today. My parents Kristen and Thomas were always supportive
  of what ever crazy ideas I had growing up, and encouraged me to pursue my passions
  wherever they may lead me.

  Finally, I'd like to thank my wonderful wife, Gemma. By far you have bore the brunt
  of all my of successes and failures as a graduate student and have been a never ending
  source of love and support for as long as I've known you. Thank you for moving all
  the way across the country and uprooting your life just to come be with me while
  I pursued my Ph.D. It's hard to express my gratitude in full, but thank you.
  % \par}
}

%
% ----- dedication
%
\dedication{\begin{center}to my dear wife, Gemma\end{center}}





\preface{
%
  The pursuit of scientific endeavors can not be done in a vacuum: it is largely a collaborative
  effort in the context of the scientific body as a whole. 
  As such, it would be  prudent to outline my contributions and the contributions of others 
  to the research presented in the following chapters.

  \Cref{ch:Theory} presents a reasonably in-depth overview of the theoretical preliminaries which
  provide a basis for the research in the chapters which follow. While a major part of my graduate
  research career has been the pursuit of a deep understanding of these theories both from a mathematical
  and physical perspective, I would be remiss to attribute any of the fundamental theoretical
  developments outlined in  \cref{ch:Theory} solely to myself. \Cref{ch:Theory} depends heavily
  on the body of scientific literature relating to electronic structure theory and quantum mechanics
  in general: both contemporary and historical texts. The novel aspect of \cref{ch:Theory} is in
  the presentation of these methods in a cohesive and consistent manner such that the treatments of
  relativistic and non--relativistic theory are in a sense equivalent. Much of the formal development
  of the manipulations of these operators in terms of the Pauli matrices was done in collaboration
  with Dr. Franco Egidi (FE).

  \Cref{ch:2CEST} outlines my primary contributions to the field of relativistic electronic structure theory.
  The first project discussed presents a scalable implementation of non--collinear Kohn--Sham density
  functional theory in a Gaussian basis set. The general protocol for the generalized density variables
  utilized in this work were developed by Peralta \emph{et al} (\emph{Phys. Rev. B}. {\bf 2007}, 75, pp 196405) 
  and Egidi, \emph{et al} (\emph{J. Chem. Theory Comput.} {\bf 2017}, 13, pp 2591),
  though no implementation details were divulged in either of those papers. My primary contribution to this
  work was the implementation of the proposed algorithm and in the formal development of the efficient
  intermediates used for the assembly of the Fock matrix. This work was done in collaboration with Dr. Alessio
  Petrone (AP), where much of the code development was written through iterations between him and myself. The
  proof for the adherence of these generalized density variables to the zero-torque theorem was done in 
  collaboration with Shichao Sun and Professor Xiaosong Li. The distributed memory implementation of the numerical integration
  was written by myself. The efficient intermediates were originally derived by myself, then further refined
  through discussions with AP. Further, the framework within which this algorithm was developed 
  (ChronusQ) has been primarily development by myself to this point. The majority of the tests were
  performed by AP and were selected by the two of us.

  The second project discussed in \cref{ch:2CEST} presents an extension of the particle-particle Tamm-Dancoff
  approximation to two-component relativistic Hamiltonians. This project was originally inspired through 
  email discussions with Professor Weitao Yang and is student Dr. Yang Yang at Duke University. They provided
  me with non-relativistic test cases which accelerated the development of this algorithm in Gaussian. The
  development of this algorithm in Gaussian was written by myself, but much of the development would have 
  been stalled had it not been for thorough discussion with FE at the time. The test cases for
  this algorithm were chosen by FE and ran by myself. Analysis of the results in the context of the
  two-component time-dependent Hartree-Fock algorithm was done in equal parts by myself and FE.

  \Cref{ch:MOR} presents a novel algorithm for the estimation of the linear absorption spectrum
  by model order reduction. This work was done in extensive collaboration with Dr. Chao Yang 
  and his student Roel van Beeuman (RB) at Lawrence Berkeley National Lab. The initial idea
  for applying model order reduction to the absorption spectrum is due to RB. The formal
  development of the final algorithm was due to extensive discussion between RB and myself.
  RB provided a proof-of-concept MATLAB implementation of the algorithm using data provided
  by myself. The MATLAB implementation provided the primary reference from which I wrote the 
  production implementation in ChronusQ using small test systems. The validation of the algorithm
  for larger test cases was done by myself. The full eigendecomposition reference values were
  obtained by RB on the Cori supercomputer using matrices provided by myself.
%
}


%
% end of the preliminary pages
