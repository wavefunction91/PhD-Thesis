\chapter{Theoretical Basis}
\label{ch:Theory}

In this chapter, I will outline the theoretical preliminaries which will serve as the basis
for the subsequent development of relativistic electronic structure theory. Further, this
chapter will serve as the primary source of notation which will be used throughout the
remainder of this work. 

% Quantum Mechanics
\section{Quantum Mechanical Preliminaries}
\label{sec:QM}

\subsection{The Physical Hilbert Space and The Slater Determinant}
\label{sec:SD}

Perhaps the most fundamental axiom of quantum mechanics is in that for every physical
system, there is associated a separable complex Hilbert space, $\mathcal{H}$, such that
the vectors of said Hilbert space represent the quantum states 
of the system \cite{VonNeumann55_book}. Such vectors are referred to as wave functions
and the inner product on $\mathcal{H}$ is referred to as an expectation value.
As the primary focus of this work will be the treatment of the many--body electronic
problem in molecular quantum mechanics, one might remark that the physical system
in question is a \emph{composite} system consisting of many, indistinguishable 
particles (electrons). Denoting the Hilbert space of a composite system consisting
of $N$ particles as $\mathcal{H}^N$, we know that as $\mathcal{H}^N$ is separable,
it must admit a countable, dense basis \cite{Lee03_book}.
The direct product space consisting of the Hilbert spaces
which describe its constituent parts, namely those spaces which describe a
single particle system: $\mathcal{H}^1$, provides a convenient basis for $\mathcal{H}^N$ such that
\begin{equation}
  \label{eq:SepHilbert}
  \mathcal{H}^N = \Span{\bigotimes_{i = 1}^N \mathcal{H}^1}.
\end{equation}
The separability condition of \cref{eq:SepHilbert} is crucial to the following developments
as it allows one to construct a simple basis for vectors in $\mathcal{H}^N$
as $N$--fold tensor products of single--particle wave functions which form a basis for $\mathcal{H}^1$, 
i.e. $\mathcal{C} = \set{\ket{\phi_p}}  \subset \mathcal{H}^1$ such that 
$\mathcal{H}^1 = \Span{\mathcal{C}}$. In the following, vectors in $\mathcal{H}^1$, in particular elements of $\mathcal{C}$,
will be referred to as orbitals.

As the electron is the moiety of interest in this work, vectors in $\mathcal{H}^N$ must adhere to certain additional 
criteria in order for them to represent physically realizable wave functions. In the case of electrons, which are 
fermions, physically relevant elements of $\mathcal{H}^N$, which we will denote $\mathcal{H}^N_- \subset \mathcal{H}^N$, must 
adhere to Fermi--Dirac statistics, namely that they must exhibit anti-symmetric behavior under particle permutation
\cite{Walecka12_book,Schuck04_book}, i.e.
\begin{equation}
  \label{eq:FermiDirac}
  \mathcal{P}_{ij} \ket{\Phi^N} = -\ket{\Phi^N}, \qquad \ket{\Phi^N} \in \mathcal{H}^N_-,
\end{equation} 
where $\mathcal{P}_{ij}$ is the particle permutation operator which interchanges particle $i$ and $j$ 
(for a more thorough discussion on linear operators acting on $\mathcal{H}^N$, see \cref{sec:LO}). Remark
that this notion of particle interchange is intimately related to \cref{eq:SepHilbert}. With this additional
constraint, we may construct a basis for $\mathcal{H}^N_-$ with elements
\begin{equation}
  \label{eq:SlaterDet}
  \ket{\Phi^N_I} = \frac{1}{\sqrt{N!}} \sum_{\xi \in S_N(\mathcal{K}^N_I)} \Sign{\xi} \text{ } \bigotimes_{i = 1 }^N \ket{\phi_{\xi(i)}},
\end{equation}
where $\mathcal{K}_I^N$ is an $N$--element subset of $\mathcal{C}$ and $S_N(\mathcal{K}_I^N)$ is the symmetric group
of $\mathcal{K}_I^N$ which consists of all permutations of its elements, which are denoted here as permutation
functions $\xi$. $\Sign{\xi} \in \{ \pm 1 \}$ denotes the sign of the permutation and ensures the anti--symmetry of
the overall wave function. \Cref{eq:SlaterDet} introduces a number of concepts with are typically jargonized 
in the quantum chemistry community. In this work, $\mathcal{K}_I^N$ will be referred to as an $N$--particle
configuration (or simply a configuration when $N$ is to be understood from the context), and $\ket{\Phi^N_I}$
will be referred to as a Slater determinant. It is important to note that $\ket{\Phi^N_I}$ is completely 
determined by $\mathcal{K}_I^N$. As a basis for $\mathcal{H}^N_-$, any vector $\ket{\Psi^N} \in \mathcal{H}^N_-$
may be written as \cite{Ostlund12_book}
\begin{equation}
\ket{\Psi^N} = \sum_I D_I \ket{\Phi^N_I},
\end{equation}
where $D_I\in\mathbb C$ is the complex expansion coefficient of the $I$-th configuration in the overall wave function,
and $I$ runs over all possible $N$--particle Slater determinants which may be constructed from $\mathcal{C}$.
The fact that the set of all Slater determinants forms a basis for $\mathcal{H}^N_-$
serves as the primary foundation for the majority of approximate quantum mechanical methods regarding molecular
systems. In the following, we will assume both $\ket{\Phi^N_I}$ and the elements of $\mathcal{C}$ are orthonormal
with respect to the metric on their respective Hilbert spaces.

While the Hilbert space representation of the wave function is the most illuminating description for the development of
general quantum mechanical theory, it is often advantageous from the perspective of practical calculations that one
projects the vectors of the Hilbert space onto a convenient basis. 
To this end, we consider a specific single particle coordinate basis, $\ket{\vc{x}} \equiv \ket{\vc{r},\sigma}$, 
which consists of the simultaneous eigenfunctions
of both the position and $z$--spin operators, denoted $\op{\vc{r}}$ and $\op{S}_z$ respectively,
such that
\begin{align}
  \op{\vc{r}}\ket{\vc{x}} &= \vc{r}\ket{\vc{x}}, \qquad \vc{r}\in\mathbb{R}^3, \\
  \op{S}_z\ket{\vc{x}}    &= \sigma\ket{\vc{x}}, \qquad \sigma \in \set{\pm\frac{1}{2}}.
\end{align}
Here, we have denoted the particle's position and $z$--axis spin projection as $\vc{r}$ and $\sigma$, respectively.
Namely, if $\mathcal{H}^1$ represents a spin--1/2 fermion, $\ket{\vc{x}}$ forms a complete basis for $\mathcal{H}^1$ 
and admits the following orthonormality  condition on the $\mathcal{H}^1$ inner product,
\begin{equation}
  \inner{\vc{x}'}{\vc{x}} = \delta^3(\vc{r} - \vc{r}')\delta_{\sigma\sigma'},
\end{equation}
where $\delta^3$ and $\delta_{\sigma\sigma'}$ are the Dirac delta function and Kronecker delta tensor, respectively.

\begin{equation}
  \inner{\vc{x}_1,\vc{x}_2,\ldots,\vc{x}_N}{\Phi^N} \equiv \Phi^N(\vc{x}_1,\vc{x}_2,\ldots,\vc{x}_N).
\end{equation}
\co{As $\ket{\vc{x}_1,\vc{x}_2,\ldots,\vc{x}_N}$ exists in a direct product space...}
\begin{equation}
  \label{eq:SlaterDetSpace}
  \Phi_I^N(\vc{x}_1,\vc{x}_2,\ldots,\vc{x}_N) = \frac{1}{\sqrt{N!}} \sum_{\xi \in S_N(\mathcal{K}^N_I)} \Sign{\xi} \text{ } 
    \prod_{i = 1 }^N \phi_{\xi(i)}(\vc{x}_i),
\end{equation}
\co{Define as spatial wave function}
\co{This is great because, unlike the tensor product definition, $\phi_k$'s commute}

\subsection{Linear Operators and Expectation Values}
\label{sec:LO}

Fundamental to the formulation of any quantum mechanical theory is the identification of linear operators on $\mathcal{H}^N$
which represent the physical observables of the system. In this work we will refer to such observables as properties. A
more precise identification of the operators relevant to this work will be presented later 
(see \cref{sec:NRH,sec:RELH,sec:SCLMI} for instance), however in this section we will focus on the general presentation
of these operators and how they will typically manifest in the context of \cref{eq:SepHilbert}. 

Despite to the fact that $\mathcal{H}^N$ may be described as a direct product space consisting of $N$ single particle
Hilbert spaces, operators which act upon $\mathcal{H}^N$ (which we will refer to as $N$--particle operators) need not carry the same structure, 
i.e. in general, these operators need not exist solely as direct products of operators on $\mathcal{H}^1$. This is not to say
that $N$--particle operators \emph{cannot} adopt a product structure, just that it is not a requirement, and indeed
if often not the case. However, the notion that some $N$--particle operators \emph{can} adopt a product structure indicates the need
to describe the action of $M$--particle operators on $\mathcal{H}^N$ (with $N>M$) while leaving $N-M$ particles unchanged.
To demonstrate this state of affairs, it is convenient to examine the action of such an operator, $\op{O}$, on a 
spatial wave function of the form \cref{eq:SlaterDetSpace},
\begin{align}
  \label{eq:MOpSpace}
  &\op{O}(i,j,\ldots) \Phi_I^N(\vc{x}_1,\vc{x}_2,\ldots,\vc{x}_N) = \nonumber \\ &\qquad \quad
    \frac{1}{\sqrt{N!}} \sum_{\xi \in S_N(\mathcal{K}^N_I)} \Sign{\xi} \text{ } 
    \left(\op{O}(i,j,\ldots) \phi_{\xi(i)}(\vc{x}_i)\phi_{\xi(j)}(\vc{x}_j)\cdots\right)
    \prod_{k \neq (i,j,\ldots) }^{N} \phi_{\xi(k)}(\vc{x}_k),
\end{align}
where $(i,j,\ldots)$ is an $M$--element tuple specifying the subset of particle upon which it acts.
In this work, we will typically be concerned with operators which act on no more than two particles at a time.

It will be often the case in this work that $N$--particle operators may be expressed as sums over one and
two particle operators. In this context , the realization of \cref{eq:MOpSpace} is of exceptional utility in that,
for an expectation value involving a particular Slater determinant,
\begin{subequations}
  \label{eq:SlaterCondon}
\begin{align}
  &\innerop{\Phi^N_I}{\sum_i\op{O}(i)}{\Phi^N_I} = \sum_{i\in\mathcal{K}^N_I} O_{ii},\\
  &\innerop{\Phi^N_I}{\sum_{i\neq j} \op{O}(i,j)}{\Phi^N_I} = 
    \sum_{i\neq j \in\mathcal{K}^N_I} O_{ijij} - O_{ijji},
\end{align}
\end{subequations}
where, in the case of spin--1/2 fermions represented in the coordinate basis of \co{add Eq ref},
\begin{subequations}
  \label{eq:DiracInts}
\begin{align}
  &O_{pq} \equiv \innerop{\phi_p}{\op{O}(1)}{\phi_q} = 
    \sum_{\sigma_1} \int_{\mathbb{R}^3} \phi_p^*(\vc{r}_1,\sigma_1) \op{O}(1) \phi_q(\vc{r}_1,\sigma_1) \dd^3 \vc{r}_1,\\
  &O_{pqrs} \equiv \innerop{\phi_p \phi_q }{\op{O}(1,2)}{\phi_r \phi_s} = \nonumber \\ &\qquad
    \sum_{\sigma_1 \sigma_2} \iint_{\mathbb{R}^3} 
      \phi_p^*(\vc{r}_1,\sigma_1) \phi_q^*(\vc{r}_2,\sigma_2) \op{O}(1,2) 
      \phi_r(\vc{r}_1,\sigma_1) \phi_s(\vc{r}_2,\sigma_2) \dd^3 \vc{r}_1 \dd^3 \vc{r}_2.
\end{align}
\end{subequations}
Namely, $M$--particle operators may be represented as rank--$2M$ tensors on $\mathcal{H}^1$; a fact which will
be used extensively in the following developments.


\subsection{Second Quantization}
\label{sec:SQ}

While \cref{eq:SlaterDet} provides the basic structure for the many--body fermionic wave function, its explicit form
is a bit unwieldy and thus its use for practical calculation of moieties such as expectation values is somewhat limited.
To this end we introduce a formalism known as second quantization, or the occupation number formalism \cite{Walecka12_book},
which aims to greatly simplify the construction and manipulation of anti-symmetric wave functions such as those in 
\cref{eq:SlaterDet}. The primary hallmark of second quantization is in the representation of a Slater determinant 
as an array of integers known as \emph{occupation numbers}, denoted $\ket{[n_p^I]}$, indicate the inclusion (or \emph{occupation}) of elements
of $\mathcal{C}$ in the configuration which describes the determinant. As such, the length of said array is $\vert \mathcal{C} \vert$
and the sum of its elements is the number of particles in the system ($N$). Due to the fact that electrons are fermions,
there is a further restriction on the possible values of the occupations numbers due to the Pauli exclusion principle,
namely that a particle can be occupied ($n^I_p=1$) or unoccupied ($n^I_p=0$). To avoid confusion in the subsequent developments,
we will define $\mathcal{H}^N_O$ as the Hilbert space which admits $\{\ket{[n_p^I]}\}$ as a basis, as operators on $\mathcal{H}^N$ 
\emph{do not} act on $\mathcal{H}^N_O$, but rather may act indirectly through the obvious isomorphism 
$\mathcal{H}^N \leftrightarrow \mathcal{H}^N_O$.

One may construct wave functions of the form \cref{eq:SlaterDet} is a consistent manner in this formalism through
the introduction of two sets of operators, $\{ c_p^\dagger \}$ and $\{ c_p \}$, which map $\mathcal{H}^N_O \mapsto \mathcal{H}^N_O$ 
and are referred to as creation and annihilation operators respectively, and the notion of a zero particle determinant known as 
the vacuum, $\Vac$, such that
\begin{subequations}
  \label{eq:SeQuantAction}
\begin{align}
  &\Vac = \ket{[0_1, 0_2, \ldots, 0_{\vert C \vert}]}, \\
  &c_p^\dagger \ket{[n_1, n_2, \ldots, n_p, \ldots, n_{\vert C \vert}]} = \frac{n_p-1}{\sqrt{N+1}} \ket{[n_1, n_2, \ldots, 1_p, \ldots, n_{\vert C \vert}]},\\
  &c_p \ket{[n_1, n_2, \ldots, n_p, \ldots, n_{\vert C \vert}]} = {n_p}{\sqrt{N}} \ket{[n_1, n_2, \ldots, 0_p, \ldots, n_{\vert C \vert}]},
\end{align}
\end{subequations}
where $N$ is the number of particles in the determinant \emph{before} action, and
\begin{subequations}
  \label{eq:SeQuantComm}
\begin{align}
  &[c_p^\dagger, c_q^\dagger]_+ = 0,\\
  &[c_p, c_q]_+ = 0,\\
  &[c_p^\dagger, c_q]_+ = \delta_{pq},
\end{align}
\end{subequations}
where $[\cdot,\cdot]_+$ is the anti--commutator and $\delta_{pq}$ is the Kronecker delta. The action of strings consisting on any number of these 
operators may be derived inductively from \cref{eq:SeQuantAction,eq:SeQuantComm}. As such, we may recast \cref{eq:SlaterDet} as
\begin{equation}
  \ket{\Phi^N_I} \mapsto \left(\prod_{i \in \mathcal{K}^N_I} c_i^\dagger\right) \Vac
\end{equation}

% Non--Relativistic Hamiltonian
\subsection{Non--Relativistic Molecular Hamiltonians: The Schr\"{o}dinger Equation}
\label{sec:NRH}

% Relativistic Hamiltonian
\subsection{Approximate Relativistic Hamiltonians: The Dirac--Coulomb Equation}
\label{sec:RELH}



% Semi--Classical Light Matter
\section{Semi--Classical Light--Matter Interaction}
\label{sec:SCLMI}

% Polarization Propagator
\subsection{The Resonant Convergent Polarization Propagator}
\label{sec:PolarProp}

% Absorption Cross Section
\subsection{The Linear Absorption Cross Section}
\label{sec:AbsorptionTheory}
