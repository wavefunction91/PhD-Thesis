\prelimpages


\Title{Towards Scalable Relativistic Electronic\\
  Structure Methods for the Treatment of Molecular
  Response}
\Author{David Williams--Young}
\Year{2018}
\Program{Chemistry}

\Chair{Xiaosong Li}{Professor}{Chemistry}
\Signature{Stefan Stoll}
\Signature{David Masiello}

\copyrightpage

\titlepage  




\setcounter{page}{-1}
\abstract{%
Abstract to go here
}
 
%
% ----- contents & etc.
%
\tableofcontents
\listoffigures
\listoftables  % I have no tables

 
%
% ----- glossary 
%
\chapter*{Glossary}      % starred form omits the `chapter x'
\addcontentsline{toc}{chapter}{Glossary}
\thispagestyle{plain}
%
\begin{glossary}
  \item[todo] Write glossary
\end{glossary}



 
%
% ----- acknowledgments
%
\acknowledgments{% \vskip2pc
  % {\narrower\noindent
  The author wishes to express sincere appreciation to
  University of Washington, where he has had the opportunity
  to pursue meaningful research in the field of electronic
  structure theory, and to Professor Xiaosong Li, who has
  facilitated said research had has provided invaluable 
  mentorship throughout the author's research career.
  % \par}
}

%
% ----- dedication
%
\dedication{\begin{center}to my dear wife, Gemma\end{center}}





\preface{
%
  The pursiut of scientific endeavours can not be done in a vacuum: it is largely a collaborative
  effort in the context of the scientific body as a whle. 
  As such, it would be  prudent to outline my contributions to the research presented in the
  following chapters.

  \Cref{ch:Theory} presents a reasonably in-depth overview of the theoretical preliminaries which
  provide a basis for the research in the chapters which follow. While a major part of my graduate
  research career was the pursuit of a deep understanding of these theories both from a mathematical
  and physical perspective, I would be remiss to attribute any of the fundamental theoretical
  developments outlines in  \cref{ch:Theory} solely to myself. \Cref{ch:Theory} depends heavily
  on the body of scientific literature relating to electronic structure theory and quantum mechanics
  in general: both contemporary and historical texts. The novel aspect of \cref{eq:Theory} is in
  the presentation of these methods in a cohesive and consistent manner such that the treatments of
  relativistic and non--relativistic theory are in a sense equivalent. This is 
% 
}


%
% end of the preliminary pages
