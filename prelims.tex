\prelimpages


\Title{Towards Scalable Relativistic Electronic\\
  Structure Methods for the Treatment of Molecular
  Response}
\Author{David Williams--Young}
\Year{2018}
\Program{Chemistry}

\Chair{Xiaosong Li}{Professor}{Chemistry}
\Signature{Stefan Stoll}
\Signature{David Masiello}

\copyrightpage

\titlepage  




\setcounter{page}{-1}
\abstract{%
Abstract to go here
}
 
%
% ----- contents & etc.
%
\tableofcontents
\listoffigures
\listoftables  % I have no tables

 
%
% ----- glossary 
%
\chapter*{Glossary}      % starred form omits the `chapter x'
\addcontentsline{toc}{chapter}{Glossary}
\thispagestyle{plain}
%
\begin{glossary}
  \item[todo] Write glossary
\end{glossary}



 
%
% ----- acknowledgments
%
\acknowledgments{% \vskip2pc
  % {\narrower\noindent
  The author wishes to express sincere appreciation to
  University of Washington, where he has had the opportunity
  to pursue meaningful research in the field of electronic
  structure theory, and to Professor Xiaosong Li, who has
  facilitated said research had has provided invaluable 
  mentorship throughout the author's research career.
  % \par}
}

%
% ----- dedication
%
\dedication{\begin{center}to my dear wife, Gemma\end{center}}





\preface{
%
  The pursiut of scientific endeavours can not be done in a vacuum: it is largely a collaborative
  effort in the context of the scientific body as a whole. 
  As such, it would be  prudent to outline my contributions to the research presented in the
  following chapters.

  \Cref{ch:Theory} presents a reasonably in-depth overview of the theoretical preliminaries which
  provide a basis for the research in the chapters which follow. While a major part of my graduate
  research career has been the pursuit of a deep understanding of these theories both from a mathematical
  and physical perspective, I would be remiss to attribute any of the fundamental theoretical
  developments outlined in  \cref{ch:Theory} solely to myself. \Cref{ch:Theory} depends heavily
  on the body of scientific literature relating to electronic structure theory and quantum mechanics
  in general: both contemporary and historical texts. The novel aspect of \cref{ch:Theory} is in
  the presentation of these methods in a cohesive and consistent manner such that the treatments of
  relativistic and non--relativistic theory are in a sense equivalent. Much of the formal development
  of the manipulations of these operators in terms of the Pauli matrices was done in collaboration
  with Dr. Franco Egidi (FE).

  \Cref{ch:2CEST} outlines my primary contributions to the field of relativistic electronic structure theory.
  The first project discussed presents a scalable implementation of non--collinear Kohn--Sham density
  functional theory in a Gaussian basis set. The general protocol for the generalized density variables
  utilized in this work were developed by Frisch \emph{et al} \todo{cite} and Egidi, \emph{et al} \todo{cite},
  through no implementation details were devulged in either of those papers. My primary contribution to this
  work was the implementation of the proposed algorithm and in the formal development of the efficient
  intermediates used for the assembly of the Fock matrix. This work was done in collaboration with Dr. Alessio
  Petrone (AP), where much of the code development was done through interations between him and myself. The
  proof for the adherence of these generalized density variables to the zero-torque theorem was done in 
  collaboration with Shichao Sun. The distributed memory implementation of the numerical intergration
  was written by myself. The efficient intermediates were originally derived by myself, then further refined
  through discussions with AP. Further, the framework within which this algorithm was developed 
  (ChronusQ) has been primarly development by myself to this point. The majority of the tests cases were
  performed by AP and were selected by the two of us.

  The second project discussed in \cref{ch:2CEST} presents an extension of the particle-particle Tamm-Dancoff
  approximation to two-component relativistic Hamiltonians. This project was originally inspired through 
  email discussions with Professor Weitao Yang and is student Dr. Yang Yang at Duke University. They provided
  me with non-relativstic test cases which accelereted the developement of this algorithm in Gaussian. The
  development of this algorithm in Gaussian was written by myself, but much of the development would have 
  been stalled had it not been for thorough discussion with FE at the time. The test cases for
  this algorithm were chosen by FE and ran by myself. Analysis of the reuslts in the context of the
  two-component time-dependnet Hartree-Fock algorithm was done in equal parts by myself and FE.

  \Cref{ch:MOR} presents a novel algorithm for the estimation of the linear absorption spectrum
  by model order reduction. This work was done in extensive collaboration with Dr. Chao Yang 
  and his student Roel van Beeuman (RB) at Lawerence Berkley National Lab. The initial idea
  for applying model order reduction to the absorption spectrum is due to RB. The formal
  development of the final algorithm was due to extensive discussion between RB and myself.
  RB provided a proof-of-concept MATLAB implementation of the algorithm using data provided
  by myself. The MATLAB implementation provided the primary reference from which I wrote the 
  production implementation in ChronusQ using small test systems. The validation of the algorithm
  for larger test cases was done by myself. The full eigendecomposition reference values were
  obtained by RB on the Cori supercomputer using matrices provided by myself.
%
}


%
% end of the preliminary pages
