\chapter{Proof of Zero Torque Theorem Using Generalized Density Variables} \label{apx:ZeroTorque}

In this appendix, we validate that the zero torque theorem for the xc magnetic field is satisfied in the generalized gradient approximation
given the transformation rules of \cref{eq:NCtrans}. 
The local torque does not need to vanish identically at every point in space and this local contribution is required to obtain accurate spin dynamics and a proper time-evolution of the magnetization \cite{Gyorffy01_206403,Gyorffy03_354,Frisch07_125119,Frisch12_2193,Scuseria13_035117,Li17_2591}.
The local torque of the xc magnetic field is the tensor field defined by \cite{Gyorffy01_206403}
\begin{equation}
\label{eq:localT}
\mathscr{T}_i(\vc{r}) = \sum_{j,k = 1}^3 \varepsilon_{ijk} \rho^j(\vc{r}) B_{GGA}^k(\vc{r}), \quad i \in \{1,2,3\},
\end{equation}
where $\varepsilon_{ijk}$ is the rank-3 Levi-Civita tensor and the xc magnetic field is given by
\begin{equation}
\label{eq:xcMagField}
B_{GGA}^k(\vc{r}) = \frac{\delta E^{GGA}[\rho,\nabla\rho]}{\delta \rho^k(\vc{r})} .
\end{equation}
From \cref{eq:localT}, we may define the global torque of the xc magnetic field as
\begin{equation}
\mathscr{T}_i^{global} = \int \mathrm{d}^3 \vc r \text{ } \mathscr{T}_i(\vc{r}) .
\end{equation}
Remark that the set of U variables in \cref{eq:NCtrans} is partitioned by those variables which depend on
$\mathbf{m}(\vc{r})$ and those which depend on $\nabla\mathbf{m}(\vc{r})$. Recognizing that our choice of U variables
are local functions and
applying the Euler--Lagrange formula for functional derivatives to \cref{eq:xcMagField}, we obtain
\begin{align}
B_{GGA}^{k}(\vc{r}) &= \frac{\partial f}{\partial \rho^k(\vc{r})} - \nabla \cdot \frac{\partial f}{\partial \nabla \rho^k(\vc{r})}  = B^{(\rho),k}_{GGA}(\vc r) - B^{(\nabla),k}_{GGA}(\vc r)
\end{align}
where
\begin{align}
&B^{(\rho),k}_{GGA}(\vc r) = \frac{\partial f}{\partial n^+(\vc{r})} \frac{\partial n^+(\vc{r})}{\partial \rho^k(\vc{r})} + 
  \frac{\partial f}{\partial n^-(\vc{r})} \frac{\partial n^-(\vc{r})}{\partial \rho^k(\vc{r})} \label{eq:BxcR}, \\ 
&B^{(\nabla),k}_{GGA}(\vc r)= \nabla \cdot 
  \left(
    \frac{\partial f}{\partial \varphi^{++}(\vc{r})}\frac{\partial \varphi^{++}(\vc r)}{\partial \nabla \rho^k(\vc{r})} + 
    \frac{\partial f}{\partial \varphi^{+-}(\vc{r})}\frac{\partial \varphi^{+-}(\vc r)}{\partial \nabla \rho^k(\vc{r})} + 
    \frac{\partial f}{\partial \varphi^{--}(\vc{r})}\frac{\partial \varphi^{--}(\vc r)}{\partial \nabla \rho^k(\vc{r})}   
  \right) . \label{eq:BxcGR}
\end{align}
We refer the reader to the Appendix of Ref. \cite{Li17_2591} for explicit expressions for the partial derivatives of $\{U^{NC}(\vc{r})\}$
given by \cref{eq:NCAuxVar}.
Thus, the local and global torque expressions may also be similarly as they are linear in $\mathbf{B}_{GGA}(\vc r)$,
\begin{align}
&\mathscr{T}_i(\vc r) = \mathscr{T}^\rho_i(\vc r) + \mathscr{T}^\nabla_i(\vc r),\\
&\mathscr{T}^\rho_i(\vc r) = \sum_{jk}\varepsilon_{ijk} \rho^j(\vc r) B^{(\rho),k}_{GGA}(\vc r), \qquad \mathscr{T}_i^{(\rho), global} = \int \mathrm{d}^3\vc r\text{ } \mathscr{T}_i^{\rho}(\vc{r}) \label{eq:localTRho},\\
&\mathscr{T}^\nabla_i(\vc r) = \sum_{jk}\varepsilon_{ijk} \rho^j(\vc r) B^{(\nabla),k}_{GGA}(\vc r), \qquad \mathscr{T}_i^{(\nabla), global} = \int \mathrm{d}^3\vc r\text{ } \mathscr{T}_i^{\nabla}(\vc{r}) \label{eq:localTG}.
\end{align}

To identify the local and global torque contributions from \cref{eq:BxcR}, we  may consolidate it with the magnetization components of \cref{eq:ZrhoVar},
\begin{equation}
B^{(\rho),k}_{GGA}(\vc r) = \mathcal{Z}^k_{\rho}(\vc r) = \frac{1}{2} \left(\dfrac{\partial f}{\partial n^+(\vc r)} - \dfrac{\partial f}{\partial n^-(\vc r)}\right) 
  \frac{\rho^k(\vc r)}{\vert \mathbf{m}(\vc r) \vert} . \label{eq:BxcR_eval}
\end{equation}
Substituting into \cref{eq:localTRho}, we obtain,
\begin{equation}
\mathscr{T}^\rho_i(\vc r) = \frac{1}{2} \left(\dfrac{\partial f}{\partial n^+(\vc r)} - \dfrac{\partial f}{\partial n^-(\vc r)}\right) 
  \sum_{jk} \frac{\varepsilon_{ijk}\rho^j(\vc r)\rho^k(\vc r)}{\vert\mathbf{m}(\vc r)\vert}  = 0, \quad \Longrightarrow \quad \mathscr{T}_i^{(\rho), global} = 0 \label{eq:torR}
\end{equation}
where we have utilized the fact that $\sum_{jk}\varepsilon_{ijk}\rho^j(\vc r)\rho^k(\vc r) = 0$. Thus the local torque contribution of this term is zero in all space, implying that
its global torque contribution is zero as well.

Similarly, we may identify \cref{eq:BxcGR} with the magnetization components of \cref{eq:ZvarphiVar},
\begin{equation}
B^{(\rho),k}_{GGA}(\vc r) = \sum_\xi \nabla_\xi \mathcal{Z}^k_{\nabla,\xi}(\vc r) \label{eq:BxcGR_eval} .
\end{equation}
However, unlike  \cref{eq:torR}, the torque arising from \cref{eq:BxcGR_eval} is \emph{not} zero in all space. We must them examine its global torque contribution,
\begin{align}
  \mathscr{T}_i^{(\nabla), global} 
    &= \sum_{jk} \sum_\xi \varepsilon_{ijk} \int\mathrm{d}^3\vc r\text{   } \rho^j(\vc r) \text{ }\nabla_\xi \mathcal{Z}^k_{\nabla,\xi}(\vc r) \nonumber \\
    &= -\sum_{jk} \sum_\xi \varepsilon_{ijk} \int\mathrm{d}^3\vc r\text{   } \mathcal{Z}^k_{\nabla,\xi}(\vc r) \left(\nabla_\xi \rho^j(\vc r)\right) .
\end{align}
Here, we have integrated the first line by parts and utilized the fact that the density and its derivatives disappear at the boundary by definition.
Substituting in the expressions from \cref{eq:ZvarphiVar},
\begin{align}
  \mathscr{T}_i^{(\nabla), global} 
    = -\frac{1}{2}\sum_{jk} \sum_\xi \varepsilon_{ijk} \int\mathrm{d}^3\vc r\text{   } 
    \nabla_\xi \rho^j(\vc r)
    &\left(
      \nabla_\xi\rho^S(\vc r) \mathfrak H^k(\vc r)  \left( \dfrac{\partial f}{\partial \varphi^{++}(\vc r)} - \dfrac{\partial f}{\partial \varphi^{--}(\vc r)} \right) 
       \right. +\nonumber\\ & \quad \left. \nabla_\xi\rho^k(\vc r)
      \left(\dfrac{\partial f}{\partial \varphi^{++}(\vc r)} - \dfrac{\partial f}{\partial \varphi^{+-}(\vc r)} + \dfrac{\partial f}{\partial \varphi^{--}(\vc r)} \right)
    \right) \nonumber \\
   % &= -\frac{1}{2}\sum_{jk} \sum_\xi \varepsilon_{ijk} \left(
   %   \int\mathrm{d}^3r\text{   } \left( \dfrac{\partial f}{\partial \varphi^{++}} - \dfrac{\partial f}{\partial \varphi^{--}} \right) 
   %     \nabla_\xi\rho^S \nabla_\xi \rho^j \mathfrak H^k + 
   %   \int\mathrm{d}^3r\text{   } \left(\dfrac{\partial f}{\partial \varphi^{++}} - \dfrac{\partial f}{\partial \varphi^{+-}} + \dfrac{\partial f}{\partial \varphi^{--}} \right)
   %     \nabla_\xi \rho^j \nabla_\xi\rho^k
   % \right) \nonumber \\
    &= 0,
\end{align}
where we have used the following relations
\begin{align}
&\nabla \rho^S(\vc r) \cdot \nabla \rho^j(\vc r) \mathfrak H^k(\vc r) = \nabla \rho^S(\vc r) \cdot \nabla \rho^k (\vc r)H^j(\vc r)  \quad \Longrightarrow \quad
  \sum_{jk} \varepsilon_{ijk} \nabla \rho^S(\vc r) \cdot \nabla \rho^j(\vc r) \mathfrak H^k(\vc r) = 0, \\
& \sum_{jk}\varepsilon_{ijk} \nabla \rho^j(\vc r) \cdot \nabla \rho^k(\vc r) = 0.
\end{align}
 
