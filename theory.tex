\chapter{Theoretical Basis}
\label{ch:Theory}

In this chapter, I will outline the theoretical preliminaries which will serve as the basis
for the subsequent development of relativistic electronic structure theory. Further, this
chapter will serve as the primary source of notation which will be used throughout the
remainder of this work. 

% Quantum Mechanics
\section{Quantum Mechanical Preliminaries}
\label{sec:QM}

\subsection{The Physical Hilbert Space and The Slater Determinant}
\label{sec:SD}

Perhaps the most fundamental axiom of quantum mechanics is in that for every physical
system, there is associated a separable complex Hilbert space, $\hilb{H}$, such that
the vectors of said Hilbert space represent the quantum states 
of the system \cite{VonNeumann55_book}. Such vectors are referred to as wave functions
and the inner product on $\hilb{H}$ is referred to as an expectation value.
The formal structure of $\hilb{H}$ is dictated by the Hamiltonian
for the physical system, $\op{H} : \hilb{H} \rightarrow \hilb{H}$, through the wave equation
\begin{equation}
  \label{eq:QWave}
\op{H}(t) \ket{\Psi(t)} = \ii \partial_t \ket{\Psi(t)}, 
  \qquad \ket{\Psi(t)} \in \hilb H.
\end{equation}
Here, $t$ is the proper time of the quantum system and $\partial_t$ is the partial derivative
of the wave function with respect to time. The explicit dependence on $t$ will be dropped
for brevity in much of the following, except for when its presence is required to
avoid ambiguity.
The nature of $\op H$ for various physical situations and approximations relevant
to this work will be discussed in detail in later sections 
(see \cref{sec:NRH,sec:RELH} for instance), however the mere existence of such
an operator at this stage is sufficient for the subsequent developments.
As the primary focus of this work will be the treatment of the many--body electronic
problem in molecular quantum mechanics, one might remark that the physical system
in question is a \emph{composite} system consisting of many, indistinguishable 
particles (electrons). Denoting the Hilbert space of a composite system consisting
of $N$ particles as $\hilb{H}^N$, we know that as $\hilb{H}^N$ is separable,
it must admit a countable, dense basis \cite{Lee03_book}.
The direct product space consisting of the Hilbert spaces
which describe its constituent parts, namely those spaces which describe a
single particle system: $\hilb{H}^1$, provides a convenient basis for $\hilb{H}^N$ such that
\begin{equation}
  \label{eq:SepHilbert}
  \hilb{H}^N = \Span{\bigotimes_{i = 1}^N \hilb{H}^1}.
\end{equation}
The separability condition of \cref{eq:SepHilbert} is crucial to the following developments
as it allows one to construct a simple basis for vectors in $\hilb{H}^N$
as $N$--fold tensor products of single--particle wave functions which form a countable basis for $\hilb{H}^1$, 
i.e. $\hilb{C} = \set{\ket{\phi_p}}  \subset \hilb{H}^1$ such that 
$\hilb{H}^1 = \Span{\hilb{C}}$. In the following, vectors in $\hilb{H}^1$, in particular elements of $\hilb{C}$,
will be referred to as orbitals.

As the electron is the moiety of interest in this work, vectors in $\hilb{H}^N$ must adhere to certain additional 
criteria in order for them to represent physically realizable wave functions. As electrons are 
fermions, physically relevant elements of $\hilb{H}^N$, which we will denote $\hilb{H}^N_- \subset \hilb{H}^N$, must 
adhere to Fermi--Dirac statistics, namely that they must exhibit anti-symmetric behavior under particle permutation
\cite{Walecka12_book,Schuck04_book}, i.e.
\begin{equation}
  \label{eq:FermiDirac}
  \hilb{P}_{ij} \ket{\Phi^N} = -\ket{\Phi^N}, \qquad \ket{\Phi^N} \in \hilb{H}^N_-,
\end{equation} 
where $\hilb{P}_{ij}$ is the particle permutation operator which interchanges particle $i$ and $j$ 
(for a more thorough discussion on linear operators acting on $\hilb{H}^N$, see \cref{sec:LO}). Remark
that this notion of particle interchange is intimately related to \cref{eq:SepHilbert}. With this additional
constraint, we may construct a basis for $\hilb{H}^N_-$ with elements \todo{find a good reference}
\begin{equation}
  \label{eq:SlaterDet}
  \ket{\Phi^N_I} = \frac{1}{\sqrt{N!}} \sum_{\xi \in S_N(\hilb{K}^N_I)} \Sign{\xi} \text{ } \bigotimes_{i = 1 }^N \ket{\phi_{\xi(i)}},
\end{equation}
where $\hilb{K}_I^N$ is an $N$--element subset of $\hilb{C}$ and $S_N(\hilb{K}_I^N)$ is the symmetric group
of $\hilb{K}_I^N$ which consists of all permutations of its elements; denoted here as permutation
functions $\xi$. $\Sign{\xi} \in \{ \pm 1 \}$ denotes the sign of the permutation and ensures the anti--symmetry of
the overall wave function. \Cref{eq:SlaterDet} introduces a number of concepts with are typically jargonized 
in the quantum chemistry community. In this work, $\hilb{K}_I^N$ will be referred to as an $N$--particle
configuration (or simply a configuration when $N$ is to be understood from the context), and $\ket{\Phi^N_I}$
will be referred to as a Slater determinant. It is important to note that $\ket{\Phi^N_I}$ is completely 
determined by $\hilb{K}_I^N$. The $N$--fold tensor product on the right hand side of \cref{eq:SlaterDet}
if referred to as a Hartree product, and while not a valid fermionic wave function in and of itself,
it provides an important building block for constructing such wave functions and will be the primary
moiety with which we will develop the arithmetic of many body quantum theory. As such, we reserve a shorthand
for the Hartree product constructed for a general set $\set{\ket{\psi_p}}\subset\hilb{H}^1$ as
$ \ket{\psi_1,\psi_2\cdots} \equiv \ket{\psi_1} \otimes \ket{\psi_2} \otimes \cdots$.

As a basis for $\hilb{H}^N_-$, any vector $\ket{\Psi_n^N} \in \hilb{H}^N_-$ may be written as \cite{Ostlund12_book}
\begin{equation}
\label{eq:SDBasis}
\ket{\Psi_n^N} = \sum_I D^n_I \ket{\Phi^N_I},
\end{equation}
where $D^n_I = \inner{\Phi^N_I}{\Psi^N_n}\in\mathbb C$ is the complex expansion coefficient of the $I$-th configuration in the overall wave function,
and $I$ runs over all unique $N$--particle Slater determinants which may be constructed from $\hilb{C}$.
The fact that the set of all Slater determinants forms a basis for $\hilb{H}^N_-$
serves as the primary foundation for the majority of approximate quantum mechanical methods regarding molecular
systems. In the following, we will assume both $\ket{\Phi^N_I}$ and the elements of $\hilb{C}$ are orthonormal
with respect to the metric on their respective Hilbert spaces.

While the Hilbert space representation of the wave function is the most illuminating description for the development of
general quantum mechanical theory, it is often advantageous from the perspective of practical calculations that one
projects the vectors of the Hilbert space onto a convenient basis. 
To this end, we consider a specific single particle basis, $\set{\ket{\vc{r},\sigma} = \ket{\vc{r}}\otimes\ket{\sigma}}$, 
which consists of the simultaneous eigenfunctions
of both the position and $z$--spin operators, denoted $\op{\vc{r}}$ and $\op{S}_z$,
such that
\begin{subequations}
\begin{align}
  \op{\vc{r}}\ket{\vc{r},\sigma} &= \ket{\vc{r},\sigma}\vc{r}, \qquad \vc{r}\in\mathbb{R}^3, \\
  \op{S}_z\ket{\vc{r},\sigma}    &= \ket{\vc{r},\sigma}\sigma, \qquad \sigma \in \set{\pm\frac{1}{2}}.
\end{align}
\end{subequations}
Here, we have denoted the particle's position and $z$--axis spin projection as $\vc{r}$ and $\sigma$, respectively.
Namely, if $\hilb{H}^1$ represents a spin--1/2 fermion, $\set{\ket{\vc{r},\sigma}}$ forms a complete basis for $\hilb{H}^1$ 
and admits the following orthonormality  condition on the $\hilb{H}^1$ inner product,
\begin{equation}
  \label{eq:SpinorInner}
  \inner{\vc{r}',\sigma'}{\vc{r},\sigma} = \delta^3(\vc{r} - \vc{r}')\delta_{\sigma\sigma'},
\end{equation}
where $\delta^3$ and $\delta_{\sigma\sigma'}$ are the Dirac delta function and Kronecker delta tensor, respectively.
As $\set{\ket{\vc{r},\sigma}}$ is continuous, i.e. its spectrum is continuous, its cardinality is uncountable. Thus,
its utility does not manifest as it does in the context of countable bases, such as is required by \cref{eq:SlaterDet},
but rather in the fact that it allows for the casting of inner products on $\hilb{H}^1$ as integrals through the resolution
of the identity on $\hilb{H}^1$ via
\begin{equation}
  \label{eq:H1Identity}
  \op{1}_1 = \sum_\sigma \int_{\mathbb R ^3} \ket{\vc{r},\sigma}\bra{\vc{r},\sigma} \dd^3\vc{r}.
\end{equation}
As such, for arbitrary $\ket{\phi},\ket{\phi'}\in\hilb{H}^1$ we may cast the inner product as
\begin{equation}
\inner{\phi}{\phi'} = \sum_\sigma \int_{\mathbb R ^3} \phi^*(\vc{r},\sigma) \phi'(\vc{r},\sigma) \dd^3\vc{r},
\end{equation}
where we have defined
\begin{equation}
  \label{eq:SpinorOrbital}
  \inner{\vc{r},\sigma}{\phi} \equiv \phi(\vc{r},\sigma), \quad \mathrm{s.t.} \quad \phi : \mathbb F \mapsto \mathbb C,
\end{equation}
and $\mathbb F = \mathbb R^3 \times \set{\pm1/2}$. 
\Cref{eq:SpinorOrbital}, as a projection onto
an element of a product space, may be further separated onto the spin basis, $\set{\alpha,\beta}$,
\begin{align}
  \phi(\vc{r},\sigma) = \phi^\alpha(\vc{r})\alpha(\sigma) + \phi^\beta(\vc{r})\beta(\sigma),
\end{align}
such that
\begin{align}
  &\alpha(\sigma) = \begin{cases} 1 & \sigma = +\frac{1}{2} \\ 0 & \sigma = -\frac{1}{2} \end{cases}, \\ 
  &\beta(\sigma)  = \begin{cases} 0 & \sigma = +\frac{1}{2} \\ 1 & \sigma = -\frac{1}{2} \end{cases}.
\end{align}
In general, single particle wave functions of the form \cref{eq:SpinorOrbital} will be referred to as spinor orbitals, or simply spinors.
For brevity in the following, we will denote $\ket{\vc{x}} \equiv \ket{\vc{r},\sigma}$ such that
\begin{subequations}
\begin{align}
\int_{\mathbb F} f(\vc{x}) \dd^4 \vc{x} &\equiv \sum_\sigma \int_{\mathrm{R}^3} f(\vc{r},\sigma) \dd^3 \vc{r},\\
\delta^4(\vc{x} - \vc{x}') &\equiv \delta^3(\vc{r} - \vc{r}')\delta_{\sigma\sigma'}
\end{align}
\end{subequations}

As a basis for $\hilb{H}^1$, we may construct a basis for $\hilb{H}^N$ through $N$--fold tensor products 
of the elements of $\set{\ket{\vc{x}}}$ via \cref{eq:SepHilbert}. Denoting $\ket{\vc{x}_i}$ as
a specific element of $\set{\ket{\vc{x}}}$, the vectors
\begin{equation}
  \label{eq:MBSpinorBasis}
  \ket{\vc{x}_1,\vc{x}_2,\ldots,\vc{x}_N} = \bigotimes_{i = 1}^N \ket{\vc{x}_i}
\end{equation}
form a basis form a basis for $\hilb{H}^N$. Extending \cref{eq:SpinorInner,eq:H1Identity} in a similar manner, we may state
\begin{equation}
  \inner{\vc{x}'_1,\vc{x}'_2,\ldots,\vc{x}'_N}{\vc{x}_1,\vc{x}_2,\ldots,\vc{x}_N} = \prod_{i=1}^N \delta^3(\vc{r}'_i - \vc{r}_i)\delta_{\sigma_i\sigma'_i},
\end{equation}
\begin{equation}
  \op{1}_N = \idotsint_{\mathbb F} \bigotimes_{i=1}^N \ket{\vc{x}_i} \bra{\vc{x}_i} \dd^4\vc{x}_i,
\end{equation}
such that for $\ket{\Phi^N},\ket{\Psi^N} \in \hilb{H}^N$,
\begin{align}
&\inner{\Phi^N}{\Psi^N} = \nonumber \\&\quad \idotsint_{\mathbb F} 
  \Phi^{N*}(\vc{x}_1,\vc{x}_2,\ldots,\vc{x}_N) \Psi^{N}(\vc{x}_1,\vc{x}_2,\ldots,\vc{x}_N) \dd^4\vc{x}_1 \cdots \dd^4\vc{x}_N,
\end{align}
where we have denoted
\begin{equation}
  \label{eq:SpinorWfn}
  \inner{\vc{x}_1,\vc{x}_2,\ldots,\vc{x}_N}{\Phi^N} \equiv \Phi^N(\vc{x}_1,\vc{x}_2,\ldots,\vc{x}_N).
\end{equation}
In this work, many--body wave functions of the form \cref{eq:SpinorWfn} will be referred to as spinor wave functions.
The utility of such as basis expansion manifests in the context of Slater determinants in that as
a direct consequence of \cref{eq:SlaterDet,eq:MBSpinorBasis,eq:SpinorOrbital}, we may express
\begin{equation}
  \label{eq:SlaterDetSpace}
  \Phi_I^N(\vc{x}_1,\vc{x}_2,\ldots,\vc{x}_N) = \frac{1}{\sqrt{N!}} \sum_{\xi \in S_N(\hilb{K}^N_I)} \Sign{\xi} \text{ } 
    \prod_{i = 1 }^N \phi_{\xi(i)}(\vc{x}_i).
\end{equation}
Unlike the tensor product definition of \cref{eq:SlaterDet}, the expression of the Slater determinant in will prove to be of 
much of practical utility due to the fact that the spinor orbital basis, as a set of complex valued functions, is commutative.

\subsection{Representation of Linear Operators and Expectation Values}
\label{sec:LO}

Fundamental to the formulation of any quantum mechanical theory is the identification of linear operators on $\hilb{H}^N$
which represent the physical observables of the system. In this work we will refer to such observables as properties. A
more precise identification of the operators relevant to this work will be presented later 
(see \cref{sec:NRH,sec:RELH,sec:SCLMI} for instance), however in this section we will focus on the general presentation
of these operators and how they will typically manifest in the context of \cref{eq:SepHilbert}. 

In this work, linear operators which act on $\hilb{H}^N$, $\op{O}^N : \hilb{H}^N \mapsto \hilb{H}^N$, will be referred to as 
$N$--particle operators. As was the case for the state vectors of $\hilb{H}^N$, the general description of $\op{O}^N$
as an operator on a Hilbert space is indeed the most illuminating treatment for general manipulations of quantum mechanical
operators which is independent of coordinate projection. However, it will often be the case that we must examine the
projection of these operators onto a coordinate space in order to perform practical calculations. To demonstrate this, we examine
the action of an operator $\op{O}^1$ on the spinor basis $\set{\ket{\vc{x}}}$ from \cref{sec:SD} through the identity resolvent in
\cref{eq:H1Identity},
\begin{align}
  \op{O}^1 = \op{1}_1 \op{O}^1 \op{1}_1 = \iint_{\mathbb F} 
    \ket{\vc{x}} \innerop{\vc{x}}{\op{O}^1}{\vc{x}'}\bra{\vc{x}'} \dd^4\vc{x} \dd^4\vc{x}',
\end{align}
such that for $\ket{\psi},\ket{\chi} \in \hilb{H}^1$,
\begin{align}
  \label{eq:SpinorOpInner1}
  \innerop{\psi}{\op{O}^1}{\chi} &= \iint_{\mathbb F}
    \psi^*(\vc{x}) \innerop{\vc{x}}{\op{O}^1}{\vc{x}'} \chi({\vc{x}'}) \dd^4\vc{x} \dd^4\vc{x}' \nonumber \\
  &= \int_{\mathbb R^3} 
    \begin{bmatrix}
      \psi^\alpha(\vc{r}) \\
      \psi^\beta(\vc{r})
    \end{bmatrix}^\dagger
    \begin{bmatrix}
      \op{O}^{1,\alpha\alpha}({\vc{r}}) & \op{O}^{1,\alpha\beta}({\vc{r}}) \\
      \op{O}^{1,\beta\alpha} ({\vc{r}}) & \op{O}^{1,\beta\beta} ({\vc{r}})
    \end{bmatrix}
    \begin{bmatrix}
      \chi^\alpha(\vc{r}) \\
      \chi^\beta(\vc{r})
    \end{bmatrix} \dd^3\vc{r},
\end{align}
where 
\begin{equation}
  \label{eq:LocalSpinorOp1}
\innerop{\vc{x}}{\op{O}^1}{\vc{x}'}  \equiv \innerop{\vc{r},\sigma}{\op{O}^1}{\vc{r}',\sigma'} = \delta^3(\vc{r} - \vc{r'}) \op{O}^{1,\sigma\sigma'}(\vc{r}').
\end{equation}
%For the remainder of this work, we will refer to the projections $\op{O}^{1,\sigma\sigma'}(\vc{r})$ as 
%spinor representations of $\op{O}^1$, or simply spinor operators. It is important to note that spinor representations
%of $N$--particle operators are still \emph{operators}, i.e. they still continue to act to the right
%to complete their operation. The utility of \cref{eq:LocalSpinorOp1} is in that it allows one to
%form operators compatible with spinor representations of the wave function as opposed to the general, 
%abstract Hilbert space definition. 
%
This notion may be generalized to $N$--particle operators in a similar manner through resolution of $\op{1}_N$,
\begin{align}
  \op{O}^N &= \op{1}_N \op{O}^N \op{1}_N 
  = \idotsint_{\mathbb F}
    \dd^4\vc{x}_1 \cdots   \dd^4\vc{x}_N
    \idotsint_{\mathbb F}
    \dd^4\vc{x}'_1 \cdots  \dd^4\vc{x}'_N \times \nonumber\\
    &\qquad\ket{\vc{x}_1,\ldots,\vc{x}_N} \innerop{\vc{x}_1,\ldots,\vc{x}_N}{\op{O}^{N}}{\vc{x}'_1,\ldots,\vc{x}'_N}\bra{\vc{x}'_1,\ldots,\vc{x}'_N} 
\end{align}
such that for arbitrary $N$--body Hartree products constructed from $\set{\ket{\psi_p}},\set{\ket{\chi_p}} \subset \hilb{H}^N$
\begin{align}
  \label{eq:SpinorOpInnerN}
  &\innerop{\psi_1,\psi_2,\ldots,\psi_N}{\op{O}^N}{\chi_1,\chi_2,\ldots,\chi_N} =  
  \sum_{\sigma_1\cdots\sigma_N}\sum_{\tau_1\cdots\tau_N} \idotsint_{\mathbb R^3} 
    \dd^3\vc{r}_1\cdots\dd^3\vc{r}_N  \nonumber \times \\ &\qquad \qquad
    \psi_1^{\sigma_1*}(\vc{r}_1) \cdots \psi_N^{\sigma_N*}(\vc{r}_N) ~ \op{O}^{N,\sigma_1\tau_1\cdots\sigma_N\tau_N}(\vc{r}_1,\ldots,\vc{r}_N) ~ 
    \chi_1^{\tau_1}(\vc{r}_1) \cdots \chi_N^{\tau_N}(\vc{r}_N),
\end{align}
where
\begin{equation}
  \label{eq:LocalSpinorOpN}
  \innerop{\vc{x}_1,\ldots,\vc{x}_N}{\op{O}^{N}}{\vc{x}'_1,\ldots,\vc{x}'_N} = 
    \left(\prod_{i=1}^N\delta^3(\vc{r}_i - \vc{r}'_i)\right) 
    \op{O}^{1,\sigma_1\sigma_1'\cdots\sigma_N\sigma_N'}(\vc{r}_1',\ldots,\vc{r}_N').
\end{equation}
For brevity, spinor representations of $N$ particle operators will 
often be interpreted as rank--2$N$ tensors in the basis of spin eigenfunctions, 
\begin{equation}
\label{eq:SpinorOpN}
\spop{O}^N (\vc{x}_1,\ldots,\vc{x}_N) \equiv \sum_{\sigma_1\cdots\sigma_N}\sum_{\sigma'_1\cdots\sigma'_N}  \op{O}^{N,\sigma_1\sigma_1'\cdots\sigma_N\sigma_N'}(\vc{r}_1,\ldots,\vc{r}_N) 
  \otimes \bigotimes_{i=1}^N \vc{e}_{\sigma_i} \otimes \vc{e}_{\sigma_i'}, 
\end{equation}
where 
\begin{align}
\vc{e}_\alpha = \begin{bmatrix} 1 \\ 0 \end{bmatrix}, \qquad \vc{e}_\beta = \begin{bmatrix} 0 \\ 1 \end{bmatrix}. 
\end{align}
The product operation of operators denoted $\spop{O}^N(\vc{x}_1,\ldots,\vc{x}_N)$ and their action onto spinor wave functions 
(or more specifically, Hartree products) will be a rank-$N$ tensor contraction over spin indices as depicted in \cref{eq:SpinorOpInner1} 
and more generally in \cref{eq:SpinorOpInnerN}. 
For the remainder of this work, we will refer to the projections $\spop{O}^N(\vc{x}_1,\ldots,\vc{x}_N)$ as 
spinor representations of $\op{O}^N$, or simply spinor operators when appropriate. It is important to note that spinor representations
of $N$--particle operators are still \emph{operators}, i.e. they still continue to act to the right
to complete their operation. The utility of \cref{eq:LocalSpinorOp1,eq:LocalSpinorOpN} is in that it allows one to
form operators compatible with spinor representations of the wave function as opposed to the general, 
abstract Hilbert space definition. 
In the following developments, it will be useful to examine \cref{eq:SpinorOpN} under a change of basis from the Kronecker products of
$\vc{e}_\alpha$ and $\vc{e}_\beta$ to that of the Pauli matrices,
\begin{equation}
  \label{eq:SpinorOpPauli}
  \spop{O}^N (\vc{x}_1,\ldots,\vc{x}_N) = \sum_{K_1\cdots K_N}\op{O}^{N,K_1\cdots K_N}(\vc{r}_1,\ldots,\vc{r}_N) \otimes \bigotimes_{i = 1}^N \pauli{K_i}
\end{equation}
where 
\begin{equation}
\label{eq:PauliMats}
\pauli{0} = \begin{bmatrix} 1 & 0 \\ 0 & 1 \end{bmatrix}, \quad
\pauli{1} = \begin{bmatrix} 0 & 1 \\ 1 & 0 \end{bmatrix}, \quad
\pauli{2} = \begin{bmatrix} 0 & -\ii \\ \ii & 0 \end{bmatrix}, \quad
\pauli{3} = \begin{bmatrix} 1 & 0 \\ 0 & -1 \end{bmatrix}.
\end{equation}
An explicit derivation for the form of the general rank--$2N$ transformation is given in \cref{apx:SpinorOp}. Here we state the special case of
one particle spinor operators as it will provide the basis for many of the manipulations in subsequent developments,
\begin{subequations}
\begin{align}
&\op{O}^{1,0}(\vc{r}) = \frac{1}{2} \left( \op{O}^{1,\alpha\alpha}(\vc{r}) + \op{O}^{1,\beta\beta}(\vc{r}) \right),\\
&\op{O}^{1,1}(\vc{r}) = \frac{1}{2} \left( \op{O}^{1,\alpha\beta}(\vc{r}) + \op{O}^{1,\beta\alpha}(\vc{r}) \right),\\
&\op{O}^{1,2}(\vc{r}) = \frac{\ii}{2} \left( \op{O}^{1,\alpha\beta}(\vc{r}) - \op{O}^{1,\beta\alpha}(\vc{r}) \right),\\
&\op{O}^{1,3}(\vc{r}) = \frac{1}{2} \left( \op{O}^{1,\alpha\alpha}(\vc{r}) - \op{O}^{1,\beta\beta}(\vc{r}) \right).
\end{align}
\end{subequations}



Despite to the fact that $\hilb{H}^N$ admits a basis of a direct product space consisting of $N$ single particle
Hilbert spaces, $N$--particle operators need not carry the same structure, 
i.e. in general, these operators need not exist solely as direct products of operators on $\hilb{H}^1$. This is not to say
that $N$--particle operators \emph{cannot} adopt a product structure, just that it is not a requirement, and indeed
if often not the case. However, the notion that some $N$--particle operators \emph{can} adopt a product structure indicates the need
to describe the action of $M$--particle operators on $\hilb{H}^N$ (with $N \geq M$) while leaving $N-M$ particles unchanged.
To demonstrate this state of affairs, it is convenient to examine the action of such an operator, $\op{O}^M(i,j,\ldots)$, on a 
spinor wave function of the form \cref{eq:SlaterDetSpace},
\begin{align}
  \label{eq:MOpSpace}
  &\spop{O}^M(\vc{x}_i,\vc{x}_j,\ldots) \Phi_I^N(\vc{x}_1,\vc{x}_2,\ldots,\vc{x}_N) = \nonumber \\ &\qquad \quad
    \frac{1}{\sqrt{N!}} \sum_{\xi \in S_N(\hilb{K}^N_I)} \Sign{\xi} \text{ } 
    \left(\spop{O}^M(\vc{x}_i,\vc{x}_j,\ldots) \phi_{\xi(i)}(\vc{x}_i)\phi_{\xi(j)}(\vc{x}_j)\cdots\right)
    \prod_{k \neq (i,j,\ldots) }^{N-M} \phi_{\xi(k)}(\vc{x}_k),
\end{align}
where $(i,j,\ldots)$ is an $M$--element tuple specifying the subset of particles upon which it acts.
In this work, we will typically be concerned with operators which act on no more than two particles at a time.

It will be often the case that $N$--particle operators may be expressed as sums over one and
two particle operators. In this context , the realization of \cref{eq:MOpSpace} is of exceptional utility in that,
for an expectation value involving a particular Slater determinant described by configuration $\mathcal{K}_I^N$
\cite{Ostlund12_book},
\begin{subequations}
  \label{eq:SlaterCondon}
\begin{align}
  &\innerop{\Phi^N_I}{\sum_i\op{O}^1(i)}{\Phi^N_I} = \sum_{i\in\hilb{K}^N_I} O^1_{ii},\\
  &\innerop{\Phi^N_I}{\sum_{i\neq j} \op{O}^2(i,j)}{\Phi^N_I} = 
    \sum_{i\neq j \in\hilb{K}^N_I} O^2_{ijij} - O^2_{ijji},
\end{align}
\end{subequations}
where for the $\hilb{H}^1$ basis $\mathcal{C} = \set{\ket{\phi_p}}$ from the previous section,
\begin{subequations}
  \label{eq:DiracInts}
\begin{align}
  &O^1_{pq} \equiv \innerop{\phi_p}{\op{O}^1(1)}{\phi_q} = 
    \int_{\mathbb F} \phi_p^*(\vc{x}_1) \spop{O}^1(\vc{x}_1) \phi_q(\vc{x}_1) \dd^4 \vc{x}_1,\\
  &O^2_{pqrs} \equiv \innerop{\phi_p, \phi_q }{\op{O}(1,2)}{\phi_r, \phi_s} = \nonumber \\ &\qquad
    \iint_{\mathbb F} 
      \phi_p^*(\vc{x}_1) \phi_q^*(\vc{x}_2) \spop{O}^2(\vc{x}_1,\vc{x}_2) 
      \phi_r(\vc{x}_1) \phi_s(\vc{x}_2) \dd^4 \vc{x}_1 \dd^4 \vc{x}_2.
\end{align}
\end{subequations}
Namely, $M$--particle on operators on $\hilb{H}^N$ may be represented as rank--$2M$ tensors on $\hilb{H}^1$; a fact which will
be used extensively in the following developments.


\co{Consider reworking, not general to MC wave functions and should be for talk on response, etc}
In the following, it will be useful to define the density operator,
\begin{equation}
  \label{eq:TensorDensity}
\op{\vc{\rho}}(\vc{r};\vc{x}_i)  = \begin{bmatrix} \op{\rho}^0(\vc{r};\vc{x}_i) \\ \op{\rho}^1 (\vc{r};\vc{x}_i)\\\op{\rho}^2(\vc{r};\vc{x}_i) \\ \op{\rho}^3(\vc{r};\vc{x}_i) \end{bmatrix},
\end{equation}
such that
\begin{equation}
\op{\rho}^K(\vc{r};\vc{x}_i) = \delta^3(\vc{r} - \vc{r}_i) \otimes \pauli{K}.
\end{equation}
Action of $\sum_i \op{\vc{\rho}}(\vc{r};\vc{x}_i)$ generates a tensor field on $\mathbb R^3$, 
%such that for some pair
%$\ket{\Psi^N_m},\ket{\Psi^N_n} \in \hilb{H}^N$,
%\begin{equation}
%\vc{\rho}_{mn}(\vc{r}) = \innerop{\Psi^N_m}{\sum_i \op{\vc{\rho}}(\vc{r};\vc{x}_i)}{\Psi^N_n} = 
%\begin{bmatrix} {\rho_{mn}^0}(\vc{r}) \\ {\rho_{mn}^1} (\vc{r})\\{\rho_{mn}^2}(\vc{r}) \\ {\rho_{mn}^3}(\vc{r}) \end{bmatrix},
%\end{equation}

Applying \cref{eq:SlaterCondon,eq:DiracInts}, 
for a particular Slater determinant we obtain,
\begin{equation}
\vc{\rho}_I(\vc{r}) = \innerop{\Phi_I^N}{\sum_i \op{\vc{\rho}}(\vc{r};\vc{x}_i)}{\Phi_I^N} = 
\begin{bmatrix} {\rho_I^0}(\vc{r}) \\ {\rho_I^1} (\vc{r})\\{\rho_I^2}(\vc{r}) \\ {\rho_I^3}(\vc{r}) \end{bmatrix},
\end{equation}
where
\begin{subequations}
\begin{align}
  &\rho_I^0(\vc{r}) = \frac{1}{2}   \sum_{i\in\hilb{K}^N_I}\left(\phi^{\alpha*}_i(\vc{r})\phi^{\alpha}_i(\vc{r}) + \phi^{\beta*}_i(\vc{r})\phi^{\beta}_i(\vc{r}) \right), \\
  &\rho_I^1(\vc{r}) = \frac{1}{2}   \sum_{i\in\hilb{K}^N_I}\left(\phi^{\alpha*}_i(\vc{r})\phi^{\beta}_i(\vc{r}) + \phi^{\beta*}_i(\vc{r})\phi^{\alpha}_i(\vc{r}) \right), \\
  &\rho_I^2(\vc{r}) = \frac{\ii}{2} \sum_{i\in\hilb{K}^N_I}\left(\phi^{\alpha*}_i(\vc{r})\phi^{\beta}_i(\vc{r}) - \phi^{\beta*}_i(\vc{r})\phi^{\alpha}_i(\vc{r}) \right), \\
  &\rho_I^3(\vc{r}) = \frac{1}{2}   \sum_{i\in\hilb{K}^N_I}\left(\phi^{\alpha*}_i(\vc{r})\phi^{\alpha}_i(\vc{r}) - \phi^{\beta*}_i(\vc{r})\phi^{\beta}_i(\vc{r}) \right). 
\end{align}
\end{subequations}
In this work, we define $\rho_I^0(\vc{r})$ as the \emph{scalar density} and 
$\vc{m}_I(\vc{r}) = \set{ \rho_I^1(\vc{r}), \rho_I^2(\vc{r}), \rho_I^3(\vc{r})}$ as
the \emph{magnetization density} for the Slater determinant $I$.
These definitions
prove to be of great practical utility
in the case where $\op{O}^1(i)$ does not contain differential operators, i.e. its spatial component is scalar
multiplicative. In such cases, expectation values in the $I$-th Slater determinant admit a simple form,
\begin{equation}
  \label{eq:DensityInner}
  \innerop{\Phi^N_I}{\sum_i\op{O}^1(i)}{\Phi^N_I} = 2\sum_K \int_{\mathbb R^3} \op{O}^{1,K}(\vc{r}) \rho^K(\vc{r}) \dd^3\vc{r}.
\end{equation}






\subsection{Second Quantization}
\label{sec:SQ}

While \cref{eq:SlaterDet,eq:SlaterDetSpace} provide the basic structure for the many--body fermionic wave function, its explicit form
is a bit unwieldy and thus its use for practical calculation of moieties such as expectation values more complicated than \cref{eq:SlaterCondon} 
is somewhat limited.
To this end we introduce a formalism known as second quantization, or the occupation number formalism \cite{Walecka12_book,Schuck04_book,Ostlund12_book},
which aims to greatly simplify the construction and manipulation of anti-symmetric wave functions such as those in 
\cref{eq:SlaterDet}. The primary hallmark of second quantization is in the representation of a Slater determinant
as an array of integers known as \emph{occupation numbers}, denoted $\set{n_p^I}$, indicate the inclusion (or \emph{occupation}) of elements
of $\hilb{C}$ in the configuration which describes the $I$-th determinant. As such, the length of said array is $\vert \hilb{C} \vert$
and the sum of its elements is the number of particles in the system for a representation of $\hilb{H}^N$. Due to the fact that electrons are fermions,
there is a further restriction on the possible values of the occupations numbers due to the Pauli exclusion principle,
namely that a particle can be occupied ($n^I_p=1$) or unoccupied ($n^I_p=0$), i.e. two electrons cannot occupy the same orbital. 
For the basis $\hilb{C}$ we may construct a Hilbert space, $\hilb{F}$, as a product of single particle occupations, denoted $\ket{n_p}$, such that \cite{Walecka12_book}
\begin{align}
&\ket{n_1,n_2,\ldots,n_{\vert\mathcal{C}\vert}} = \bigotimes_{i=1}^{\vert \mathcal{C} \vert} \ket{n_i}, 
  \quad \ket{n_1,n_2,\ldots,n_{\vert\mathcal{C}\vert}} \in \hilb{F},
\end{align}
where we may define inner and outer products as
\begin{subequations}
  \label{eq:OccOrthoComp}
\begin{align}
&\inner{n_1,n_2,\ldots,n_{\vert\mathcal{C}\vert}}{n'_1,n'_2,\ldots,n'_{\vert\mathcal{C}\vert}} = \prod_{i = 1}^{\vert\mathcal{C}\vert} \delta_{n_i n_i'},\\
&\sum_{n_1n_2\cdots} \ket{n_1,n_2,\ldots,n_{\vert\mathcal{C}\vert}}\bra{n_1,n_2,\ldots,n_{\vert\mathcal{C}\vert}} = \op{1}_\hilb{F}.
\end{align}
\end{subequations}
Here, $\op{1}_\hilb{F}$ is the identity operator on $\hilb{F}$. $\hilb{F}$ is commonly referred to as the Fock space.
%To avoid confusion in the subsequent developments, remark that operators on $\hilb{H}^N$ 
%\emph{do not} act on $\hilb{F}$, but rather may act indirectly through the injective map
%$\hilb{H}^N \rightarrow \hilb{F}$.

One may construct representations of Slater determinants (\cref{eq:SlaterDet}) in a consistent manner in this formalism through
the introduction of two sets of operators, $\set{ c_p^\dagger : \hilb{F} \rightarrow \hilb{F}}$ and $\set{ c_p : \hilb{F} \rightarrow \hilb{F} }$, which are referred to as creation and annihilation operators 
respectively, and the notion of a zero particle occupation vector known as the physical vacuum, $\Vac \in \hilb{F}$, such that
\begin{subequations}
  \label{eq:SeQuantAction}
\begin{align}
  &\Vac = \ket{[0_1, 0_2, \ldots, 0_{\vert C \vert}]}, \\
  &c_p^\dagger \ket{[n_1, n_2, \ldots, n_p, \ldots, n_{\vert C \vert}]} = \frac{\delta_{n_p0}}{\sqrt{N+1}} \ket{[n_1, n_2, \ldots, 1_p, \ldots, n_{\vert C \vert}]},\\
  &c_p \ket{[n_1, n_2, \ldots, n_p, \ldots, n_{\vert C \vert}]} = {\delta_{n_p1}}{\sqrt{N}} \ket{[n_1, n_2, \ldots, 0_p, \ldots, n_{\vert C \vert}]}.
\end{align}
\end{subequations}
Here, $N$ is the number of particles in the occupation vector \emph{before} action. To ensure anti-symmetry in the representation of
the wave function, we may specify the following commutation relations
\begin{subequations}
  \label{eq:SeQuantComm}
\begin{align}
  &[c_p^\dagger, c_q^\dagger]_+ = 0,\\
  &[c_p, c_q]_+ = 0,\\
  &[c_p, c_q^\dagger]_+ = \delta_{pq},
\end{align}
\end{subequations}
where $[\cdot,\cdot]_+$ is the anti--commutator. The action of products consisting on any number of these 
operators may be derived inductively from \cref{eq:SeQuantAction,eq:SeQuantComm}. Such products are often referred to as
strings. As such, we may recast \cref{eq:SlaterDet} as a string of creation operators,
\begin{equation}
  \label{eq:SeQuantSlaterDet}
  \ket{\Phi^N_I} \mapsto \ket{\Phi^N_I}_\hilb{F}  = \left(\prod_{i \in \hilb{K}^N_I} c_i^\dagger\right) \Vac,
\end{equation}
where $\ket{\Phi^N_I}_\hilb{F} \in \hilb{F}$ is the occupation vector representation of $\ket{\Phi^N_I}$.



There are a number of remarkable simplifications that arise from the adoption of second quantization
for the practical treatment of many--body quantum theory. The first that we will discuss here
relates back to the completeness of the basis of Slater determinants in the construction
of many--body fermionic wave functions, i.e. all many body wave functions may be constructed 
as a linear combination of all of the unique Slater determininants which may be constructed
from $\mathcal{C}$ (\cref{eq:SDBasis}), given that $\mathcal{C}$ is a complete basis for $\hilb{H}^1$. 
To this end, we introduce the notion of an $N$--particle reference
determinant which will be denoted $\ket{0^N}$ and is described by some configuration $\hilb{K}^N_0$.
$\hilb{K}^N_0$ will be referred to as the \emph{occupied} space (denoted $\hilb{O} = \hilb{K}^N_0$), 
and elements of the occupied space will be denoted with a subscript $i,j,k,\dots$,  i.e.
$\ket{\phi_i} \in \hilb{O}$. Similarly, we will refer the elements of $\hilb{C}$ which are 
not in $\hilb{O}$ as the unoccupied (or \emph{virtual}, denoted $\hilb{V} = \hilb{C} \setminus \hilb{O}$) space and elements of this space
will be denoted with a subscript $a,b,c,\ldots$, i.e. $\ket{\phi_a} \in \hilb{V}$. Subscripts $p,q,r,\ldots$
will remain as the notation for a general element of $\hilb{C}$ from either $\hilb{O}$ or $\hilb{V}$.
Clearly, all $M$ particle Slater determinants (with $M$ not necessarily equal to $N$) which may be constructed from $\hilb{C}$ may be constructed 
by constructing a configuration from $N_o$ elements of $\hilb{O}$ and $N_v$ elements of $\hilb{V}$
such that $N_o + N_v = M$. Thus, logically, it follows that all $M$ particle determininants may be
constructed by removing $ n_o = \min{(N-M,0)} + P $ orbitals from $\hilb{O}$ (with $P\in[0,\max{(N,N-M)}]$) and replacing them $ n_v = \max{(M-N ,0)} + P$ orbitials
from $\hilb{V}$. In the language of second quantization, the act of removing and replacing orbitals 
is made simple through the introduction of so called transition operators
\begin{align}
\left(\ket{0^N}_{ijk\cdots}^{abc\cdots}\right)_\hilb{F} = \tau_{ijk\cdots}^{abc\cdots} \ket{0^N}_\hilb{F},
\end{align}
where
\begin{align}
\tau_{ijk\cdots}^{abc\cdots} = \left(c_a^\dagger c_b^\dagger c_c^\dagger \cdots\right) \left( c_i c_j c_k \cdots\right).
\end{align}
Upon action, $\ket{0^N}_{ijk\cdots}^{abc\cdots}$ is an $ M = N - n_o + n_v $ particle Slater determinant built from removing
orbitals $i,j,k,\ldots$ from $\ket{0^N}$ and replacing them with orbitals $a,b,c,\ldots \in \hilb{V}$. The set of all
$\tau$ operators generates all $M$--particle Slater determinants from the $N$--particle reference \todo{cite Dalgaard}.
It is worth noting at this point that due to that anticommutation relations of \cref{eq:SeQuantComm}, $\tau^{abc\cdots}_{ijk\cdots}$
is anti symmetric with respect to permutation of adjacent creation and annihilation operators, i.e.
\begin{equation}
  \label{eq:AntisymmTau}
  \tau^{abc\cdots}_{ijk\cdots} = - \tau^{acb\cdots}_{ijk\cdots} = - \tau^{abc\cdots}_{jik\cdots} = \tau^{acb\cdots}_{jik\cdots}.
\end{equation} 
Thus, necessarily, transition operators with repeated indicies are zero, $\tau^{aab\cdots}_{ijk\cdots} = \tau^{abc\cdots}_{iij\cdots} = 0$.
Within this ansatz for constructing arbitrary Slater determinants from a reference, we may recast \cref{eq:SDBasis} 
as
\begin{equation}
  \label{eq:CIExp}
  \ket{\Psi^N_n} = \left(t^n\right)_0\ket{0^N} + \sum_{ai} \left(t^n\right)_i^a \ket{0^N}_i^a + 
                   \sum_{abij} \left(t^n\right)_{ij}^{ab} \ket{0^N}_{ij}^{ab} + \sum_{abcijk} \left(t^n\right)_{ijk}^{abc} \ket{0^N}_{ijk}^{abc} + \cdots,
\end{equation}
where the transition amplitudes, $\vc{t}^n$, have taken the place of the $D^n_I$ amplitudes of the original expansion. Based on this
construction, the components of $\vc{t}^n$ must also constitute an antisymmetric tensor in the same manner as \cref{eq:AntisymmTau}.
The expansion of 
\cref{eq:CIExp} may be further cast into a more general, and convienient form in terms of explicit action of the transition
operators such that for an arbitrary $M$ particle Slater determinant,
\begin{equation}
  \label{eq:ExcitationOp}
  \ket{\Psi^M_n}_\hilb{F} = \op{R}_n^{M,N} \ket{0^N}_\hilb{F},
\end{equation}
where $\op{R}_n^{N,M}$ is an ``excitation" operator which is defined as
\begin{align}
&\op{R}_n^{M,N} = 
  \begin{cases}
    \displaystyle \sum_{abc\cdots} \left( t^n \right)^{abc\cdots} \tau^{abc\cdots} + \sum_{abc\cdots di} \left( t^n \right)^{abc\cdots d}_i \tau^{abc\cdots d}_i +
      \sum_{abc\cdots deij} \left( t^n \right)^{abc\cdots de}_{ij} \tau^{abc\cdots de}_{ij} + \cdots & M > N \\
    \displaystyle\left( t^n \right)_0  + \sum_{ai} \left( t^n \right)^{a}_i \tau^{a}_i +
      \sum_{abij} \left( t^n \right)^{ab}_{ij} \tau^{ab}_{ij} + \sum_{abcijk} \left( t^n \right)^{abc}_{ijk} \tau^{abc}_{ijk} + \cdots & M = N \\
    \displaystyle \sum_{ijk\cdots} \left( t^n \right)_{ijk\cdots} \tau_{ijk\cdots} + \sum_{a ijk\cdots l} \left( t^n \right)^{a}_{ijk\cdots l} \tau^{a}_{ijk\cdots l} +
      \sum_{abijk\cdots lm} \left( t^n \right)^{ab}_{ijk\cdots lm} \tau^{ab}_{ijk\cdots lm} + \cdots & M < N \\
  \end{cases}
\end{align}
Here the extent of extended sums of transition operators involving unbalanced creating and annihliation operators in the cases $M\neq N$ is to be understood from
the context, i.e. for the special case that $M= N+1$, we obtain
\begin{align}
&\op{R}_n^{N+1,N} = \sum_a \left( t^n \right)^a \tau^a + \sum_{abi} \left( t^n \right)^{ab}_i \tau^{ab}_i + 
  \sum_{abcij} \left( t^n \right)^{abc}_{ij} \tau^{abc}_{ij} + \cdots
\end{align}
The remarkable thing about \cref{eq:ExcitationOp} is that all possible many body wave functions may constructed from a single Slater determinant
within an operator formalism. Thus one need only understand the manipulation of $\op{R}^{M,N}_n$ in the context of other many--body operators
and how their combined action affects the reference determinant to to determine expectation values: the primary moiety which we will 
manipulate in this work.

To this point, the next property of second quantization which we will exploit in this work,
and the one that we will use most often, is the fact that one may
cast $M$--particle operators on $\hilb{H}^N$ as scalar contractions of $2M$--element operator 
strings with rank--$2M$ tensors on $\hilb{H}^1$ \cite{Walecka12_book,Schuck04_book,Ostlund12_book},
\begin{subequations}
\begin{align}
  &\sum_i \op{O}^1(i) \mapsto \op{O}_\hilb{F}^1 = \sum_{pq}   O^1_{pq} c^\dagger_p c_q,\\
  &\sum_{i\neq j}\op{O}^2(i,j) \mapsto \op{O}_\hilb{F}^2 = \sum_{pqrs} O^2_{qprs} c^\dagger _p c^\dagger _q c_r c_s.
\end{align}
\end{subequations}
The tensors $O^1_{pq}$ and $O^2_{qprs}$ are defined as in \cref{eq:DiracInts}. This is a truely remarkable
result from a practical perspective at it allows a simple prescription  for the evaluation of 
inner products for general elements $\ket{\Psi_n^N}, \ket{\Psi_m^N} \in \hilb{H}^N$
\begin{subequations}
\label{eq:SeQuantInner}
\begin{align}
&\innerop{\Psi_n^N}{\sum_i \op{O}^1(i)}{\Psi_m^N} = \sum_{pq} O^1_{pq} \innerop{\Psi_n^N}{c_p^\dagger c_q}{\Psi_m^N}_\hilb{F}, \\
&\innerop{\Psi_n^N}{\sum_{i\neq j} \op{O}^2(i,j)}{\Psi_m^N} = \sum_{pq} O^1_{qprs} \innerop{\Psi_n^N}{c_p^\dagger c_q^\dagger c_r c_s}{\Psi_m^N}_\hilb{F}.
\end{align}
\end{subequations}
The moieties $\innerop{\cdot}{\cdot}{\cdot}_\hilb{F}$ denote that all quantities in that inner product are expressed in second quantization.
To demonstrate the utility of this concept, we examine the substitution of the reference excitation ansatz from \cref{eq:ExcitationOp}
into the one--body inner product such that
\begin{align}
\label{eq:SeQuantInnerExp}
&\innerop{\Psi_n^N}{c_p^\dagger c_q}{\Psi_m^N}_\hilb{F} = \innerop{0^N}{\op{R}^{N,N\dagger}_n c_p^\dagger c_q \op{R}^{N,N}_m}{0^N}_\hilb{F} \nonumber \\
&\qquad= \left( t^n \right)^*_0 \left(
     \left( t^m \right)_0 \innerop{0^N}{c_p^\dagger c_q}{0^N}_\hilb{F} +
     \sum_{ai} \left( t^m \right)^{a}_i \innerop{0^N}{c_p^\dagger c_q \tau^{a}_i}{0^N}_\hilb{F} + \cdots
   \right) + \nonumber \\
&\qquad \quad \sum_{ai}\left( t^n \right)^{a*}_i \left(
         \left( t^m \right)_0 \innerop{0^N}{\tau_i^{a\dagger} c_p^\dagger c_q}{0^N}_\hilb{F} +
         \sum_{bj} \left( t^m \right)^{b}_j \innerop{0^N}{\tau_i^{a\dagger} c_p^\dagger c_q \tau^{b}_j}{0^N}_\hilb{F} + \cdots
       \right) + \nonumber\\
&\qquad \quad \cdots
\end{align}
As first glance, it may seems as though casting the inner product into a second quantized form has drastically complicated its evaluation.
However, upon careful introspection, one sees that the rather complicated task of evaluating inner productions for determinants 
of the form \cref{eq:SlaterDet} or \cref{eq:SlaterDetSpace}, which involve the evaluation of rather complicated many body integrals,
into tensor contractions of few body integrals (\cref{eq:DiracInts}) and transition amplidues ($\vc{t}^n$),
with inner products involving strings of creation and annihilation operators. Given the prior, one may heavily exploit the
commutation relationships of \cref{eq:SeQuantComm} to easily evaluate inner products of arbitrary operator strings through
various diagramatic techniques or explicitly by Wick's theorem \cite{Walecka12_book,Schuck04_book} \todo{more cites}.

In essence, second quantization provides a common language which allows one to factor elements of a particular many--body
quantum theory into parts which are dependent and independent of the single--particle basis. Such a state of affairs has
been demonstrated in \cref{eq:SeQuantInner,eq:SeQuantInnerExp}. If one can define an inner product on $\hilb{H}^1$ and
is able to develop an ansatz for the form of \cref{eq:ExcitationOp} for a particualr theory, it is immediately compatible
with second quantization. In this work, second quantization will play a crutial role in the translation of the
results of non--relativistic quantum mechanics to the consistent treatment of relativistic effects.



\subsection{Mean--Field Quantum Mechanics and Basis Set Expansions}
\label{sec:MF}

Up to this point, no approximations in the treatment of the many body wave function have been introduced. That is
to say, given that $\mathcal{C}$ is a complete, countable basis for $\hilb{H}^1$, all of the developments
in the previous sections treat the electronic many body problem \emph{exactly}. However, even a cursorary
inspection of the nature of $\mathcal{C}$ indicates that it must be countably infinite in order to 
satisfy completeness on $\hilb{H}^1$ \todo{confirm, cite?}. Thus it must be truncated in some systematic
way in order for it to be useful in any practical treatment of the many body problem.

The electronic Hamiltonians one typically encounters for molecular systems with $N_\mathrm{el}$ electrons,
$\op{H}_{el} : \hilb{H}^{N_{el}} \rightarrow \hilb{H}^{N_{el}}$ take the general form
\begin{equation}
\op{H}_{el} = \sum_{i}^{N_\mathrm{el}} \left( \op{h}(i) + \op{v}(i) \right) + \frac{1}{2}\sum_{i \neq j}^{N_\mathrm{el}} \op{g}(i,j),
\end{equation}
where $\op{h}$ and $\op{v}$ are the one--body free particle Hamiltonian and potential operators respectively.
$\op{g}$ is a two--body operator which describes the interaction between two electrons.
At this point, the explicit structure of these operators are immaterial to the following developments; simply
noting that the presence of $\op{g}$ renders direct solution of \cref{eq:QWave} for the many body
wave function impractical \todo{cite}. Given this general form, there are several ways one may construct a suitable
$\mathcal{C}$ \todo{cites?}. In this work, we will be primarily concerned with optimizing a set $\mathcal{C}$ such that
we may construct a reference determinant (per \cref{eq:ExcitationOp}) which minimizes an energy functional,
\begin{equation}
E[\Psi] = \frac{\innerop{\Psi}{\op{H}_{el}}{\Psi}}{\inner{\Psi}{\Psi}},
\end{equation}
where $\ket{\Psi}$ is an arbitrary element of $\hilb{H}^{N_{\mathrm{el}}}$. 


% Non--Relativistic Hamiltonian
\subsection{Non--Relativistic Molecular Hamiltonians: The Schr\"{o}dinger Equation}
\label{sec:NRH}

Fundamental to the description of any quantum molecular system, relativistic or non--relativistic, is the non--relativistic
molecular Hamiltonian, $\op{H}^\NR : \hilb{H}^\NR \mapsto \hilb{H}^\NR$, where $\hilb{H}^\NR$ describes a quantum
system containing $N_\mathrm{el}$ electrons and $N_\mathrm{nuc}$ nuclei in the absense of relativistic effects. As such,
it is an approximation to the true Hilbert space which describes the physical state of affairs exactly, but it often
provides a sufficient model for systems when effects such as those which arise from special relativity are unimportant.
$\op{H}^\NR$ is so fundamental to the description
of quantum molecular systems in that it is the linear operator on $\hilb{H}^\NR$ which represents the non--relativistic 
total energy of the system.  In the absence of external fields, $\op{H}^\NR$ takes the form
\begin{equation}
  \label{eq:FullHNR}
  \op{H}^\NR = \op{H}_{el}^\NR + \op{H}_{nuc}^\NR + \op{H}_{mix}^\NR
\end{equation}
where 
\begin{subequations}
\begin{align}
  &\op{H}_{el}^\NR  = \sum_i^{N_\mathrm{el}}\op{T}^\NR(i) + \frac{1}{2}\sum_{i \neq j}^{N_\mathrm{el}} \op{g}^C(i,j),\\ 
  &\op{H}_{nuc}^\NR = \sum_A^{N_\mathrm{nuc}}\op{T}^\NR(A) + \frac{1}{2}\sum_{A \neq B}^{N_\mathrm{nuc}} \op{g}^C(A,B),\\ 
  &\op{H}^\NR_{mix} = \sum_i^{N_\mathrm{el}} \sum_A^{N_\mathrm{nuc}} \op{g}^C(i,A).
\end{align}
\end{subequations}
Here we have denoted operator action onto the electronic degrees of freedom as $i,j$ and $A,B$ for the nuclear degrees of
freedom. For a general (electronic or nuclear) coordinate, $\xi$, the non--relativistic kinetic energy operator, 
$\op{T}^\NR$, is given by
\begin{align}
  \op{T}^\NR(\xi) = \frac{1}{2m_\xi} \op{\vc{p}}(\xi) \cdot \op{\vc{p}}(\xi),
\end{align}
where $\op{\vc{p}}$ is the linear momentum operator and $m_\xi$ is the mass of the $\xi$-th particle.
$\op{g}^C$ is the two body Coulomb operator which describes the electrostatic interaction between charged particles.

In the spinor representation, these operators take the general form (in atomic units)
\begin{align}
  \label{eq:SpinorKinMom}
  &\spop{\vc{p}}(\vc{x}_\xi) = -\ii \nabla_\xi \otimes \op{I}_S(\xi) \quad \Longrightarrow \quad 
  \spop{T}^\NR(\vc{x}_\xi) = - \frac{1}{2m_\xi} \Delta_\xi \otimes \op{I}_S(\xi),
\end{align}
\begin{align}
  \label{eq:SpinorCoulomb}
  &\spop{g}^C(\vc{x}_\xi,\vc{x}_\zeta) = 
    \frac{L_\xi L_\zeta}{\vert \vc{r}_\xi - \vc{r}_\zeta \vert} \otimes \op{I}_S(\xi) \otimes \op{I}_S(\zeta),
\end{align}
where $\nabla_\xi$ and $\Delta_\xi$ are the gradient and Laplacian operators acting on the $\xi$-th spatial coordinate,
respectively. $L_\xi$ is the charge of the $\xi$-th particle, which in atomic units is given by $-1$ for electrons and 
$Z_A$ for the $A$--th nucleus, where $Z_A$ is the number of protons in said nucleus. 
$\op{I}_S(\xi)$  is the identity spin operator for particle $\xi$, which has been introduced to make a
careful distinction between the spinor basis of electrons, which are spin--1/2 fermions, and nuclei, which are in general
\emph{not} spin--1/2 fermions and thus carry a much more complicated spin structure \todo{cite}. For electrons, $\op{I}_S=\pauli{0}$.
The presence of $\op{I}_S$ in \cref{eq:SpinorKinMom,eq:SpinorCoulomb} may be interpreted as the action of these operators 
\emph{do not} manipulate the spin degrees of freedom of the total wave function.

As has been previously stated on numerous occasions, the primary focus of this work is to treat the many body
electronic problem for molecular systems, not the quantum nature of the nuclei. The combined quantum treatment
of \cref{eq:FullHNR} obfuscates this point in that the presence of $\op{H}^\NR_{mix}$ intimately couples the
electronic and nuclear degrees of freedom. Thus it would be of practical utility to, in some way, decouple
the quantum treatment of the electrons and the nuclei such that they may be treated separately. 
%To this end,
%we will work within the Born--Oppenheimer approximation \todo{cite} for the molecular Hamiltonian such that
%we will consider a single tensor product ansatz for the total molecular wave function $\ket{\Psi_{tot}}$,
To this end, we will work within the Born--Oppenheimer ansatz for the molecular wave function \cite{Oppenheimer27_457,Tully98_407} such
that it may be written as a single tensor product of an electronic and nuclear wave function,
\begin{equation}
  \ket{\Psi_{tot}} \approx \ket{\Psi_{el}} \otimes \ket{\Theta_{nuc}}.
\end{equation}
Due to a large disparity in mass between electrons and nuclei, the energetic regimes which describe their respective
dynamics are typically well separated. Namely, from the inertial frame of the electrons, one might approximate
the nuclear kinetic energy to be negligible, i.e.
\begin{equation}
\innerop{\Psi_{tot}}{\sum_A^{N_\mathrm{nuc}} \op{T}^\NR(A) }{\Psi_{tot}} \approx 0.
\end{equation}
This assumption is referred to as the Born--Oppenheimer approximation (a concept distinct from from the 
Born--Oppenheimer ansatz for the molecular wave function). This is typically a safe assumption for reasonably
heavy nuclei as the ratio of the electronic and nuclear kinetic energies are of the same order as
the ratio of the electronic and nuclear masses, $\approx 10^{4}$. Thus, from the electronic perspective,
the nuclear configuration is approximately static, and the electrons only ,,feel" the electrostatic
potential of a fixed nuclear wave function at any given time. Denoting a particular fixed nuclear wave
function $\ket{\Theta_{nuc}^{fix}}$, we may write down a Hamiltonian one acts on the electronic 
component of the wave function and is valid in the inertial frame of the electrons,
\begin{equation}
\label{eq:NRBOH}
\op{H}^\BO_{el} = \innerop{\Theta_{nuc}^{fix}}{\op{H}^\NR - \sum_A^{N_\mathrm{nuc}}\op{T}^\NR(A)}{\Theta_{nuc}^{fix}}
=E_{nn} + \sum_i^{N_\mathrm{el}} \op{T}^\NR(i) + \op{V}_{ne}(i) + \sum_{i\neq j}^{N_\mathrm{el}} \op{g}^C(i,j),
\end{equation}
where
\begin{align}
  &\op{V}_{ne}(\vc{x}_i) = 
    -\sum^{N_\mathrm{nuc}}_A \left(\int_{\mathbb R^3} \frac{Z_A\rho^0_\mathrm{nuc,A}(\vc{R})}{\vert \vc{r}_i - \vc{R} \vert} \dd^3\vc{R}\right) \otimes \pauli{0}, \\
&E_{nn} = \frac{1}{2} \sum^{N_\mathrm{nuc}}_{A\neq B} \iint_{\mathbb R^3} \frac{Z_AZ_B\rho^0_\mathrm{nuc,A}(\vc{R})\rho^0_\mathrm{nuc,B}(\vc{R}')}{\vert \vc{R} - \vc{R}' \vert} \dd^3\vc{R} \dd^3\vc{R}'.
\end{align}
$\rho^0_{nuc,A}(\vc{R})$ is the scalar density of the $A$--the nucleus defined through a proper generalization of
\cref{eq:TensorDensity,eq:DensityInner} for non spin--1/2 fermions. The explicit form of $\rho^0_{nuc,A}(\vc{R})$ immaterial
to this work except for the property,
\begin{equation}
\sum_A^{N_\mathrm{nuc}} \int_{\mathbb R^3} \rho_{nuc,A}^0(\vc{R}) \dd^3\vc{R} = N_\mathrm{nuc}.
\end{equation}
This criteria is clearly met if $\rho_{nuc,A}(\vc{R})$ is a normalized function. In this work, we will approximate
$\rho_{nuc,A}(\vc{R})$ as a classical charge distribution described by a single Gaussian function \todo{cite},
\begin{equation}
\rho_{nuc,A}(\vc{R}) = \mathcal{N} e^{-\gamma_A(\vc{R} - \vc{R}_A)^2},
\end{equation}
where $\vc{R}_A$ is the classical nuclear position and \todo{give expressions for exponent and normalization condition}


The wave equation governed by \cref{eq:NRBOH} as a specialization of \cref{eq:QWave} is given by
\begin{equation}
\label{eq:NRSCH}
\op{H}^\BO_{el} \ket{\Psi_{el} (t)} = \ii \partial_t \ket{\Psi_{el} (t)}
\end{equation}
and will be referred to as the electronic Schr\"{o}dinger equation within the Born--Oppenheimer approximation \cite{Tully98_407}.
In this work, all non--relativistic specializations of \cref{eq:QWave} will referred to as some variant of the Schr\"{o}dinger
equation due to the explicit presence of the kinetic energy operator in the Hamiltonian. This is due to the fact that they
are specialization of the original Schr\"{o}dinger Hamiltonian of the form
\begin{equation}
\label{eq:SchHam}
\op{H}(t) = \op{T}^\NR + \op{V}(t).
\end{equation}
There are a number of problems inherent in the form of \cref{eq:SchHam,eq:NRBOH}. The first problem that is of interest to this work,
and perhaps the most glaring in the context of molecular calculations involving the quantum treatment of electrons,
is that in the limit of electrostatics in the absence of external fields there is only trivial action of the
Hamiltonian onto the spin components of the electronic  wave function. To be clear, electronic spin is inherent in any quantum
treatment of electrons as it manifests naturally through the irreducible representations of the Galilean group
which governs non--relativistic mechanics \cite{Levy67_286}. \co{Shure up or remove} Phenomenologically, one may introduce non--trivial spin manipulation into the non--relativistic
Hamiltonian through the interaction with an external magnetic field, $\vc{B}$, via the spin--Zeeman term,
\begin{equation}
\op{H}^\BO_{el} \mapsto  \op{H}^\BO_{el} + \op{H}^\mathrm{Zeeman}, \qquad \op{H}^\mathrm{Zeeman}  = \frac{1}{2} \sum_{k=1}^3 B^k \pauli{k},
\end{equation}
due to observation of the linear relationship between the magnetic field strength and energy level splittings in the Stern--Gerlach 
experiment\cite{Napolitano17_book}. However, it is well known that $\op{H}^\mathrm{Zeeman}$ only yields a proper physical description
of the electronic spin degrees of freedom in the strong field limit \todo{cite}, and clearly yields no mechanism to treat
intrinsic effects such as zero--field splitting of electronic energy levels.

The second, perhaps more subtle problem relating specifically to the forms of \cref{eq:SchHam,eq:NRSCH} is that they are manifestly
incapable of adhering to the laws of special relatively, i.e. it is impossible to write down a Hamiltonian of the form in
\cref{eq:SchHam} which is Lorentz covariant. This problem is easily identified through recognizing that the Schr\"{o}dinger
equation is quadratic in spatial coordinates through $\op{T}^\NR$ and linear in time, and thus incapable of being compatible with Lorentz boosts. At first glance,
one might be tempted to think in terms of a classical picture for a quantum analogue, where for velocities much lower than the speed of light,
the dynamics of classical bodies is approximately governed by Newtonian mechanics. Thus, within such a mindset, one might consider
relativistic effects as only being important for heavy elements, such as Gold or Uranium, due to the fact that their core electrons
move at velocities which are a considerable fraction of the speed of light. However, as is often the case with the quantization
of classical mechanics, such a simplistic assumption yields qualitatively incorrect model physics even for light elements such as Carbon and Oxygen\todo{cite Pyyko}.
The realization of relativistic effects in light elements typically manifests in context of the \emph{ab initio} introduction of spin couplings
into the Hamiltonian, which not surprisingly also solves the aforementioned problem with treating electronic spin non--relativistically. 
The following section provides a brief overview of the treatment of relativistic effects in molecular quantum mechanics.



%Note that in the absence of external fields and within the approximation that the nuclei are stationary, $\op{H}^\BO_{el}$ 
%is time--independent, and thus has the general solution\todo{cite Sakurai}
%\begin{equation}
%\ket{\Psi_{el}(t)} = \exp \left[ -\ii \Delta t \op{H}^\BO_{el} \right]\ket{\Psi_{el}(t_0)}, \qquad \Delta t = t - t_0,
%\end{equation}
%where $t_0$ is a reference time for which one may specify an initial condition for $\ket{\Psi_{el}(t)}$.




% Relativistic Hamiltonian
\subsection{Approximate Relativistic Hamiltonians: The Dirac Equations and Two--Component Variants}
\label{sec:RELH}

To be clear, although it is rarely outright stated in the context of the electronic structure theory
literature, there is no \emph{truely} relativistically covariant formulation of many--body quantum 
mechanics. This is due to the fact that the Coulomb interaction only treats the instantaneous interactions
of charged particles and is thus manifestly incapable of adhering to the principles of special relativity.
One must venture into the realm of quantum field theory (QFT) in order to develop truely covariant quantized 
descriptions of the electromagnetic force; however, QFT has proven to be rather difficult to exploit in
the context of practical calculations on molecular systems \todo{cite Liu}. Here, we will work with
\emph{approximate} relativsitic descriptions of molecular quantum mechanics within a Hamiltonian
formulation, which have been shown to have good agreement with experiment.
For simplicity in the subsequent developements, we will posit \emph{a priori} the validity of the 
Born--Oppenheimer ansatz and approximations of the previous section in the context of relativistic
theory \co{justify?}. Further, while no approximation of the quantum nature of the nuclei was required in the 
context of the Sch\"{o}dinger equation, in the context of relativistic theory we will approximate
nuclei to be classical charge distributions, i.e. no intrinic angular momentum (spin) which
yields an absence of magnetic interactions into the Hamiltonian within the fixed nuclei approximation.
This approximation is justified \co{because?}, and will drastically simplify the resulting 
quantum mechanical treatment.

Within the context of the electronic problem, relativistic quantum mechanics is approximately governed
by the Breit equation,
\begin{equation}
\op{H}^\DCB \ket{\Psi_{el} (t)} = \ii \partial_t \ket{\Psi_{el} (t)},
\end{equation}
where $\op{H}^\DCB : \hilb{H}^\REL \rightarrow \hilb{H}^\REL$ is the Dirac--Coulomb--Breit Hamiltonian.




\todo{Dirac Hamiltonian + why it fixes things}
\todo{DCB Hamiltoninan}
\todo{BP hamiltonian from DCB}
\todo{NR limit of BP}
\todo{General Two component}
\todo{X2C (why its better), possibly appendix on implementation?}
\todo{Uncontracted basis?}





% Semi--Classical Light Matter
\section{Semi--Classical Light--Matter Interaction}
\label{sec:SCLMI}

% Polarization Propagator
\subsection{The Resonant Convergent Polarization Propagator}
\label{sec:PolarProp}

% Absorption Cross Section
\subsection{The Linear Absorption Cross Section}
\label{sec:AbsorptionTheory}
