\chapter{Theoretical Preliminaries}
\label{ch:Theory}

In this chapter, I will outline the theoretical preliminaries which will serve as the basis
for the subsequent development of relativistic electronic structure theory. Further, this
chapter will serve as the primary source of notation which will be used throughout the
remainder of this work. 


\section{The Physical Hilbert Space and The Slater Determinant}
\label{sec:SD}

Perhaps the most fundamental axiom of quantum mechanics is in that for every physical
system, there is associated a separable complex Hilbert space, $\hilb{H}$, such that
the vectors of said Hilbert space represent the quantum states 
of the system \cite{VonNeumann55_book}. Such vectors are referred to as wave functions
and the inner product on $\hilb{H}$ is referred to as an expectation value.
The formal structure of $\hilb{H}$ is dictated by the Hamiltonian
for the physical system, $\op{H} : \hilb{H} \rightarrow \hilb{H}$, through the wave equation
\begin{equation}
  \label{eq:QWave}
\op{H}(t) \ket{\Psi(t)} = \ii \partial_t \ket{\Psi(t)}, 
  \qquad \ket{\Psi(t)} \in \hilb H.
\end{equation}
Here, $t$ is the proper time of the quantum system and $\partial_t$ is the partial derivative
of the wave function with respect to time. The explicit dependence on $t$ will be dropped
for brevity in much of the following, except for when its presence is required to
avoid ambiguity.
The nature of $\op H$ for various physical situations and approximations relevant
to this work will be discussed in detail in later sections 
(see \cref{sec:NRH,sec:RELH} for instance), however the mere existence of such
an operator at this stage is sufficient for the subsequent developments.
As the primary focus of this work will be the treatment of the many--body electronic
problem in molecular quantum mechanics, one might remark that the physical system
in question is a \emph{composite} system consisting of many, indistinguishable 
particles (electrons). Denoting the Hilbert space of a composite system consisting
of $N$ particles as $\hilb{H}^N$, we know that as $\hilb{H}^N$ is separable,
it must admit a countable, dense basis \cite{Lee03_book}.
The direct product space consisting of the Hilbert spaces
which describe its constituent parts, namely those spaces which describe a
single particle system: $\hilb{H}^1$, provides a convenient basis for $\hilb{H}^N$ such that
\begin{equation}
  \label{eq:SepHilbert}
  \hilb{H}^N = \Span{\bigotimes_{i = 1}^N \hilb{H}^1}.
\end{equation}
The separability condition of \cref{eq:SepHilbert} is crucial to the following developments
as it allows one to construct a simple basis for vectors in $\hilb{H}^N$
as $N$--fold tensor products of single--particle wave functions which form a countable basis for $\hilb{H}^1$, 
i.e. $\hilb{C} = \set{\ket{\phi_p}}  \subset \hilb{H}^1$ such that 
$\hilb{H}^1 = \Span{\hilb{C}}$. In the following, vectors in $\hilb{H}^1$, in particular elements of $\hilb{C}$,
will be referred to as orbitals.

As the electron is the moiety of interest in this work, vectors in $\hilb{H}^N$ must adhere to certain additional 
criteria in order for them to represent physically realizable wave functions. As electrons are 
fermions, physically relevant elements of $\hilb{H}^N$, which we will denote $\hilb{H}^N_- \subset \hilb{H}^N$, must 
adhere to Fermi--Dirac statistics, namely that they must exhibit anti-symmetric behavior under particle permutation
\cite{Walecka12_book,Schuck04_book}, i.e.
\begin{equation}
  \label{eq:FermiDirac}
  \hilb{P}_{ij} \ket{\Phi^N} = -\ket{\Phi^N}, \qquad \ket{\Phi^N} \in \hilb{H}^N_-,
\end{equation} 
where $\hilb{P}_{ij}$ is the particle permutation operator which interchanges particle $i$ and $j$ 
(for a more thorough discussion on linear operators acting on $\hilb{H}^N$, see \cref{sec:LO}). Remark
that this notion of particle interchange is intimately related to \cref{eq:SepHilbert}. With this additional
constraint, we may construct a basis for $\hilb{H}^N_-$ with elements \todo{find a good reference}
\begin{equation}
  \label{eq:SlaterDet}
  \ket{\Phi^N_I} = \frac{1}{\sqrt{N!}} \sum_{\xi \in S_N(\hilb{K}^N_I)} \Sign{\xi} \text{ } \bigotimes_{i = 1 }^N \ket{\phi_{\xi(i)}},
\end{equation}
where $\hilb{K}_I^N$ is an $N$--element subset of $\hilb{C}$ and $S_N(\hilb{K}_I^N)$ is the symmetric group
of $\hilb{K}_I^N$ which consists of all permutations of its elements; denoted here as permutation
functions $\xi$. $\Sign{\xi} \in \{ \pm 1 \}$ denotes the sign of the permutation and ensures the anti--symmetry of
the overall wave function. \Cref{eq:SlaterDet} introduces a number of concepts with are typically jargonized 
in the quantum chemistry community. In this work, $\hilb{K}_I^N$ will be referred to as an $N$--particle
configuration (or simply a configuration when $N$ is to be understood from the context), and $\ket{\Phi^N_I}$
will be referred to as a Slater determinant. It is important to note that $\ket{\Phi^N_I}$ is completely 
determined by $\hilb{K}_I^N$ but is only unique up to a unitary transformation \cite{Ostlund12_book}, i.e.
for a unitary matrix $\vc{U} \in \mathbb C^{N\times N}$,
\begin{equation}
\label{eq:UnitarySD}
\ket{\Phi_I^N} = \ket{\Phi_J^N} \quad \text{ iff } \quad \ket{\psi_j} = \sum_{i=1}^N \vc{U}_{ji} \ket{\phi_i} \quad 
  \forall \ket{\psi_j} \in \mathcal{K}^N_J, \ket{\phi_i} \in \mathcal{K}^N_I.
\end{equation}
The $N$--fold tensor product on the right hand side of \cref{eq:SlaterDet}
if referred to as a Hartree product, and while not a valid fermionic wave function in and of itself,
it provides an important building block for constructing such wave functions and will be the primary
moiety with which we will develop the arithmetic of many body quantum theory. As such, we reserve a shorthand
for the Hartree product constructed for a general set $\set{\ket{\psi_p}}\subset\hilb{H}^1$ as
$ \ket{\psi_1,\psi_2\cdots} \equiv \ket{\psi_1} \otimes \ket{\psi_2} \otimes \cdots$.

As a basis for $\hilb{H}^N_-$, any vector $\ket{\Psi_n^N} \in \hilb{H}^N_-$ may be written as \cite{Ostlund12_book}
\begin{equation}
\label{eq:SDBasis}
\ket{\Psi_n^N} = \sum_I D^n_I \ket{\Phi^N_I},
\end{equation}
where $D^n_I = \inner{\Phi^N_I}{\Psi^N_n}\in\mathbb C$ is the complex expansion coefficient of the $I$-th configuration in the overall wave function,
and $I$ runs over all unique $N$--particle Slater determinants which may be constructed from $\hilb{C}$.
The fact that the set of all Slater determinants forms a basis for $\hilb{H}^N_-$
serves as the primary foundation for the majority of approximate quantum mechanical methods regarding molecular
systems. In the following, we will assume both $\ket{\Phi^N_I}$ and the elements of $\hilb{C}$ are orthonormal
with respect to the metric on their respective Hilbert spaces.

While the Hilbert space representation of the wave function is the most illuminating description for the development of
general quantum mechanical theory, it is often advantageous from the perspective of practical calculations that one
projects the vectors of the Hilbert space onto a convenient basis. 
To this end, we consider a specific single particle basis, $\set{\ket{\vc{r},\sigma} = \ket{\vc{r}}\otimes\ket{\sigma}}$, 
which consists of the simultaneous eigenfunctions
of both the position and $z$--spin operators, denoted $\op{\vc{r}}$ and $\op{S}_z$,
such that
\begin{subequations}
\begin{align}
  \op{\vc{r}}\ket{\vc{r},\sigma} &= \ket{\vc{r},\sigma}\vc{r}, \qquad \vc{r}\in\mathbb{R}^3, \\
  \op{S}_z\ket{\vc{r},\sigma}    &= \ket{\vc{r},\sigma}\sigma, \qquad \sigma \in \set{\pm\frac{1}{2}}.
\end{align}
\end{subequations}
Here, we have denoted the particle's position and $z$--axis spin projection as $\vc{r}$ and $\sigma$, respectively.
Namely, if $\hilb{H}^1$ represents a spin--1/2 fermion, $\set{\ket{\vc{r},\sigma}}$ forms a complete basis for $\hilb{H}^1$ 
and admits the following orthonormality  condition on the $\hilb{H}^1$ inner product,
\begin{equation}
  \label{eq:SpinorInner}
  \inner{\vc{r}',\sigma'}{\vc{r},\sigma} = \delta^3(\vc{r} - \vc{r}')\delta_{\sigma\sigma'},
\end{equation}
where $\delta^3$ and $\delta_{\sigma\sigma'}$ are the Dirac delta function and Kronecker delta tensor, respectively.
As $\set{\ket{\vc{r},\sigma}}$ is continuous, i.e. its spectrum is continuous, its carnality is uncountable. Thus,
its utility does not manifest as it does in the context of countable bases, such as is required by \cref{eq:SlaterDet},
but rather in the fact that it allows for the casting of inner products on $\hilb{H}^1$ as integrals through the resolution
of the identity on $\hilb{H}^1$ via
\begin{equation}
  \label{eq:H1Identity}
  \op{1}_1 = \sum_\sigma \int_{\mathbb R ^3} \ket{\vc{r},\sigma}\bra{\vc{r},\sigma} \dd^3\vc{r}.
\end{equation}
As such, for arbitrary $\ket{\phi},\ket{\phi'}\in\hilb{H}^1$ we may cast the inner product as
\begin{equation}
\inner{\phi}{\phi'} = \sum_\sigma \int_{\mathbb R ^3} \phi^*(\vc{r},\sigma) \phi'(\vc{r},\sigma) \dd^3\vc{r},
\end{equation}
where we have defined
\begin{equation}
  \label{eq:SpinorOrbital}
  \inner{\vc{r},\sigma}{\phi} \equiv \phi(\vc{r},\sigma), \quad \mathrm{s.t.} \quad \phi : \mathbb F \mapsto \mathbb C,
\end{equation}
and $\mathbb F = \mathbb R^3 \times \set{\pm1/2}$. 
\Cref{eq:SpinorOrbital}, as a projection onto
an element of a product space, may be further separated onto the spin basis, $\set{\alpha,\beta}$,
\begin{align}
  \phi(\vc{r},\sigma) = \phi^\alpha(\vc{r})\alpha(\sigma) + \phi^\beta(\vc{r})\beta(\sigma),
\end{align}
such that
\begin{align}
  &\alpha(\sigma) = \begin{cases} 1 & \sigma = +\frac{1}{2} \\ 0 & \sigma = -\frac{1}{2} \end{cases}, \\ 
  &\beta(\sigma)  = \begin{cases} 0 & \sigma = +\frac{1}{2} \\ 1 & \sigma = -\frac{1}{2} \end{cases}.
\end{align}
In general, single particle wave functions of the form \cref{eq:SpinorOrbital} will be referred to as spinor orbitals, or simply spinors.
For brevity in the following, we will denote $\ket{\vc{x}} \equiv \ket{\vc{r},\sigma}$ such that
\begin{subequations}
\begin{align}
\int_{\mathbb F} f(\vc{x}) \dd^4 \vc{x} &\equiv \sum_\sigma \int_{\mathrm{R}^3} f(\vc{r},\sigma) \dd^3 \vc{r},\\
\delta^4(\vc{x} - \vc{x}') &\equiv \delta^3(\vc{r} - \vc{r}')\delta_{\sigma\sigma'}
\end{align}
\end{subequations}

As a basis for $\hilb{H}^1$, we may construct a basis for $\hilb{H}^N$ through $N$--fold tensor products 
of the elements of $\set{\ket{\vc{x}}}$ via \cref{eq:SepHilbert}. Denoting $\ket{\vc{x}_i}$ as
a specific element of $\set{\ket{\vc{x}}}$, the vectors
\begin{equation}
  \label{eq:MBSpinorBasis}
  \ket{\vc{x}_1,\vc{x}_2,\ldots,\vc{x}_N} = \bigotimes_{i = 1}^N \ket{\vc{x}_i}
\end{equation}
form a basis form a basis for $\hilb{H}^N$. Extending \cref{eq:SpinorInner,eq:H1Identity} in a similar manner, we may state
\begin{equation}
  \inner{\vc{x}'_1,\vc{x}'_2,\ldots,\vc{x}'_N}{\vc{x}_1,\vc{x}_2,\ldots,\vc{x}_N} = \prod_{i=1}^N \delta^3(\vc{r}'_i - \vc{r}_i)\delta_{\sigma_i\sigma'_i},
\end{equation}
\begin{equation}
  \op{1}_N = \idotsint_{\mathbb F} \bigotimes_{i=1}^N \ket{\vc{x}_i} \bra{\vc{x}_i} \dd^4\vc{x}_i,
\end{equation}
such that for $\ket{\Phi^N},\ket{\Psi^N} \in \hilb{H}^N$,
\begin{align}
&\inner{\Phi^N}{\Psi^N} = \nonumber \\&\quad \idotsint_{\mathbb F} 
  \Phi^{N*}(\vc{x}_1,\vc{x}_2,\ldots,\vc{x}_N) \Psi^{N}(\vc{x}_1,\vc{x}_2,\ldots,\vc{x}_N) \dd^4\vc{x}_1 \cdots \dd^4\vc{x}_N,
\end{align}
where we have denoted
\begin{equation}
  \label{eq:SpinorWfn}
  \inner{\vc{x}_1,\vc{x}_2,\ldots,\vc{x}_N}{\Phi^N} \equiv \Phi^N(\vc{x}_1,\vc{x}_2,\ldots,\vc{x}_N).
\end{equation}
In this work, many--body wave functions of the form \cref{eq:SpinorWfn} will be referred to as spinor wave functions.
The utility of such as basis expansion manifests in the context of Slater determinants in that as
a direct consequence of \cref{eq:SlaterDet,eq:MBSpinorBasis,eq:SpinorOrbital}, we may express
\begin{equation}
  \label{eq:SlaterDetSpace}
  \Phi_I^N(\vc{x}_1,\vc{x}_2,\ldots,\vc{x}_N) = \frac{1}{\sqrt{N!}} \sum_{\xi \in S_N(\hilb{K}^N_I)} \Sign{\xi} \text{ } 
    \prod_{i = 1 }^N \phi_{\xi(i)}(\vc{x}_i).
\end{equation}
Unlike the tensor product definition of \cref{eq:SlaterDet}, the expression of the Slater determinant in will prove to be of 
much of practical utility due to the fact that the spinor orbital basis, as a set of complex valued functions, is commutative.

\section{Representation of Linear Operators and Expectation Values}
\label{sec:LO}

Fundamental to the formulation of any quantum mechanical theory is the identification of linear operators on $\hilb{H}^N$
which represent the physical observables of the system. In this work we will refer to such observables as properties. A
more precise identification of the operators relevant to this work will be presented later 
(see \cref{sec:NRH,sec:RELH,sec:SCLMI} for instance), however in this section we will focus on the general presentation
of these operators and how they will typically manifest in the context of \cref{eq:SepHilbert}. 

In this work, linear operators which act on $\hilb{H}^N$, $\op{O}^N : \hilb{H}^N \mapsto \hilb{H}^N$, will be referred to as 
$N$--particle operators. As was the case for the state vectors of $\hilb{H}^N$, the general description of $\op{O}^N$
as an operator on a Hilbert space is indeed the most illuminating treatment for general manipulations of quantum mechanical
operators which is independent of coordinate projection. However, it will often be the case that we must examine the
projection of these operators onto a coordinate space in order to perform practical calculations. To demonstrate this, we examine
the action of an operator $\op{O}^1$ on the spinor basis $\set{\ket{\vc{x}}}$ from \cref{sec:SD} through the identity resolvent in
\cref{eq:H1Identity},
\begin{align}
  \op{O}^1 = \op{1}_1 \op{O}^1 \op{1}_1 = \iint_{\mathbb F} 
    \ket{\vc{x}} \innerop{\vc{x}}{\op{O}^1}{\vc{x}'}\bra{\vc{x}'} \dd^4\vc{x} \dd^4\vc{x}',
\end{align}
such that for $\ket{\psi},\ket{\chi} \in \hilb{H}^1$,
\begin{align}
  \label{eq:SpinorOpInner1}
  \innerop{\psi}{\op{O}^1}{\chi} &= \iint_{\mathbb F}
    \psi^*(\vc{x}) \innerop{\vc{x}}{\op{O}^1}{\vc{x}'} \chi({\vc{x}'}) \dd^4\vc{x} \dd^4\vc{x}' \nonumber \\
  &= \int_{\mathbb R^3} 
    \begin{bmatrix}
      \psi^\alpha(\vc{r}) \\
      \psi^\beta(\vc{r})
    \end{bmatrix}^\dagger
    \begin{bmatrix}
      \op{O}^{1,\alpha\alpha}({\vc{r}}) & \op{O}^{1,\alpha\beta}({\vc{r}}) \\
      \op{O}^{1,\beta\alpha} ({\vc{r}}) & \op{O}^{1,\beta\beta} ({\vc{r}})
    \end{bmatrix}
    \begin{bmatrix}
      \chi^\alpha(\vc{r}) \\
      \chi^\beta(\vc{r})
    \end{bmatrix} \dd^3\vc{r},
\end{align}
Here we have made an assumption which is crucial to the following developments, namely that the operators that we will work with
are \emph{spatially local}, i.e.
\begin{equation}
  \label{eq:LocalSpinorOp1}
\innerop{\vc{x}}{\op{O}^1}{\vc{x}'}  \equiv \innerop{\vc{r},\sigma}{\op{O}^1}{\vc{r}',\sigma'} = \delta^3(\vc{r} - \vc{r'}) \op{O}^{1,\sigma\sigma'}(\vc{r}').
\end{equation}
In general, operators need not be spatially local, and as such must be specified with two coordinate indices, i.e. $\op{O}^1(\vc{x};\vc{x}')$
(see \cref{eq:DenMatSpace} for instance).
However, it will be the case that the linear operators which correspond to observables of the physical system will typically be spatially local,
and instances where this is not the case will be treated explicitly. 


This notion of operator representation may be generalized to $N$--particle operators in a similar manner through resolution of $\op{1}_N$,
\begin{align}
  \op{O}^N &= \op{1}_N \op{O}^N \op{1}_N 
  = \idotsint_{\mathbb F}
    \dd^4\vc{x}_1 \cdots   \dd^4\vc{x}_N
    \idotsint_{\mathbb F}
    \dd^4\vc{x}'_1 \cdots  \dd^4\vc{x}'_N \times \nonumber\\
    &\qquad\ket{\vc{x}_1,\ldots,\vc{x}_N} \innerop{\vc{x}_1,\ldots,\vc{x}_N}{\op{O}^{N}}{\vc{x}'_1,\ldots,\vc{x}'_N}\bra{\vc{x}'_1,\ldots,\vc{x}'_N} 
\end{align}
such that for arbitrary $N$--body Hartree products constructed from $\set{\ket{\psi_p}},\set{\ket{\chi_p}} \subset \hilb{H}^N$
\begin{align}
  \label{eq:SpinorOpInnerN}
  &\innerop{\psi_1,\psi_2,\ldots,\psi_N}{\op{O}^N}{\chi_1,\chi_2,\ldots,\chi_N} =  
  \sum_{\sigma_1\cdots\sigma_N}\sum_{\tau_1\cdots\tau_N} \idotsint_{\mathbb R^3} 
    \dd^3\vc{r}_1\cdots\dd^3\vc{r}_N  \nonumber \times \\ &\qquad \qquad
    \psi_1^{\sigma_1*}(\vc{r}_1) \cdots \psi_N^{\sigma_N*}(\vc{r}_N) ~ \op{O}^{N,\sigma_1\tau_1\cdots\sigma_N\tau_N}(\vc{r}_1,\ldots,\vc{r}_N) ~ 
    \chi_1^{\tau_1}(\vc{r}_1) \cdots \chi_N^{\tau_N}(\vc{r}_N).
\end{align}
where we have again assumed operator spatial locality,
\begin{equation}
  \label{eq:LocalSpinorOpN}
  \innerop{\vc{x}_1,\ldots,\vc{x}_N}{\op{O}^{N}}{\vc{x}'_1,\ldots,\vc{x}'_N} = 
    \left(\prod_{i=1}^N\delta^3(\vc{r}_i - \vc{r}'_i)\right) 
    \op{O}^{1,\sigma_1\sigma_1'\cdots\sigma_N\sigma_N'}(\vc{r}_1',\ldots,\vc{r}_N').
\end{equation}
For brevity, this representation of $N$ particle operators will 
often be interpreted as rank--2$N$ tensors in the basis of spin eigenfunctions, 
\begin{equation}
\label{eq:SpinorOpN}
\spop{O}^N (\vc{x}_1,\ldots,\vc{x}_N) \equiv \sum_{\sigma_1\cdots\sigma_N}\sum_{\sigma'_1\cdots\sigma'_N}  \op{O}^{N,\sigma_1\sigma_1'\cdots\sigma_N\sigma_N'}(\vc{r}_1,\ldots,\vc{r}_N) 
  \otimes \bigotimes_{i=1}^N \vc{e}_{\sigma_i} \otimes \vc{e}_{\sigma_i'}, 
\end{equation}
where 
\begin{align}
\vc{e}_\alpha = \begin{bmatrix} 1 \\ 0 \end{bmatrix}, \qquad \vc{e}_\beta = \begin{bmatrix} 0 \\ 1 \end{bmatrix}. 
\end{align}
The product operation of operators denoted $\spop{O}^N(\vc{x}_1,\ldots,\vc{x}_N)$ and their action onto spinor wave functions 
(or more specifically, Hartree products) will be a rank-$N$ tensor contraction over spin indices as depicted in \cref{eq:SpinorOpInner1} 
and more generally in \cref{eq:SpinorOpInnerN}. 
For the remainder of this work, we will refer to the projections $\spop{O}^N(\vc{x}_1,\ldots,\vc{x}_N)$ as 
spinor representations of $\op{O}^N$, or simply spinor operators when appropriate. 
The super-scripted moieties $\op{O}^{N,\sigma_1\sigma'_1\cdots}$ will be referred to as the spinor operator
coefficients relative to the spin basis.
It is important to note that spinor representations
of $N$--particle operators are still \emph{operators}, i.e. they still continue to act to the right
to complete their operation. The utility of \cref{eq:LocalSpinorOp1,eq:LocalSpinorOpN} is in that it allows one to
form operators compatible with spinor representations of the wave function as opposed to the general, 
abstract Hilbert space definition. 
In the following developments, it will be useful to examine \cref{eq:SpinorOpN} under a change of basis from the Kronecker products of
$\vc{e}_\alpha$ and $\vc{e}_\beta$ to that of the Pauli matrices,
\begin{equation}
  \label{eq:SpinorOpPauli}
  \spop{O}^N (\vc{x}_1,\ldots,\vc{x}_N) = \sum_{K_1\cdots K_N}\op{O}^{N,K_1\cdots K_N}(\vc{r}_1,\ldots,\vc{r}_N) \otimes \bigotimes_{i = 1}^N \pauli{K_i}
\end{equation}
where 
\begin{equation}
\label{eq:PauliMats}
\pauli{0} = \begin{bmatrix} 1 & 0 \\ 0 & 1 \end{bmatrix}, \quad
\pauli{1} = \begin{bmatrix} 0 & 1 \\ 1 & 0 \end{bmatrix}, \quad
\pauli{2} = \begin{bmatrix} 0 & -\ii \\ \ii & 0 \end{bmatrix}, \quad
\pauli{3} = \begin{bmatrix} 1 & 0 \\ 0 & -1 \end{bmatrix}.
\end{equation}
An explicit derivation for the form of the general rank--$2N$ transformation is given in \cref{apx:SpinorOp}. Here we state the special case of
one particle spinor operators as it will provide the basis for many of the manipulations in subsequent developments,
\begin{subequations}
\begin{align}
&\op{O}^{1,0}(\vc{r}) = \frac{1}{2} \left( \op{O}^{1,\alpha\alpha}(\vc{r}) + \op{O}^{1,\beta\beta}(\vc{r}) \right),\\
&\op{O}^{1,1}(\vc{r}) = \frac{1}{2} \left( \op{O}^{1,\alpha\beta}(\vc{r}) + \op{O}^{1,\beta\alpha}(\vc{r}) \right),\\
&\op{O}^{1,2}(\vc{r}) = \frac{\ii}{2} \left( \op{O}^{1,\alpha\beta}(\vc{r}) - \op{O}^{1,\beta\alpha}(\vc{r}) \right),\\
&\op{O}^{1,3}(\vc{r}) = \frac{1}{2} \left( \op{O}^{1,\alpha\alpha}(\vc{r}) - \op{O}^{1,\beta\beta}(\vc{r}) \right).
\end{align}
\end{subequations}
The coefficients labeled $\op{O}^{N,K_1K_2,\cdots}$ will be referred to as the spinor operator coefficients relative
to the Pauli matrices, or simply the Pauli coefficients for brevity.
This representation is especially convenient when the operator in question is hermitian, i.e. $\op{O}^N = \op{O}^{N\dagger}$.
In \cref{eq:SpinorOpN}, hermiticity of $\op{O}^N$ simply implies that the \emph{product} of the operator coefficients and 
basis vectors must be invariant under the hermitian adjoint and states nothing about the hermiticity of the coefficients
themselves. Due to the fact that the Pauli matrices are self adjoint, \cref{eq:SpinorOpPauli} implies that $\op{O}^N$
is hermitian if and only if the Pauli coefficients are hermitian. Working with hermitian operators
carries a number of benefits, most notably that the spectrum of such operators is strictly real.


Despite the fact that $\hilb{H}^N$ admits a basis of a direct product space consisting of $N$ single particle
Hilbert spaces, $N$--particle operators need not carry the same structure, 
i.e. in general, these operators need not exist solely as direct products of operators on $\hilb{H}^1$. This is not to say
that $N$--particle operators \emph{cannot} adopt a product structure, just that it is not a requirement, and indeed
if often not the case. However, the notion that some $N$--particle operators \emph{can} adopt a product structure indicates the need
to describe the action of $M$--particle operators on $\hilb{H}^N$ (with $N \geq M$) while leaving $N-M$ particles unchanged.
To demonstrate this state of affairs, it is convenient to examine the action of such an operator, $\op{O}^M(i,j,\ldots)$, on a 
spinor wave function of the form \cref{eq:SlaterDetSpace},
\begin{align}
  \label{eq:MOpSpace}
  &\spop{O}^M(\vc{x}_i,\vc{x}_j,\ldots) \Phi_I^N(\vc{x}_1,\vc{x}_2,\ldots,\vc{x}_N) = \nonumber \\ &\qquad \quad
    \frac{1}{\sqrt{N!}} \sum_{\xi \in S_N(\hilb{K}^N_I)} \Sign{\xi} \text{ } 
    \left(\spop{O}^M(\vc{x}_i,\vc{x}_j,\ldots) \phi_{\xi(i)}(\vc{x}_i)\phi_{\xi(j)}(\vc{x}_j)\cdots\right)
    \prod_{k \neq (i,j,\ldots) }^{N-M} \phi_{\xi(k)}(\vc{x}_k),
\end{align}
where $(i,j,\ldots)$ is an $M$--element tuple specifying the subset of particles upon which it acts.
In this work, we will typically be concerned with operators which act on no more than two particles at a time.

It will be often the case that $N$--particle operators may be expressed as sums over one and
two particle operators. In this context , the realization of \cref{eq:MOpSpace} is of exceptional utility in that,
for an expectation value involving a particular Slater determinant described by configuration $\mathcal{K}_I^N$
\cite{Ostlund12_book},
\begin{subequations}
  \label{eq:SlaterCondon}
\begin{align}
  &\innerop{\Phi^N_I}{\sum_i\op{O}^1(i)}{\Phi^N_I} = \sum_{i\in\hilb{K}^N_I} O^1_{ii},\\
  &\innerop{\Phi^N_I}{\sum_{i\neq j} \op{O}^2(i,j)}{\Phi^N_I} = 
    \sum_{i\neq j \in\hilb{K}^N_I} O^2_{ijij} - O^2_{ijji},
\end{align}
\end{subequations}
where for the $\hilb{H}^1$ basis $\mathcal{C} = \set{\ket{\phi_p}}$ from the previous section,
\begin{subequations}
  \label{eq:DiracInts}
\begin{align}
  &O^1_{pq} \equiv \innerop{\phi_p}{\op{O}^1(1)}{\phi_q} = 
    \int_{\mathbb F} \phi_p^*(\vc{x}_1) \spop{O}^1(\vc{x}_1) \phi_q(\vc{x}_1) \dd^4 \vc{x}_1,\\
  &O^2_{pqrs} \equiv \innerop{\phi_p, \phi_q }{\op{O}(1,2)}{\phi_r, \phi_s} = \nonumber \\ &\qquad
    \iint_{\mathbb F} 
      \phi_p^*(\vc{x}_1) \phi_q^*(\vc{x}_2) \spop{O}^2(\vc{x}_1,\vc{x}_2) 
      \phi_r(\vc{x}_1) \phi_s(\vc{x}_2) \dd^4 \vc{x}_1 \dd^4 \vc{x}_2.
\end{align}
\end{subequations}
Namely, $M$--particle on operators on $\hilb{H}^N$ may be represented as rank--$2M$ tensors on $\hilb{H}^1$; a fact which will
be used extensively in the following developments.


\section{Density Matrices}
\label{sec:DenMat}


In the following, it will be useful to define the $N$--particle density matrix, $\op{\gamma}^N$,
associated with a particular state vector $\ket{\Psi^N} \in \hilb{H}^N$ \cite{Yang89_book},
\begin{equation}
\op{\gamma}^N = \ket{\Psi^N}\bra{\Psi^N}.
\end{equation}
Clearly, $\op{\gamma}^N$ is an hermitian $N$--particle on $\hilb{H}^N$ which projects its elements onto $\ket{\Psi^N}$.
The density matrix plays a very central role in the development of quantum mechanical theory in that
it contains the same information as the state vector while its nature as an operator allows for a number
of useful manipulations which simplify many expressions, such as inner products. 
In the spinor coordinate basis, $\op{\gamma}^N$ takes the form
\begin{align}
\label{eq:DenMatSpace}
\innerop{\vc{x}_1,\vc{x}_2,\ldots,\vc{x}_N}{\op{\gamma}^N}{\vc{x}'_1,\vc{x}'_2,\ldots,\vc{x}'_N}
  &\equiv \gamma^N(\vc{x}_1,\vc{x}_2,\ldots,\vc{x}_N; \vc{x}'_1,\vc{x}'_2,\ldots,\vc{x}'_N) \\
  &= \Psi^N(\vc{x}_1,\vc{x}_2,\ldots,\vc{x}_N)\Psi^{N*}(\vc{x}'_1,\vc{x}'_2,\ldots,\vc{x}'_N). \nonumber
\end{align}
From the $N$--particle density matrix, it is possible to define $P$--particle reduced density matrices ($P < N$)
via contractions over single particle indices,
\begin{align}
  &\gamma^P(\vc{x}_1,\vc{x}_2,\ldots,\vc{x}_P;\vc{x}'_1,\vc{x}'_2,\ldots,\vc{x}'_P) = \frac{N!}{P!(N-P)!}
    \idotsint_\mathbb{F} \dd^4\vc{x}_{P+1}\cdots\dd^4\vc{x}_N \times \nonumber \\
  &\qquad \quad \gamma^N(\vc{x}_1,\vc{x}_2,\ldots,\vc{x}_P,\vc{x}_{P+1},\ldots,\vc{x}_N;\vc{x}'_1,\vc{x}'_2,\ldots,\vc{x}'_P,\vc{x}_{P+1},\ldots,\vc{x}_N),
\end{align}
which are hermitian projectors onto the $P$--particle subspaces of $\hilb{H}^N$ which construct $\ket{\Psi^N}$. 
Of particular interest to this work are the one-- and two--particle reduced density matrices,
\begin{align}
&\gamma^1(\vc{x}_1;\vc{x}'_1) = N \idotsint_\mathbb{F}
  \gamma^N(\vc{x}_1,\vc{x}_2,\ldots,\vc{x}_N;\vc{x}'_1,\vc{x}_2,\ldots,\vc{x}_N) \dd^4\vc{x}_{2}\cdots\dd^4\vc{x}_N, \\
&\gamma^2(\vc{x}_1,\vc{x}_2;\vc{x}'_1,\vc{x}'_2) =  \nonumber \\ &\qquad \quad \frac{N(N-1)}{2} \idotsint_\mathbb{F}
  \gamma^N(\vc{x}_1,\vc{x}_2,\vc{x}_3,\ldots,\vc{x}_N;\vc{x}'_1,\vc{x}'_2,\vc{x}_3,\ldots,\vc{x}_N) \dd^4\vc{x}_{3}\cdots\dd^4\vc{x}_N,
\end{align}
which play in important role in the description of one-- and two--body interactions in the electronic Hamiltonian. 
For brevity, we will refer to the one-- and two--particle reduced density matrices for a particular state vector 
as the 1RDM and 2RDM, respectively.
In the case of a single Slater determinant, $\ket{\Psi^N} = \ket{\Phi^N_I}$, the 1 and 2RDMs take on an especially simple form due
to \cref{eq:SlaterDet} \cite{Yang89_book},
\begin{align}
  \label{eq:RDMSD}
  &\gamma^1(\vc{x}_1;\vc{x}'_1) = \sum_{i \in \mathcal{K}_I^N} \phi_i(\vc{x}_1) \phi^*_i(\vc{x}'_1), \\
  &\gamma^2(\vc{x}_1,\vc{x}_2; \vc{x}'_1, \vc{x}'_2) = 
    \frac{1}{2} \left( \gamma^1(\vc{x}_1;\vc{x}'_1)\gamma^1(\vc{x}_2;\vc{x}'_2) - \gamma^1(\vc{x}_1;\vc{x}'_2)\gamma^1(\vc{x}_2;\vc{x}'_1) \right).
\end{align}
In other words, all $P$--particle RDMs corresponding to a single Slater determinant are simply $P$--fold Grassman products of the 1RDM, and thus
the state is completely determined by the 1RDM. This is not the case for general many body wave functions written as \cref{eq:SDBasis} and will
prove to be of great utility in the manipulation of Slater determinants.

In direct analogy to \cref{eq:SlaterCondon}, we may cast the expectation values of 1-- and 2--body operators of a general state vector 
in terms of the 1 and 2RDMs which correspond to that vector
\cite{Yang89_book},
\begin{subequations}
  \label{eq:GenDenMatInner}
\begin{align}
&\innerop{\Psi^N}{\sum_i \op{O}^1(i)}{\Psi^N} = \sum_{\sigma_1\sigma_1'} \int_{\mathbb{R}^3} 
  \left.\left( \op{O}^{1,\sigma_1\sigma_1'}(\vc{r}_1) \gamma^1(\vc{r}_1,\sigma_1; \vc{r}'_1, \sigma_1') \right) \right \vert_{\vc{r}'_1 = \vc{r}_1}
    \dd^3\vc{r}_1,\\
&\innerop{\Psi^N}{\sum_{i \neq j} \op{O}^2(i,j)}{\Psi^N} = \sum_{\sigma_1\sigma_1'\sigma_2\sigma_2'}\iint_{\mathbb{R}^3} \dd^3\vc{r}_1 \dd^3\vc{r}_2  \times\nonumber \\
&\qquad \left.\left( 
    \op{O}^{2,\sigma_1\sigma_1'\sigma_2\sigma_2'}(\vc{r}_1,\vc{r}_2) \gamma^2(\vc{r}_1,\sigma_1,\vc{r}_2,\sigma_2; \vc{r}'_1, \sigma_1',\vc{r}'_2, \sigma_2')
  \right) \right \vert_{\vc{r}'_1 = \vc{r}_1, \vc{r}'_2 = \vc{r}_2}. 
\end{align}
\end{subequations}
It is to be understood from the context that the restriction of the spatial indices is to be performed \emph{after} action of the spinor operator
on the RDM. This allows for generality even in the case where the spinor operator contains differential character.
Substituting in the expressions for the Slater determinant RDMs in \cref{eq:RDMSD} yields \cref{eq:SlaterCondon} exactly. However, the expressions in
\cref{eq:GenDenMatInner} are much more general than that of \cref{eq:SlaterCondon} as they treat an arbitrary many body wave function. Thus
\cref{eq:GenDenMatInner} will be the primary focus in the following.

Motivated by the spin sums in \cref{eq:GenDenMatInner}, it is useful to cast the RDMs as rank--2$P$ tensors on the spin basis in the spirit of 
\cref{eq:LocalSpinorOp1,eq:LocalSpinorOpN} such that
\begin{equation}
  \label{eq:RDMSpinBasis}
  \spn{\gamma}^P(\vc{x}_1,\vc{x}_2,\ldots;\vc{x}'_1,\vc{x}'_2,\ldots ) \equiv 
    \gamma^{P,\,\sigma_1\sigma'_1\sigma_2\sigma'_2\cdots}(\vc{r}_1,\vc{r}_2,\ldots;\vc{r}'_1,\vc{r}'_2,\ldots ) 
    \otimes \bigotimes_{i=1}^P \vc{e}_{\sigma_i} \otimes \vc{e}_{\sigma_i'}.
\end{equation}
In this work, we will refer to this representation as the spinor representation of the RDM and the
super scripted moieties $\gamma^{P,\sigma_1\sigma'_1\cdots}$ as the spinor coefficients relative to
the spin basis.
As was the case for spinor operators in \cref{sec:LO}, hermiticity of the density matrix does not
place any specific restrictions on the hermiticity on the spinor coefficients.
In analogy to \cref{eq:SpinorOpPauli}, we may construct hermitian coefficients via
\begin{equation}
  \spn{\gamma}^P(\vc{x}_1,\vc{x}_2,\ldots;\vc{x}'_1,\vc{x}'_2,\ldots ) \equiv 
    \gamma^{P,\,K_1 K_2 \cdots}(\vc{r}_1,\vc{r}_2,\ldots;\vc{r}'_1,\vc{r}'_2,\ldots ) 
    \otimes \bigotimes_{i=1}^P \pauli{K_i}.
\end{equation}
This realization allows for a simplification of  \cref{eq:GenDenMatInner}, such that, \todo{reference appendix}
\begin{subequations}
  \label{eq:SpinBasisDenMatInner}
\begin{align}
&\innerop{\Psi^N}{\sum_i \op{O}^1(i)}{\Psi^N} = 2 \sum_K \int_{\mathbb R^3} \left.\left( \op{O}^{1,K}(\vc{r}_1) \gamma^{1,K}(\vc{r}_1;\vc{r}'_1) \right) 
  \right \vert_{\vc{r}'_1 = \vc{r}_1} \dd^3\vc{r}_1,\\
&\innerop{\Psi^N}{\sum_{i\neq j} \op{O}^2(i,j)}{\Psi^N} = 4 \sum_{K_1 K_2} \iint_{\mathbb R^3} \dd^3\vc{r}_1 \dd^3\vc{r}_2 \nonumber \times \\ &\qquad \quad
  \left.\left( \op{O}^{2,K_1K_2}(\vc{r}_1,\vc{r}_2) \gamma^{2,K_1 K_2}(\vc{r}_1,\vc{r}_2;\vc{r}'_1,\vc{r}'_2) \right) 
  \right \vert_{\vc{r}'_1 = \vc{r}_1, \vc{r}'_2 = \vc{r}_2},
\end{align}
\end{subequations}
where all of the Pauli components of the spinor operator and RDMs are hermitian. In the case of a single Slater 
determinant constructed from configuration $\hilb{K}^N_I$, the Pauli components of the 1RDM are given by
\begin{subequations}
  \label{eq:SpinorDensityMat}
\begin{align}
  &\gamma^{1,0}(\vc{r};\vc{r}') = \frac{1}{2}   \sum_{i\in\hilb{K}^N_I}\left(\phi^{\alpha}_i(\vc{r})\phi^{\alpha*}_i(\vc{r}') + \phi^{\beta}_i(\vc{r})\phi^{\beta *}_i(\vc{r}') \right), \\
  &\gamma^{1,1}(\vc{r};\vc{r}') = \frac{1}{2}   \sum_{i\in\hilb{K}^N_I}\left(\phi^{\alpha}_i(\vc{r})\phi^{\beta *}_i(\vc{r}') + \phi^{\beta}_i(\vc{r})\phi^{\alpha*}_i(\vc{r}') \right), \\
  &\gamma^{1,2}(\vc{r};\vc{r}') = \frac{\ii}{2} \sum_{i\in\hilb{K}^N_I}\left(\phi^{\alpha}_i(\vc{r})\phi^{\beta *}_i(\vc{r}') - \phi^{\beta}_i(\vc{r})\phi^{\alpha*}_i(\vc{r}') \right), \\
  &\gamma^{1,3}(\vc{r};\vc{r}') = \frac{1}{2}   \sum_{i\in\hilb{K}^N_I}\left(\phi^{\alpha}_i(\vc{r})\phi^{\alpha*}_i(\vc{r}') - \phi^{\beta}_i(\vc{r})\phi^{\beta *}_i(\vc{r}') \right). 
\end{align}
\end{subequations}
The expressions for the 2RDM may be derived from the Grassmann product form in \cref{eq:RDMSD}. It is important to note that
\cref{eq:SpinorDensityMat} is $N$--representable (i.e. it represents an $N$--particle Slater determinant) if and only if
the orbitals used in its construction are orthonormal \cite{Yang89_book}. Thus this places a restriction on the possible
choices of orbitals which may be used in the construction of Slater determinants for which we would like to utilize the
properties of density matrices.


In the case where the operator in equation is scalar multiplicative, i.e. does not contain differential operators, 
the order in which the coordinate restriction under the integration of \cref{eq:GenDenMatInner,eq:SpinBasisDenMatInner} 
occurs is irrelevant. Thus, \cref{eq:SpinBasisDenMatInner} reduces to \cite{Yang89_book}
\begin{subequations}
\begin{align}
&\innerop{\Psi^N}{\sum_i \op{O}^1(i)}{\Psi^N} = 2 \sum_K \int_{\mathbb R^3}  \op{O}^{1,K}(\vc{r}_1) \rho^{1,K}(\vc{r}_1) 
  \dd^3\vc{r}_1,\\
&\innerop{\Psi^N}{\sum_{i\neq j} \op{O}^2(i,j)}{\Psi^N} = 4 \sum_{K_1 K_2} \iint_{\mathbb R^3} 
  \op{O}^{2,K_1K_2}(\vc{r}_1,\vc{r}_2) \rho^{2,K_1 K_2}(\vc{r}_1,\vc{r}_2)\dd^3\vc{r}_1 \dd^3\vc{r}_2
\end{align}
\end{subequations}
where we have defined
\begin{subequations}
  \label{eq:SpinorDensity}
\begin{align}
&\rho^{1,K}(\vc{r}) = \gamma^{1,K}(\vc{r};\vc{r}),\\
&\rho^{2,KK'}(\vc{r},\vc{r}') = \gamma^{2,KK'}(\vc{r},\vc{r}';\vc{r},\vc{r}'),
\end{align}
\end{subequations}
as the diagonal elements of the Pauli components of the 1 and 2RDMs. They may be thought of themselves
as Pauli components of moieties satisfying
\begin{align}
  \label{eq:Densities}
&\spn{\rho^1}(\vc{x}) = \sum_K \rho^{1,K}(\vc{r}) \otimes \pauli{K}\\
&\spn{\rho^2}(\vc{x},\vc{x}') = \sum_{KK'} \rho^{2,KK'}(\vc{r},\vc{r}') \otimes \pauli{K}\otimes \pauli{K'}
\end{align} 
In this work, we will refer to
the moieties $\spn{\rho}^{1}$ and $\spn{\rho}^{2}$ as the  one and two particle
densities, respectively.  With respect the density matrices, manipulating densities directly 
allows for drastic simplifications in the development of many--body theory as the hermiticity
of the density matrix implies that they are strictly real. In regards to the one particle density,
$\rho^{1,0}(\vc{r})$ will be referred to as the \emph{scalar} density,
$\vc{m}(\vc{r}) = \{ \rho^{1,1}(\vc{r}), \rho^{1,2}(\vc{r}), \rho^{1,3}(\vc{r}) \}$ as the magnetization
density, and $\vert \vc{m}(\vc{r}) \vert$ as the spin density \cite{Wullen02_779}.





\section{Second Quantization}
\label{sec:SQ}

While \cref{eq:SlaterDet,eq:SlaterDetSpace} provide the basic structure for the many--body fermionic wave function, its explicit form
is a bit unwieldy and thus its use for practical calculation of moieties such as expectation values more complicated than \cref{eq:SlaterCondon} 
is somewhat limited.
To this end we introduce a formalism known as second quantization, or the occupation number formalism \cite{Walecka12_book,Schuck04_book,Ostlund12_book},
which aims to greatly simplify the construction and manipulation of anti-symmetric wave functions such as those in 
\cref{eq:SlaterDet}. The primary hallmark of second quantization is in the representation of a Slater determinant
as an array of integers known as \emph{occupation numbers}, denoted $\set{n_p^I}$, indicate the inclusion (or \emph{occupation}) of elements
of $\hilb{C}$ in the configuration which describes the $I$-th determinant. As such, the length of said array is $\vert \hilb{C} \vert$
and the sum of its elements is the number of particles in the system for a representation of $\hilb{H}^N$. Due to the fact that electrons are fermions,
there is a further restriction on the possible values of the occupations numbers due to the Pauli exclusion principle,
namely that a particle can be occupied ($n^I_p=1$) or unoccupied ($n^I_p=0$), i.e. two electrons cannot occupy the same orbital. 
For the basis $\hilb{C}$ we may construct a Hilbert space, $\hilb{F}$, as a product of single particle occupations, denoted $\ket{n_p}$, such that \cite{Walecka12_book}
\begin{align}
&\ket{n_1,n_2,\ldots,n_{\vert\mathcal{C}\vert}} = \bigotimes_{i=1}^{\vert \mathcal{C} \vert} \ket{n_i}, 
  \quad \ket{n_1,n_2,\ldots,n_{\vert\mathcal{C}\vert}} \in \hilb{F},
\end{align}
where we may define inner and outer products as
\begin{subequations}
  \label{eq:OccOrthoComp}
\begin{align}
&\inner{n_1,n_2,\ldots,n_{\vert\mathcal{C}\vert}}{n'_1,n'_2,\ldots,n'_{\vert\mathcal{C}\vert}} = \prod_{i = 1}^{\vert\mathcal{C}\vert} \delta_{n_i n_i'},\\
&\sum_{n_1n_2\cdots} \ket{n_1,n_2,\ldots,n_{\vert\mathcal{C}\vert}}\bra{n_1,n_2,\ldots,n_{\vert\mathcal{C}\vert}} = \op{1}_\hilb{F}.
\end{align}
\end{subequations}
Here, $\op{1}_\hilb{F}$ is the identity operator on $\hilb{F}$. $\hilb{F}$ is commonly referred to as the Fock space.
%To avoid confusion in the subsequent developments, remark that operators on $\hilb{H}^N$ 
%\emph{do not} act on $\hilb{F}$, but rather may act indirectly through the injective map
%$\hilb{H}^N \rightarrow \hilb{F}$.

One may construct representations of Slater determinants (\cref{eq:SlaterDet}) in a consistent manner in this formalism through
the introduction of two sets of operators, $\set{ c_p^\dagger : \hilb{F} \rightarrow \hilb{F}}$ and $\set{ c_p : \hilb{F} \rightarrow \hilb{F} }$, which are referred to as creation and annihilation operators 
respectively, and the notion of a zero particle occupation vector known as the physical vacuum, $\Vac \in \hilb{F}$, such that
\begin{subequations}
  \label{eq:SeQuantAction}
\begin{align}
  &\Vac = \ket{[0_1, 0_2, \ldots, 0_{\vert C \vert}]}, \\
  &c_p^\dagger \ket{[n_1, n_2, \ldots, n_p, \ldots, n_{\vert C \vert}]} = \frac{\delta_{n_p0}}{\sqrt{N+1}} \ket{[n_1, n_2, \ldots, 1_p, \ldots, n_{\vert C \vert}]},\\
  &c_p \ket{[n_1, n_2, \ldots, n_p, \ldots, n_{\vert C \vert}]} = {\delta_{n_p1}}{\sqrt{N}} \ket{[n_1, n_2, \ldots, 0_p, \ldots, n_{\vert C \vert}]}.
\end{align}
\end{subequations}
Here, $N$ is the number of particles in the occupation vector \emph{before} action. To ensure anti-symmetry in the representation of
the wave function, we may specify the following commutation relations
\begin{subequations}
  \label{eq:SeQuantComm}
\begin{align}
  &[c_p^\dagger, c_q^\dagger]_+ = 0,\\
  &[c_p, c_q]_+ = 0,\\
  &[c_p, c_q^\dagger]_+ = \delta_{pq},
\end{align}
\end{subequations}
where $[\cdot,\cdot]_+$ is the anti--commutator. The action of products consisting on any number of these 
operators may be derived inductively from \cref{eq:SeQuantAction,eq:SeQuantComm}. Such products are often referred to as
strings. As such, we may recast \cref{eq:SlaterDet} as a string of creation operators,
\begin{equation}
  \label{eq:SeQuantSlaterDet}
  \ket{\Phi^N_I} \mapsto \ket{\Phi^N_I}_\hilb{F}  = \left(\prod_{i \in \hilb{K}^N_I} c_i^\dagger\right) \Vac,
\end{equation}
where $\ket{\Phi^N_I}_\hilb{F} \in \hilb{F}$ is the occupation vector representation of $\ket{\Phi^N_I}$.



There are a number of remarkable simplifications that arise from the adoption of second quantization
for the practical treatment of many--body quantum theory. The first that we will discuss here
relates back to the completeness of the basis of Slater determinants in the construction
of many--body fermionic wave functions, i.e. all many body wave functions may be constructed 
as a linear combination of all of the unique Slater determinants which may be constructed
from $\mathcal{C}$ (\cref{eq:SDBasis}), given that $\mathcal{C}$ is a complete basis for $\hilb{H}^1$. 
To this end, we introduce the notion of an $N$--particle reference
determinant which will be denoted $\ket{0^N}$ and is described by some configuration $\hilb{K}^N_0$.
$\hilb{K}^N_0$ will be referred to as the \emph{occupied} space (denoted $\hilb{O} = \hilb{K}^N_0$), 
and elements of the occupied space will be denoted with a subscript $i,j,k,\dots$,  i.e.
$\ket{\phi_i} \in \hilb{O}$. Similarly, we will refer the elements of $\hilb{C}$ which are 
not in $\hilb{O}$ as the unoccupied (or \emph{virtual}, denoted $\hilb{V} = \hilb{C} \setminus \hilb{O}$) space and elements of this space
will be denoted with a subscript $a,b,c,\ldots$, i.e. $\ket{\phi_a} \in \hilb{V}$. Subscripts $p,q,r,\ldots$
will remain as the notation for a general element of $\hilb{C}$ from either $\hilb{O}$ or $\hilb{V}$.
Clearly, all $M$ particle Slater determinants (with $M$ not necessarily equal to $N$) which may be constructed from $\hilb{C}$ may be constructed 
by constructing a configuration from $N_o$ elements of $\hilb{O}$ and $N_v$ elements of $\hilb{V}$
such that $N_o + N_v = M$. Thus, logically, it follows that all $M$ particle determinants may be
constructed by removing $ n_o = \min{(N-M,0)} + P $ orbitals from $\hilb{O}$ (with $P\in[0,\max{(N,N-M)}]$) and replacing them $ n_v = \max{(M-N ,0)} + P$ orbitals
from $\hilb{V}$. In the language of second quantization, the act of removing and replacing orbitals 
is made simple through the introduction of so called transition operators
\begin{align}
\left(\ket{0^N}_{ijk\cdots}^{abc\cdots}\right)_\hilb{F} = \tau_{ijk\cdots}^{abc\cdots} \ket{0^N}_\hilb{F},
\end{align}
where
\begin{align}
\tau_{ijk\cdots}^{abc\cdots} = \left(c_a^\dagger c_b^\dagger c_c^\dagger \cdots\right) \left( c_i c_j c_k \cdots\right).
\end{align}
Upon action, $\ket{0^N}_{ijk\cdots}^{abc\cdots}$ is an $ M = N - n_o + n_v $ particle Slater determinant built from removing
orbitals $i,j,k,\ldots$ from $\ket{0^N}$ and replacing them with orbitals $a,b,c,\ldots \in \hilb{V}$. The set of all
$\tau$ operators generates all $M$--particle Slater determinants from the $N$--particle reference \todo{cite Dalgaard}.
It is worth noting at this point that due to that anti-commutation relations of \cref{eq:SeQuantComm}, $\tau^{abc\cdots}_{ijk\cdots}$
is anti symmetric with respect to permutation of adjacent creation and annihilation operators, i.e.
\begin{equation}
  \label{eq:AntisymmTau}
  \tau^{abc\cdots}_{ijk\cdots} = - \tau^{acb\cdots}_{ijk\cdots} = - \tau^{abc\cdots}_{jik\cdots} = \tau^{acb\cdots}_{jik\cdots}.
\end{equation} 
Thus, necessarily, transition operators with repeated indices are zero, $\tau^{aab\cdots}_{ijk\cdots} = \tau^{abc\cdots}_{iij\cdots} = 0$.
Within this ansatz for constructing arbitrary Slater determinants from a reference, we may recast \cref{eq:SDBasis} 
as
\begin{equation}
  \label{eq:CIExp}
  \ket{\Psi^N_n} = \left(t^n\right)_0\ket{0^N} + \sum_{ai} \left(t^n\right)_i^a \ket{0^N}_i^a + 
                   \sum_{abij} \left(t^n\right)_{ij}^{ab} \ket{0^N}_{ij}^{ab} + \sum_{abcijk} \left(t^n\right)_{ijk}^{abc} \ket{0^N}_{ijk}^{abc} + \cdots,
\end{equation}
where the transition amplitudes, $\vc{t}^n$, have taken the place of the $D^n_I$ amplitudes of the original expansion. Based on this
construction, the components of $\vc{t}^n$ must also constitute an anti-symmetric tensor in the same manner as \cref{eq:AntisymmTau}.
The expansion of 
\cref{eq:CIExp} may be further cast into a more general, and convenient form in terms of explicit action of the transition
operators such that for an arbitrary $M$ particle Slater determinant,
\begin{equation}
  \label{eq:ExcitationOp}
  \ket{\Psi^M_n}_\hilb{F} = \op{R}_n^{M,N} \ket{0^N}_\hilb{F},
\end{equation}
where $\op{R}_n^{N,M}$ is an ``excitation" operator which is defined as
\begin{align}
&\op{R}_n^{M,N} = 
  \begin{cases}
    \displaystyle \sum_{abc\cdots} \left( t^n \right)^{abc\cdots} \tau^{abc\cdots} + \sum_{abc\cdots di} \left( t^n \right)^{abc\cdots d}_i \tau^{abc\cdots d}_i +
      \sum_{abc\cdots deij} \left( t^n \right)^{abc\cdots de}_{ij} \tau^{abc\cdots de}_{ij} + \cdots & M > N \\
    \displaystyle\left( t^n \right)_0  + \sum_{ai} \left( t^n \right)^{a}_i \tau^{a}_i +
      \sum_{abij} \left( t^n \right)^{ab}_{ij} \tau^{ab}_{ij} + \sum_{abcijk} \left( t^n \right)^{abc}_{ijk} \tau^{abc}_{ijk} + \cdots & M = N \\
    \displaystyle \sum_{ijk\cdots} \left( t^n \right)_{ijk\cdots} \tau_{ijk\cdots} + \sum_{a ijk\cdots l} \left( t^n \right)^{a}_{ijk\cdots l} \tau^{a}_{ijk\cdots l} +
      \sum_{abijk\cdots lm} \left( t^n \right)^{ab}_{ijk\cdots lm} \tau^{ab}_{ijk\cdots lm} + \cdots & M < N \\
  \end{cases}
\end{align}
Here the extent of extended sums of transition operators involving unbalanced creating and annihilation operators in the cases $M\neq N$ is to be understood from
the context, i.e. for the special case that $M= N+1$, we obtain
\begin{align}
&\op{R}_n^{N+1,N} = \sum_a \left( t^n \right)^a \tau^a + \sum_{abi} \left( t^n \right)^{ab}_i \tau^{ab}_i + 
  \sum_{abcij} \left( t^n \right)^{abc}_{ij} \tau^{abc}_{ij} + \cdots
\end{align}
The remarkable thing about \cref{eq:ExcitationOp} is that all possible many body wave functions may constructed from a single Slater determinant
within an operator formalism. Thus one need only understand the manipulation of $\op{R}^{M,N}_n$ in the context of other many--body operators
and how their combined action affects the reference determinant to to determine expectation values: the primary moiety which we will 
manipulate in this work.

To this point, the next property of second quantization which we will exploit in this work,
and the one that we will use most often, is the fact that one may
cast $M$--particle operators on $\hilb{H}^N$ as scalar contractions of $2M$--element operator 
strings with rank--$2M$ tensors on $\hilb{H}^1$ \cite{Walecka12_book,Schuck04_book,Ostlund12_book},
\begin{subequations}
\label{eq:SeQuantOp}
\begin{align}
  &\sum_i \op{O}^1(i) \mapsto \op{O}_\hilb{F}^1 = \sum_{pq}   O^1_{pq} c^\dagger_p c_q,\\
  &\sum_{i\neq j}\op{O}^2(i,j) \mapsto \op{O}_\hilb{F}^2 = \sum_{pqrs} O^2_{qprs} c^\dagger _p c^\dagger _q c_r c_s.
\end{align}
\end{subequations}
The tensors $O^1_{pq}$ and $O^2_{qprs}$ are defined as in \cref{eq:DiracInts}. This is a truly remarkable
result from a practical perspective at it allows a simple prescription  for the evaluation of 
inner products for general elements $\ket{\Psi_n^N}, \ket{\Psi_m^N} \in \hilb{H}^N$
\begin{subequations}
\label{eq:SeQuantInner}
\begin{align}
&\innerop{\Psi_n^N}{\sum_i \op{O}^1(i)}{\Psi_m^N} = \sum_{pq} O^1_{pq} \innerop{\Psi_n^N}{c_p^\dagger c_q}{\Psi_m^N}_\hilb{F}, \\
&\innerop{\Psi_n^N}{\sum_{i\neq j} \op{O}^2(i,j)}{\Psi_m^N} = \sum_{pq} O^1_{qprs} \innerop{\Psi_n^N}{c_p^\dagger c_q^\dagger c_r c_s}{\Psi_m^N}_\hilb{F}.
\end{align}
\end{subequations}
The moieties $\innerop{\cdot}{\cdot}{\cdot}_\hilb{F}$ denote that all quantities in that inner product are expressed in second quantization.
To demonstrate the utility of this concept, we examine the substitution of the reference excitation ansatz from \cref{eq:ExcitationOp}
into the one--body inner product such that
\begin{align}
\label{eq:SeQuantInnerExp}
&\innerop{\Psi_n^N}{c_p^\dagger c_q}{\Psi_m^N}_\hilb{F} = \innerop{0^N}{\op{R}^{N,N\dagger}_n c_p^\dagger c_q \op{R}^{N,N}_m}{0^N}_\hilb{F} \nonumber \\
&\qquad= \left( t^n \right)^*_0 \left(
     \left( t^m \right)_0 \innerop{0^N}{c_p^\dagger c_q}{0^N}_\hilb{F} +
     \sum_{ai} \left( t^m \right)^{a}_i \innerop{0^N}{c_p^\dagger c_q \tau^{a}_i}{0^N}_\hilb{F} + \cdots
   \right) + \nonumber \\
&\qquad \quad \sum_{ai}\left( t^n \right)^{a*}_i \left(
         \left( t^m \right)_0 \innerop{0^N}{\tau_i^{a\dagger} c_p^\dagger c_q}{0^N}_\hilb{F} +
         \sum_{bj} \left( t^m \right)^{b}_j \innerop{0^N}{\tau_i^{a\dagger} c_p^\dagger c_q \tau^{b}_j}{0^N}_\hilb{F} + \cdots
       \right) + \nonumber\\
&\qquad \quad \cdots
\end{align}
As first glance, it may seems as though casting the inner product into a second quantized form has drastically complicated its evaluation.
However, upon careful introspection, one sees that the rather complicated task of evaluating inner productions for determinants 
of the form \cref{eq:SlaterDet} or \cref{eq:SlaterDetSpace}, which involve the evaluation of rather complicated many body integrals,
into tensor contractions of few body integrals (\cref{eq:DiracInts}) and transition amplitudes ($\vc{t}^n$),
with inner products involving strings of creation and annihilation operators. Given the prior, one may heavily exploit the
commutation relationships of \cref{eq:SeQuantComm} to easily evaluate inner products of arbitrary operator strings through
various diagrammatic techniques or explicitly by Wick's theorem \cite{Walecka12_book,Schuck04_book,Ohrn73_book} \todo{more cites}.

\Cref{eq:SeQuantOp} may be cast into a more illuminating form by the introduction of the so called field operators which connect
the second quantized and spinor representations of operators and expectation values \cite{Walecka12_book,Schuck04_book,Yang89_book,Ohrn73_book},
\begin{subequations}
\label{eq:FieldOperators}
\begin{align}
\psi (\vc{x}) = \sum_p \phi_p(\vc{x}) c_p &\quad \text{ field annihilation operator},\\
\psi^\dagger(\vc{x}) = \sum_p \phi^*_p(\vc{x}) c^\dagger_p &\quad \text{ field creation operator}.
\end{align}
\end{subequations}
%Analogously to \cref{eq:SeQuantComm}, the field operators obey the following commutation relations,
%\begin{subequations}
%  \label{eq:FieldComm}
%\begin{align}
%  &[\psi^\dagger(\vc{x}), \psi^\dagger(\vc{x}')]_+ = 0,\\
%  &[\psi(\vc{x}), \psi(\vc{x}')]_+ = 0,\\
%  &[\psi(\vc{x}), \psi^\dagger(\vc{x}')]_+ = \delta^4(\vc{x} - \vc{x}'),
%\end{align}
%\end{subequations}
From these definitions and the expressions for the tensor elements of \cref{eq:DiracInts}, it is clear
that the second quantized operator representations may be cast as \cite{Walecka12_book,Yang89_book},
\begin{subequations}
\label{eq:FieldOpRep}
\begin{align}
  &\op{O}_\hilb{F}^1 = \int_{\mathbb F} \psi^\dagger(\vc{x}_1) \,\spop{O}^1(\vc{x}_1) \,\psi(\vc{x}_1) \,\,\dd^4 \vc{x}_1, \\
  &\op{O}_\hilb{F}^2 = \iint_{\mathbb F} \psi^\dagger(\vc{x}_2)\psi^\dagger(\vc{x}_1) \,\spop{O}^2(\vc{x}_1,\vc{x}_2) \,\psi(\vc{x}_1)\psi(\vc{x}_2) 
    \,\,\dd^4 \vc{x}_1\dd^4 \vc{x}_2.
\end{align}
\end{subequations}
Here the action of the 1-- and 2--particle spinor operators on the field operators is understood to be the same
as their action on the products on spinor orbitals as in \cref{eq:DiracInts}, i.e. tensor contraction. Upon immediate
inspection, \cref{eq:FieldOpRep} offers no simplification over \cref{eq:SeQuantOp}. However, we may recast 
\cref{eq:FieldOpRep} as the expectation value of a non--local operator, which in the case of one particle operators
yields,
\begin{align}
  &\op{O}_\hilb{F}^1 = \sum_{\sigma_1\sigma'_1}\int_{\mathbb R^3} 
    \left.\left(\op{O}^{1,\sigma_1\sigma'_1} (\vc{r}_1) \, \psi^\dagger(\vc{r}'_1,\sigma'_1) \psi(\vc{r}_1,\sigma_1) \right) \right\vert_{\vc{r}'_1 = \vc{r}_1 } 
    \,\dd^3 \vc{r}_1. 
\end{align}
Here we have used the fact that as a single particle operator, operators which depended on primed indices to not act on those
without primes, and visa verse. In analogy to \cref{eq:GenDenMatInner,eq:RDMSpinBasis}, we can define a second quantized analogue to
the 1RDM such that,
\begin{align}
  \label{eq:DenOpInner}
  &\op{O}_\hilb{F}^1 = \int_{\mathbb F} 
    \left.\left(\spop{O}^{1} (\vc{x}_1) \, \spop{\gamma}^1(\vc{x}_1; \vc{x}'_1) \right) \right\vert_{\vc{x}'_1 = \vc{x}_1 } 
    \,\dd^4 \vc{x}_1, \qquad \spop{\gamma}^1(\vc{x},\vc{x}') = \psi^\dagger(\vc{x}') \psi(\vc{x}).
\end{align}
In this work, we will refer to $\spop{\gamma}^1$ as the one--particle density matrix operator due to the fact that for
a particular vector $\ket{\Psi} \in \hilb{H}^N$ we obtain
\begin{equation}
\label{eq:DenMatOpRelation}
\gamma^1(\vc{x},\vc{x}') = \innerop{\Psi}{\spop{\gamma}^1(\vc{x},\vc{x}')}{\Psi}_\hilb{F},
\end{equation}
where $\gamma^1$ is the 1RDM corresponding to $\ket{\Psi}$. In further analogy to the treatment of density
matrices in \cref{sec:DenMat}, we may define the Pauli components of the one-particle density matrix operators,
\begin{subequations}
  \label{eq:PauliDenMatOp}
\begin{align}
  &\op{\gamma}^{1,0}(\vc{r};\vc{r}') = \frac{1}{2}   \left(\op{\gamma}^1{}(\vc{r},\alpha;\vc{r}',\alpha) + \op{\gamma}^1(\vc{r},\beta; \vc{r}',\beta )\right) , \\
  &\op{\gamma}^{1,1}(\vc{r};\vc{r}') = \frac{1}{2}   \left(\op{\gamma}^1{}(\vc{r},\alpha;\vc{r}',\beta ) + \op{\gamma}^1(\vc{r},\beta; \vc{r}',\alpha)\right) , \\
  &\op{\gamma}^{1,2}(\vc{r};\vc{r}') = \frac{\ii}{2} \left(\op{\gamma}^1{}(\vc{r},\alpha;\vc{r}',\beta ) - \op{\gamma}^1(\vc{r},\beta; \vc{r}',\alpha)\right) , \\
  &\op{\gamma}^{1,3}(\vc{r};\vc{r}') = \frac{1}{2}   \left(\op{\gamma}^1{}(\vc{r},\alpha;\vc{r}',\alpha) - \op{\gamma}^1(\vc{r},\beta; \vc{r}',\beta )\right) .,
\end{align}
\end{subequations}
such that \cref{eq:DenOpInner} becomes,
\begin{equation}
  \op{O}_\hilb{F}^1 = 2\sum_K \int_{\mathbb R^3} 
    \left.\left(\op{O}^{1,K} (\vc{r}_1) \, \op{\gamma}^{1,K}(\vc{r}_1; \vc{r}'_1) \right) \right\vert_{\vc{r}'_1 = \vc{r}_1 } 
    \,\dd^3 \vc{r}_1.
\end{equation}
In particular, if $\op{O}^1$ is void of any differential character, we may recast
\begin{equation}
  \op{O}_\hilb{F}^1 = 2\sum_K \int_{\mathbb R^3} 
    \op{O}^{1,K} (\vc{r}_1) \, \op{\rho}^{1,K}(\vc{r}_1)  
    \,\dd^3 \vc{r}_1,
\end{equation}
where we have defined an analogy to the Pauli components of the spinor density as
the \emph{density operators},
\begin{equation}
\op{\rho}^{1,K}(\vc{r}) \equiv \op{\gamma}^{1,K}(\vc{r},\vc{r}).
\end{equation}
%It follows immediately that these operators are the Pauli components of a spinor operator,
%\begin{equation}
%\spop{\rho}^1 (\vc{x}) = \sum_K \op{\rho}^{1,K}(\vc{r}) \otimes \pauli{K},
%\end{equation}
%which we will refer to as the spinor density operator. 
It follows from \cref{eq:PauliDenMatOp} that
\begin{equation}
\rho^K(\vc{r}) = \innerop{\Psi}{\op{\rho}^K(\vc{r})}{\Psi}.
\end{equation}
Consistent analogies for the 2RDM can be made, but as the primary focus in this work will be the manipulation
of single Slater determinants, all of resulting arithmetic can be done in terms of the 1RDM and analogous
spinor operators. It should be noted that unlike most operators, the expectation values of 
$\spop{\gamma}^1$ and $\spop{\rho}^1$ are tensor fields on $\mathbb R^3$, rather than scalar fields.

\todo{Note that rho does not consist of the elements of a tensor operator}
\todo{Transition densities}
%One of the key aspects of defining the density operators as opposed to the densities and density matrices
%themselves is that it allows for proper generalization to off diagonal density analogues, i.e.
%for $\ket{\Psi^N_m}, \ket{\Psi^N_n} \in \hilb{H}^N$,
%\begin{equation}
%T^{mn}(\vc{x}) =
%\end{equation}

In essence, second quantization provides a common language which allows one to factor elements of a particular many--body
quantum theory into parts which are dependent and independent of the single--particle basis. Such a state of affairs has
been demonstrated in \cref{eq:SeQuantInner,eq:SeQuantInnerExp}. If one can define an inner product on $\hilb{H}^1$ and
is able to develop an ansatz for the form of \cref{eq:ExcitationOp} for a particular theory, it is immediately compatible
with second quantization. In this work, second quantization will play a crucial role in the translation of the
results of non--relativistic quantum mechanics to the consistent treatment of relativistic effects.



\section{Time Independent Solutions of the Wave Equation}
\label{sec:MF}

Examination of the time--independent solutions of \cref{eq:QWave} plays a central role in the development
of both time--dependent and time--independent quantum theory. Thus, for the time being, we will restrict
ourselves to Hamiltonians which are independent of time, though this will not be the case in general
(see \cref{sec:SCLMI} for instance).
Within this consideration, the electronic Hamiltonians one typically encounters for molecular systems with $N$ electrons,
$\op{H}_{el} : \hilb{H}^{N_{el}} \rightarrow \hilb{H}^{N_{el}}$ take the general form
\begin{equation}
\label{eq:GenElHam}
\op{H}_{el} = \sum_{i}^{N} \left( \op{h}(i) + \op{v}(i) \right) + \frac{1}{2}\sum_{i \neq j}^{N} \op{g}(i,j),
\end{equation}
where $\op{h}$ and $\op{v}$ are the one--body free particle Hamiltonian and potential operators respectively.
$\op{g}$ is a two--body operator which describes the interaction between two electrons.
At this point, the explicit structure of these operators are immaterial to the following developments; simply
noting that the presence of $\op{g}$ renders direct solution of \cref{eq:QWave} for the many body
wave function impractical \todo{cite}. Nevertheless, one may write down a general form for the solution
of \cref{eq:QWave} with this Hamiltonian as \todo{cite Sakurai}
\begin{equation}
\ket{\Psi (t)} = \exp \left[ -i(t - t_0) \op{H}_{el} \right] \ket{\Psi (t_0) },
\end{equation}
where $t_0$ is a reference time point for which one may specify an initial condition for the electronic 
wave function. Without loss of generality, we will typically take $t_0 = 0$. We now consider the eigen
spectrum of the electronic Hamiltonian such that there exists a set of wave functions $\ket{\Psi_n} \in \hilb{H}^N$
which yield
\begin{equation}
\label{eq:TIQWave}
\op{H}_{el} \ket{\Psi_n} = E_n \ket{\Psi_n},
\end{equation}
where $E_n$ is the energy eigenvalue of $\ket{\Psi_n}$. The pairs $(\ket{\Psi_n},E_n)$ will be referred to
as the adiabatic electronic states, and  \cref{eq:TIQWave} as the time--independent quantum wave equation 
in the following. As a spectral decomposition of $\op{H}_{el}$, $\hilb{H}^{N} = \Span{\set{\ket{\Psi_n}}}$
such that any arbitrary vector in $\hilb{H}^{N}$ may be decomposed as
\begin{equation}
\ket{\Psi (t) } = \sum_n c_n(t) \ket{\Psi_n},\qquad c_n(t) = \inner{\Psi_n}{\Psi(t)}.
\end{equation}
Here, the coefficients are assumed to be unit normal, i.e. $\sum_n\vert c_n (t) \vert^2 = 1$. 
Note that the coefficients $c_n(t)$ are different from those of \cref{eq:SDBasis} as the
adiabatic states are in general \emph{not} Slater determinants due to the presence of $\op{g}$.


Suppose that we specify the initial condition of the wave function to be an adiabatic state
of the electronic Hamiltonian, $\ket{\Psi(0)} = \ket{\Psi_m}$. \Cref{eq:TIQWave} immediately 
yields the action of the Hamiltonian on the state such that
\begin{equation}
\label{eq:StatSol}
\ket{\Psi(t)} = \exp \left[ -it E_m \right] \ket{\Psi_m}. 
\end{equation}
Time dependent solutions such as this comprise a special class of solutions to \cref{eq:QWave} 
known as stationary states. As $E_m$ is a real constant (due to the hermiticity of $\op{H}_{el}$),
\cref{eq:StatSol} is referred to as stationary as it trivially rotates through the complex plane
such that for operators void of differentiation with respect to time, $\op{O}^N$
\begin{equation}
\innerop{\Psi(t)}{\op{O}^N}{\Psi(t)} = \innerop{\Psi_m}{\op{O}^N}{\Psi_m} \quad \forall t,
\end{equation}
i.e. the state itself is not, strictly speaking time independent, but its expectation values are.
Of particular interest to this work, at least in the development of a theoretical framework, will be 
the so called ground electronic state: the electronic adiabat which has the lowest energy, $(E_0,\ket{\Psi_0})$.
In practice, one may determine the ground state by minimizing the energy functional,
\begin{equation}
\label{eq:EnergyFunctional}
E[\Psi] = \frac{\innerop{\Psi}{\op{H}_{el}}{\Psi}}{\inner{\Psi}{\Psi}},
\end{equation}
which admits a sole minimum at the ground electronic state due to the variational theorem \todo{cite}.

\section{Mean Field Quantum Mechanics and Basis Set Expansions}

\subsection{The Hartree--Fock Approximation}
\label{sec:HF}
Up to this point, no approximations in the treatment of the many body wave function have been introduced. That is
to say, given that $\mathcal{C}$ is a complete, countable basis for $\hilb{H}^1$, all of the developments
in this and the previous sections treat the electronic many body problem \emph{exactly}. However, even a cursory
inspection of the nature of $\mathcal{C}$ indicates that it must be countably infinite in order to 
satisfy completeness on $\hilb{H}^1$ \todo{confirm, cite?}. Thus it must be truncated in some systematic
way in order for it to be useful in any practical treatment of the many body problem.
Unfortunately, any truncation of \cref{eq:SDBasis} or \cref{eq:CIExp} will make it impossible in general to
construct exact many body states, such as those of \cref{eq:TIQWave}.
In this work, we will be primarily concerned with optimizing a set $\mathcal{C}$ such that
we may construct a reference determinant (per \cref{eq:ExcitationOp}) which minimizes the energy functional
in \cref{eq:EnergyFunctional} over all possible Slater determinants. In effect, such a minimization will
construct the best single Slater determinant approximation to the many--body ground electronic state
which may serve as a reference determinant in the context of \cref{eq:ExcitationOp} for better approximations
for  true many body wave functions. The approximation which describes the ground state wave function as the 
minimizing Slater determinant of \cref{eq:EnergyFunctional} is known as the Hartree--Fock (HF) approximation \todo{cites}.

Before delving into the specifics of how one might go about optimizing $\mathcal{C}$ to minimize 
\cref{eq:EnergyFunctional}, one might note that in the absence of $\op{g}$, i.e. many body operators
of the form
\begin{equation}
\label{eq:FockOp}
\op{f}^N = \sum_i^{N} \op{f}^1(i), \qquad \op{f}^1(i) = \op{h}(i) + \op{v}^\mathrm{eff}(i),
\end{equation}
admit eigen functions which may be written as single Slater determinants. That is to say
that given a set $\mathcal{C}$ whose elements are eigen functions of $\op{f}^1(1)$,
\begin{equation}
\label{eq:FockEq}
\op{f}^1(1) \ket{\phi_p} = \ket{\phi_p} \epsilon_p,
\end{equation}
the Slater determinants which may be constructed from $\mathcal{C}$ are eigen functions of
\cref{eq:FockOp} \cite{Ostlund12_book}. Here, $\epsilon_p$ will be referred to as the orbital
eigenenergy corresponding to $\ket{\phi_p}$. Due to the lack of explicit two body interaction
in \cref{eq:FockOp}, the determinants which may be constructed from the solutions of \cref{eq:FockEq} will be referred
to as a \emph{non--interacting} system, i.e. they particles are treated as independent particles
in an effective external potential $\op{v}^\mathrm{eff}$. Given an configuration subset, $\mathcal{K}^{N}_I \subset \mathcal{C}$,
the energy of a non--interacting system is given by
\begin{equation}
\label{eq:NIE}
E_\mathrm{NI} = \sum_{i \in \mathcal{K}_I^{N}}  \epsilon_i.
\end{equation}
If one wishes to minimize
$E_\mathrm{NI}$ over the possible Slater determinants which may be constructed from $\mathcal{C}$,
one must select the $N$ orbitals with the lowest eigenenergy to build the determinant.
This choice yields an ordering on $\mathcal{C}$ which orders orbitals in increasing orbital eigenenergy.
This ordering also happens to correspond identically to the phenomenological ordering dictated by the
Aufbau principle \todo{cite}.

Due to the fact that a single Slater determinant is completely described by its orbital configuration,
minimizing \cref{eq:EnergyFunctional} over Slater determinants amounts to determining a configuration
which minimizes the Hamiltonian expectation value. Thus it is useful to think of \cref{eq:EnergyFunctional}
not as a functional of a single parameter, but rather as a functional of several parameters, namely $N$
orbitals which comprise some configuration. However, as was discussed in \cref{sec:DenMat}, if we wish
to utilize $N$--representable density matrices in the development of our theories, we must constrain
the choice of possible configurations to only include those which contain orthonormal orbitals. To this
end we will employ the a constrained energy Lagrangian \cite{Ostlund12_book},
\begin{equation}
\mathcal{L}[\mathcal{K}^N] = E^\mathrm{HF}[\mathcal{K}^N] - \sum_{i,j \in \mathcal{K}^N} \lambda_{ij} \left( \inner{\phi_i}{\phi_j} - \delta_{ij} \right),
\end{equation}
where $\mathcal{K}^N$ is an arbitrary $N$ orbital configuration and $\lambda_{ij}$ is a set of to--be--determined Lagrange multipliers
which will constrain our choice of orbitals to be orthonormal. HF has been used to denote the fact that this form is only valid
for single Slater determinants.
A necessary condition for minimization of \cref{eq:EnergyFunctional} in this context is given by
\begin{equation}
\label{eq:ConstrMin}
\frac{\delta \mathcal{L}[\mathcal{K}^N]}{\delta \phi_i (\vc{x})} = 0 ,\quad \forall \phi_i(\vc{x}) \in \mathcal{K}^N,
\end{equation}
where derivatives set to zero are \emph{functional derivatives}. 

The energy functional for a configuration consisting of orthonormal orbitals is often most simply 
written as a functional of the 1RDM which may be constructed from that configuration \cite{Yang89_book},
\begin{align}
&E^\mathrm{HF}[\mathcal{K}^N] = E[\gamma^1] = H^\core[\gamma^1] + J[\gamma^1] - K[\gamma^1],
\end{align}
where we have defined
\begin{align}
  &H^\core[\gamma^1] = 
    \iint_\mathbb{F} \left( \spop{h}(\vc{x}_1; \vc{x}_1') + \spop{v}(\vc{x}_1; \vc{x}_1')\right) \gamma^1(\vc{x}'_1; \vc{x}_1) \, \dd^4\vc{x}_1 \dd^4\vc{x}'_1\,\,  \label{eq:CoreFunc}
    \\
  &J[\gamma^1] = \iiiint_\mathbb{F} \spop{g}(\vc{x}_1, \vc{x}_2; \vc{x}_1', \vc{x}_2') 
      \gamma^1(\vc{x}'_1; \vc{x}_1)\gamma^1(\vc{x}'_2; \vc{x}_2) 
    \dd^4\vc{x}_1 \dd^4\vc{x}'_1\dd^4\vc{x}_2 \dd^4\vc{x}'_2 \label{eq:ClassCoulFunc}\\
  &K[\gamma^1] = \iiiint_\mathbb{F} \spop{g}(\vc{x}_1, \vc{x}_2; \vc{x}_1', \vc{x}_2') 
      \gamma^1(\vc{x}'_1; \vc{x}_2)\gamma^1(\vc{x}'_2; \vc{x}_1) 
    \dd^4\vc{x}_1 \dd^4\vc{x}'_1\dd^4\vc{x}_2 \dd^4\vc{x}'_2 
\end{align}
\Cref{eq:ConstrMin} then becomes
\begin{equation}
    \label{eq:RawHFEq}
    \iint_{\mathbb F}\frac{\delta E^\mathrm{HF}[\mathcal{K}^N]}{\delta \gamma^{1} (\vc{x}'_1; \vc{x}_1)} \frac{\delta \gamma^{1}(\vc{x}'_1; \vc{x}_1)}{\delta \phi_i(\vc{x})} \,\dd^4\vc{x}'_1 \dd^4 \vc{x}_1 - 
    \sum_{j,k \in \mathcal{K}^N} \lambda_{jk} \frac{\delta \inner{\phi_j}{\phi_k}}{\delta \phi_i(\vc{x})}  = 0 ,\quad \forall \phi_i(\vc{x}) \in \mathcal{K}^N.
\end{equation}
Due to the hermiticity of both $\gamma^1$ and the inner products $\inner{\phi_p}{\phi_q}$, \cref{eq:RawHFEq} will factor into two parts which are 
conjugates of each other, thus requiring both terms to go to zero simultaneously. By rearranging the summation indices, \cref{eq:RawHFEq} may be written as
\begin{align}
  \label{eq:HFNonCan}
  \sum_{j \in \mathcal{K}^N}
  \int_{\mathbb F} \left( \spop{F}^\HF(\vc{x}_1; \vc{x}_1')\delta_{ij}   - \delta^4(\vc{x} _1- \vc{x}_1')\lambda_{ij} \right)
    \phi_j(\vc{x}_1')\,\dd^4\vc{x}_1' = 0 ,\quad \forall \phi_i(\vc{x}) \in \mathcal{K}^N,
\end{align}
where
\begin{align}
  \label{eq:HFFockSpace}
&\spop{F}^\HF(\vc{x}; \vc{x}') = 
  \frac{\delta E^\mathrm{HF}[\gamma^1]}{\delta \gamma^{1}(\vc{x}'; \vc{x})} = \spop{H}(\vc{x};\vc{x}') + \spop{J}(\vc{x};\vc{x}') - \spop{K}(\vc{x};\vc{x}'),\\
&\spop{H}^\core(\vc{x}; \vc{x}') = \frac{\delta H[\gamma^1]}{\delta \gamma^1(\vc{x}'; \vc{x})},\qquad 
\spop{J}(\vc{x}; \vc{x}') = \frac{\delta J[\gamma^1]}{\delta \gamma^1(\vc{x}'; \vc{x})}, \qquad
\spop{K}(\vc{x}; \vc{x}') = \frac{\delta K[\gamma^1]}{\delta \gamma^1(\vc{x}'; \vc{x})} .
\end{align}
\Cref{eq:HFNonCan} is the general form of a set of non--linear integrodifferential equations known as the Hartree--Fock equations and $\spop{F}^\HF$ is
known as the Fock operator corresponding to \cref{eq:GenElHam} \todo{cite}. The non--linearity of \cref{eq:HFNonCan} comes in the fact that the functionals
$J$ and $K$ are quadratic in the density matrix, thus their functional derivative still maintains a dependence on $\gamma^1$. 
A configuration's satisfaction of \cref{eq:HFNonCan} is the minimal requirement for the corresponding Slater determinant to minimize \cref{eq:EnergyFunctional}

Using the fact that $\gamma^1$ (and thus $\spop F$) is invariant to unitary transformation of the orbitals which comprise its construction (\cref{eq:UnitarySD}),
we may choose to transform \cref{eq:HFNonCan} in such a way as to diagonalize the matrix of Lagrange multipliers, 
$\mathcal U: \lambda_{ij} \rightarrow \tilde\lambda_{i} \delta_{ij}$,
\begin{equation}
\label{eq:HFEq}
\int_{\mathbb F} \spop{F}^\HF(\vc{x}_1; \vc{x}_1') \tilde\phi_i(\vc{x}_1'), \dd^4\vc{x}'_1 = \tilde\lambda_i \tilde\phi_i(\vc{x}_1), 
  \qquad \tilde\phi_i = \sum_j\mathcal U_{ij}\phi_j. 
\end{equation}
\Cref{eq:HFEq} is known as the canonical Hartree--Fock equation and $\tilde \phi_i$ is referred to as a canonical Hartree-Fock molecular orbital (HF-MO). 
Clearly, \cref{eq:HFEq} has the same form as \cref{eq:FockOp,eq:FockEq}. Thus within the HF approximation, the effective external potential
is given by
\begin{equation}
\label{eq:HFPot}
\op{v}^\mathrm{HF} = \op{v} + \op{J} - \op{K},
\end{equation}
and we may identify the diagonalized Lagrange multipliers as orbital eigenenergies,
\begin{equation}
\tilde\lambda_i \equiv \epsilon^\HF_i.
\end{equation}
$\{ \epsilon^\HF_i \}$ will be referred to as the set of canonical HF eigenenrgies.
The realization of \cref{eq:HFPot} is rather profound in that states that the minimizing Slater determinant of \cref{eq:EnergyFunctional}
is the one constructed from a set of \emph{self--consistent, mean field} orbitals. That is to say that each of the HF-MOs which make of
this configuration in effect ``feel" the effect of the other orbitals through action of the Fock operator, i.e. the orbitals themselves 
are non--interacting, but each orbital ``knows" about all of the others through the $\op{J}$ and $\op{K}$ operators. Thus the HF wave function
(the minimizing Slater determinant) is often referred to as the mean field solution to \cref{eq:TIQWave} and thus \cref{eq:QWave}.

In and of themselves, \cref{eq:HFNonCan,eq:HFEq} represent a challenging class of nonlinear integrodifferential equations. It was in the
realization of Roothaan \todo{cite} that \cref{eq:HFEq} may instead be cast into a numerical linear algebra problem by expanding the
orbitals of $\mathcal{C}$ in a basis set,
\begin{equation}
\label{eq:LCAO}
\phi^\sigma_p(\vc{r}) = \sum_{\mu = 1}^{N_b} C^{\HF,\sigma}_{\mu p} \chi_\mu(\vc{r}),
\end{equation}
where $C^{\HF,\sigma}_{\mu p} = \inner{\chi_\mu}{\phi^\sigma_p}$ is a tensor of expansion coefficients which expand each HF-MO as a linear 
combination of $N_b$ basis functions, $\{\chi_\mu\}$. The tensor $\vc{C}^\HF$ will be referred to as the HF-MO coefficients in the following.
Substituting \cref{eq:LCAO} into \cref{eq:HFEq} projecting on the left by $\chi_\nu$, we obtain
\begin{equation}
 \sum_{\mu = 1}^{N_b} \sum_{\sigma'} \iint_{\mathbb R^3} \chi^*_\nu(\vc{r}) \op{F}^{\HF,\sigma\sigma'}(\vc{r}; \vc{r}') \chi_\mu(\vc{r}') C_{\mu i}^{\HF,\sigma'}  
   \,\dd^3\vc{r}\dd^3\vc{r}' = \sum_{\mu = 1}^{N_b} \int_{\mathbb R^3} \chi^*_\nu(\vc{r}) \chi_\mu(\vc{r}) C_{\mu i}^{\HF,\sigma} \epsilon^\HF_i \,\dd^3\vc{r},
\end{equation}
or more compactly
\begin{equation}
\label{eq:OccRHEq}
\spvc F^\HF \,\spvc C^\HF_o = \spvc S\, \spvc C^\HF_o\, \spvc \epsilon_o^\HF
\end{equation}
where the Fock matrix, $\spvc F^\HF \in \MatSpace{C}{2N_b}{2N_b}$ and overlap matrix $\spvc S \in \MatSpace{C}{2N_b}{2N_b}$
are given by
\begin{align}
&\spvc{F}^\HF = \begin{bmatrix} \vc{F}^{\HF,\saa} & \vc{F}^{\HF,\sab} \\ \vc{F}^{\HF,\sba} & \vc{F}^{\HF,\sbb} \end{bmatrix}, \\
&\spvc{S} = \begin{bmatrix} \vc{S} & \vc{0} \\ \vc{0} & \vc{S} \end{bmatrix},
\end{align}
with
\begin{equation}
  \spvc C_o^\HF = \begin{bmatrix} \vc C_o^{\HF,\alpha} \\ \vc C_o^{\HF,\beta} \end{bmatrix}
\end{equation}
\begin{subequations}
\begin{align}
  &\vc{F}^\HF=  \vc{H}^\core + \vc J - \vc K, \label{eq:HFMat}\\ 
&X_{\mu\nu}^{\sigma\sigma'} =  \iint_{\mathbb R^3} \chi^*_\nu(\vc{r}) \op{X}^{\sigma\sigma'}(\vc{r}; \vc{r}') \chi_\mu(\vc{r}')\,\dd^3\vc{r}\dd^3\vc{r}',
\quad \vc{X} \in \{ \vc{H}^\core , \vc J , \vc K \}. \\
&S_{\mu\nu} = \inner{\chi_\mu}{\chi_\nu}. 
\end{align}
\end{subequations}
Here, the rectangular matrices $\vc{C}_o^{\HF,\sigma} \in \MatSpace{C}{N_b}{N}$
are the spin--basis coefficients which expand the occupied HF-MOs (in the sense
of second quantization) in the basis set. $\spvc \epsilon_o^\HF \in
\MatSpace{R}{2N_b}{N}$ is the diagonal matrix of occupied HF-MO eigenenergies. 

Clearly, \cref{eq:OccRHEq} represents a partial diagonalization of $\spvc F^\HF$ in the case when $N < 2N_b$ and does not
yield coefficients of all of the elements of $\mathcal{C}$. If instead we examine the full diagonalization of $\spvc F^\HF$,
\begin{equation}
\label{eq:RHEq}
\spvc F^\HF \,\spvc C^\HF = \spvc S\, \spvc C^\HF\, \spvc \epsilon^\HF,
\end{equation}
where $\spvc{C}^\HF \in \MatSpace{C}{2N_b}{2N_b}, \spvc \epsilon^\HF \in \MatSpace{R}{2N_b}{2N_b}$ represent a full rank 
eigendecomposition of $\spvc F^\HF$. Thus from a set of $N_b$ basis functions, one may construct a set of orbitals
which has $\vert \mathcal{C} \vert = 2N_b$ such that $\Span{\mathcal{C}} = \Span{\{\chi_\mu\}}$. That is to say that if $\{\chi_\mu\}$
is complete on $\hilb H^1$, so is $\mathcal{C}$. \Cref{eq:RHEq} is known as the Roothaan--Hall equation.
To find the minimizing HF wave function, we must minimize \cref{eq:NIE}
by choosing the orbitals with the lowest $N$ eigenenergies from which we will construct the ground state HF configuration.
In general, \cref{eq:RHEq} may be solved interatively to self--consistency by constructing the Fock from the ground
state configuration and rediagonalizing until the eigenvectors obtained are unique up to a unitary transformation.
This procedure is known as the self--consistent field (SCF) procedure \cite{Ostlund12_book}.

\subsection{Density Functional Theory}
\label{sec:DFT}

The Hartree--Fock approximation provides an excellent reference determinant for systematic improvement
by way of \cref{eq:ExcitationOp}. However, even at very low orders of truncation, obtaining the 
coefficients for these types of expansions is often so computationally demanding that only relatively
small molecular systems are able to be studied on a routine basis. This poses a rather formidable problem
in the field of electronic structure theory in that the Hartree--Fock wave function itself is not a sufficient
description of the many--body wave function for all but the simplest problems, and any reasonable improvement
of the wave function leads to computationally intractable problems for experimentally relevant molecular 
systems. Luckily, the deficiencies of the Hartree--Fock approximation are well known: it simply lacks
explicit two--body interactions in the effective Hamiltonian, i.e. the non--interacting system is said to
lack electron correlation. Thus if one were able to account for electronic correlation 
 in an effective one--body Hamiltonian, one would not need to use expensive many--body expansions
such as \cref{eq:ExcitationOp}. To this end, we will employ density functional theory (DFT) as a means to approximately
account for electron correlation effects.

The Hohenberg--Kohn (HK) theorem states that there exists a bijection between
an external scalar potential and the scalar one--body density of the many--body
wave function \todo{cite Hohenberg}. More generally, the HK theorem may be
extended to the 1RDM if the external potential has spin structure or is
generally non--local in character \todo{cite Gilbert}. Thus the total energy of
the many--body state may be written in terms of functionals of the 1RDM,
\begin{equation}
  \label{eq:HK}
  E[\Psi] = E[\gamma^1] = F^\mathrm{HK}[\gamma^1] + 
    \iint_\mathbb{F} \op{v}(\vc{x};\vc{x}') \gamma^1(\vc{x}'; \vc{x})\, \dd^4\vc{x}\dd^4\vc{x}'
\end{equation}
where $F^\mathrm{HK}$ is an energy functional which describes the energetic contributions arising from 
all one and two body interactions separate from the external potential. $F^\mathrm{HK}$ may be
broken down into its substituent contributions as follows,
\begin{equation}
  \label{eq:HKFunc}
  F^\mathrm{HK}[\gamma^1] = H^\mathrm{free}[\gamma^1] + J[\gamma^1] + E^{xc}[\gamma^1]
\end{equation}
where $H^\mathrm{free}$ is the energy functional pertaining to the free particle part of the electronic Hamiltonian
describing the system, $J$ is the classical Coulomb functional of \cref{eq:ClassCoulFunc}, and $E^{xc}$ is the
\emph{exchange--correlation} (xc) functional which describes all purely quantum many body energetic contributions, including those
which enforce the proper spin statistics (i.e. Fermi--Dirac in the case of fermions) for the wave function. 
Thus given the exact forms of $H^\mathrm{free}$ and $E^{xc}$, one would be able to  obtain all information pertaining 
to the physical system.

There are a number of problems implicit in \cref{eq:HK,eq:HKFunc}. The most problematic is that the 
exact form of $E^{xc}$ is unknown, thus rendering the practical treatment of DFT as an exact method impossible. In
practice however, the lack of knowledge of an exact $E^{xc}$ is not so egregious that it becomes an impractical
tool for chemical inquiry. Due to the vast availability of approximate xc--functionals which have been developed to
reproduce certain physical quantities, such as energies and \todo{add more things here}, the cost--to--accuracy ratio 
over methods which use explicit many--body expansions is within the realm of tolerability for routine inquiry into chemical 
phenomena. There is another, more subtle problem relating to the form \cref{eq:HKFunc} in that the exact form for
$H^\mathrm{free}$ is just as complex as $E^{xc}$ in the case when $\gamma^1$ is derived from a truely interacting
many--body wave function \cite{Yang89_book}. While the explicit form of the 1RDM is a simple finite sum
in the case of a single Slater determinant (\cref{eq:RDMSD}), the form for the true many body wave function
contains an infinite sum over orbital products such that
\begin{equation}
  H^\mathrm{free}[\gamma^1] = \sum_{i = 1}^\infty f_i \innerop{\psi_i}{\op{H}^\mathrm{free}}{\psi_i}, \quad \text{s.t.} \quad
  \int_{\mathbb{F}} \gamma^1(\vc{x};\vc{x}') \psi_i(\vc{x}') \,\dd^4\vc{x}' = f_i\psi_i(\vc{x}).
\end{equation}
Thus in DFT methods which rely solely on the HK theorems, one must make rather crude approximations to the $H^\mathrm{free}$
functional, such as the Thomas--Fermi functional \todo{cite}, for practical calculations.

A solution to the problem of handling $H^\mathrm{free}$ in a consistent manner
comes from Kohn--Sham density functional theory (KS-DFT), where one
reintroduces the concept of a fictitious non--interacting (Slater determinant)
wave function to represent the true many--body wave function \todo{cite KS}. As
such the total energy of the system may be written as,
\begin{equation}
  \label{eq:KSFunc}
  E^\KS[\gamma_s^1] = H_s^\mathrm{free}[\gamma_s^1] + J[\gamma_s^1] + \tilde E^{xc}[\gamma_s^1] + 
    \iint_\mathbb{F} \op{v}(\vc{x};\vc{x}') \gamma_s^1(\vc{x}'; \vc{x})\, \dd^4\vc{x}\dd^4\vc{x}',
\end{equation}
where all of the moieties with an $s$ subscript denote that they relate to the
fictitious non--interaction system, i.e. a single Slater determinant such that
$\gamma^1_s$ may be composed as \cref{eq:SpinorDensityMat}. Thus
\begin{equation}
  H^\mathrm{free}_s = \sum_i \innerop{\phi_i}{\op{H}^\mathrm{free}}{\phi_i},
\end{equation}
where $\{\phi_i\}$ is the set of orbitals used to construct $\gamma^1_s$.
$\tilde E^{xc}$ is an augmented xc functional which accounts for the difference
of $H_s^\mathrm{free}$ and $H^\mathrm{free}$,
\begin{equation}
  \tilde E^{xc}[\gamma^1_s] = H^\mathrm{free}[\gamma^1_s] - H_s^\mathrm{free}[\gamma^1_s] + E^{xc}[\gamma^1_s].
\end{equation}
Due to the introduction of the non--interacting system, or equivalently a set of orbitals,
we may derive a set of SCF equations in analogy to the HF approximation which minimizes \cref{eq:KSFunc}.
In analogy of \cref{eq:HFEq}, we may write
\begin{equation}
  \label{eq:KSEqCan}
  \int_{\mathbb F} \spop{F}^\KS(\vc{x}_1; \vc{x}_1') \phi_i(\vc{x}_1') \,\dd^4\vc{x}'_1 = 
    \epsilon^\KS_i \phi_i(\vc{x}_1), 
\end{equation}
where
\begin{align}
  \label{eq:KSFockSpace}
  &\spop{F}^\KS(\vc{x}; \vc{x}') = 
    \spop{H}^\mathrm{core}(\vc{x}; \vc{x}') + \spop{J}(\vc{x}; \vc{x}') +
    \spop{V}^{xc}(\vc{x}; \vc{x}') \\
  &\spop{V}^{xc}(\vc{x}; \vc{x}') = \frac{\delta \tilde E^{xc}[\gamma^1_s]}{\delta \gamma^1_s(\vc{x}';\vc{x})},
\end{align}
and $\spop V^{xc}$ is referred to as the \emph{exchange--correlation} (xc) potential. \Cref{eq:KSEqCan}
is known generally as the Kohn--Sham (KS) equations. In comparison with \cref{eq:FockEq}, we may
identify \cref{eq:KSFockSpace} as an effective one--body operator by the effective potential for KS-DFT as,
\begin{equation}
\label{eq:KSPot}
  \op{v}^\KS = \op{v} + \op{J} + \op{V}^{xc}.
\end{equation}
In further analogy
to the treatment of the HF approximation, by examining the similarities between \cref{eq:HFFockSpace} and
\cref{eq:KSFockSpace}; setting $E^{xc}[\gamma_s^1] = -K[\gamma_s^1]$ yields the canonical HF equations.
Knowing that the HF wave function lacks explicit treatment of electron correlation, the HF approximation
may be thought of as a special case of KS-DFT where the xc functional only contains enough information
to ensure the proper (antisymmetric) particle exchange symmetry while neglecting any treatment of electron
correlation.
The attractive aspect on KS-DFT is that for relatively the same computational cost as obtaining a HF
wave function one is able to (at least approximately) treat the electron correlation of the many--body problem.
Further, unlike the HF approximation, in the limit where the orbital set $\mathcal{C}$ is taken to be complete
and one has exact forms for the $E^{xc}$ functional, KS-DFT is \emph{exact}. Neither of these limits are realizable
in practice, but nevertheless, electronic structure methods based on KS-DFT have seen enormus success in
the prediction of many chemical phenomena which would be computatioanlly inaccessable over explicit many--body methods
\todo{cite HG + others}.

Casting \cref{eq:KSFockSpace} onto a finite basis as in \cref{eq:LCAO}, we obtain a generalized eigenvalue problem analogus to
the Roothaan--Hall equation of \cref{eq:RHEq} \todo{cite Pople},
\begin{equation}
\label{eq:RH-KSEq}
\spvc F^\KS \,\spvc C^\KS = \spvc S\, \spvc C^\KS\, \spvc \epsilon^\KS,
\end{equation}
where
\begin{align}
  &\spvc F^\KS = \spvc H^\mathrm{core} + \spvc J + \spvc V^{xc},\label{eq:KSMat} \\
  &\spvc V^{xc} = \begin{bmatrix} \vc{V}^{xc,\saa} & \vc{V}^{xc,\sab} \\ \vc{V}^{xc,\sba} & \vc{V}^{xc,\sbb} \end{bmatrix},
\end{align}
\begin{align}
    \label{eq:VXCBasisNonLocal}
    &V^{xc,\sigma\sigma'}_{\mu\nu} =
    \iint_{\mathbb R^3} \chi^*_\mu(\vc{r}) \frac{\delta \tilde E^{xc}[\gamma^1_s]}{\delta \gamma^{\sigma'\sigma}(\vc{r}';\vc{r})} \chi_\nu(\vc{r}')\,\dd^3\vc{r}\dd^3\vc{r}'.
\end{align}
Here, $\spvc{C}^\KS \in \MatSpace{C}{2N_b}{2N_b}$ and $\spvc\epsilon^KS \in \MatSpace{R}{2N_b}{2N_b}$ are the Kohn--Sham
molecular orbital (KS-MO) coefficients and orbital eigenenergies, respectively. In the limit when the external potential
is spatially local but has spin structure, \cref{eq:VXCBasis} may be written as
\begin{equation}
    \label{eq:VXCBasis}
    V^{xc,\sigma\sigma'}_{\mu\nu} =
    \int_{\mathbb R^3} \frac{\delta \tilde E^{xc}[\gamma^1_s]}{\delta \rho^{\sigma\sigma'}(\vc{r})} \chi^*_\mu(\vc{r})  \chi_\nu(\vc{r})\,\dd^3\vc{r},
\end{equation}
where $\rho^{\sigma\sigma'}(\vc{r})$ are the spin basis coefficients of the one particle density defined in \cref{eq:Densities}
and we have exploited the hermiticity of its spin labels. The form of \cref{eq:VXCBasis} will allow for several simplifications
in the methods which will be developed in \cref{sec:NCDFT}.


\subsection{Spin Symmetries of Non--Interacting Wave Functions} 

In this section, we consider the spin symmetries of a general Fock matrix, $\spvc F$, which could be described as either
\cref{eq:HFMat} for Hartree--Fock wave functions, or \cref{eq:KSMat} for Kohn--Sham wave functions.
The full treatment of the spin--structure of the Fock matrix is known as the generalized self--consistent field (GSCF) method
which takes the form of generalized Hartree--Fock (GHF) and generalized Kohn--Sham (GKS) for the two methods, respectively.
In GSCF, no assumptions are made regarding the spin symmetry of the non--interacting wave function, i.e. the resulting lowest i
energy
GSCF wave function need not be an eigenfunction of $\op{S}_z$ or $\op{S}^2$ regardless of the symmetries are inherent
from the electronic Hamiltonian \todo{cite Josh and Pauldus}. In general, if the electronic Hamiltonian contains spin structure,
a GSCF framework must be employed and the Fock matrix may be decomposed in the same manner as the general operators
in \cref{eq:SpinorOpPauli} via ,
\begin{equation}
  \spvc F^\mathrm{GSCF} = \vc F^0 \otimes \pauli{0} + \vc F^1 \otimes \pauli{1} + \vc F^2 \otimes \pauli{2} + \vc F^3 \otimes \pauli{3}.
\end{equation}
However, if the electronic Hamiltonian admits spin as a symmetry, 
$[\op{H}_{el},\op X] = 0$, $\op{X} \in \{\op{S}_z,\op{S}^2\}$, several simplifications can be made in the treatment of the
Fock matrix, and more generally in the structure of the MO coefficients.

We examine the difference spin symmetries in turn, noting that $\op{S}^2$ as a symmetry implies $\op{S}_z$ as a symmetry
but not the converse. If we restrict our solutions of the SCF equations to adhere to $\op{S}_z$ symmetry, but not
$\op{S}^2$, the SCF matrix equations can adopt the block diagonal form,
\begin{equation}
  \label{eq:USCFEq}
  \begin{bmatrix}
    \vc{F}^\saa & \vc{0} \\ \vc{0} & \vc{F}^\sbb
  \end{bmatrix}
  \begin{bmatrix}
    \vc{C}^{\alpha} & \vc{0} \\ \vc{0} & \vc{C}^{\beta}
  \end{bmatrix}
  =
  \begin{bmatrix}
    \vc{S} & \vc{0} \\ \vc{0} & \vc{S}
  \end{bmatrix}
  \begin{bmatrix}
    \vc{C}^{\alpha} & \vc{0} \\ \vc{0} & \vc{C}^{\beta}
  \end{bmatrix}
  \begin{bmatrix}
    \vc{\epsilon}^{\alpha} & \vc{0} \\ \vc{0} & \vc{\epsilon}^{\beta}
  \end{bmatrix}.
\end{equation}
\Cref{eq:USCFEq} is referred to as the Pople--Nebst equation \todo{cite}, and the restriction on the SCF
solution is known as unrestricted SCF (USCF), or analogusly UHF (UKS) for Hartree--Fock (Kohn--Sham) wave functions.
The block structore of \cref{eq:USCFEq} warrants brief discussion as a number of new concepts have been implicitly
introduced. Due to the enforced block structure, the eigenvectors of \cref{eq:USCFEq} are not generally able
to be placed into energetic order. In a sense, USCF separates the spin components of the SCF equations into
two coupled (through $\op J$) eigenvalue problems which partition $\mathcal{C}$ such that all orbtials are either
purely spin--up ($\alpha$) or spin--down ($\beta$) in nature. That is to say
$\mathcal{C} = \mathcal{C}^\alpha  \cup \mathcal{C}^\beta$ with $\mathcal C^\alpha \cap \mathcal C^\beta = \emptyset$
such that
\begin{equation}
  \mathcal C^\alpha = \{ \ket{\phi_p} \in \mathcal C \text{ s.t. } \phi_p(\vc{x}) = \phi^\alpha_p(\vc{r}) \alpha(\sigma) \},
\end{equation}
and similarly for $\mathcal C^\beta$.














% Non--Relativistic Hamiltonian
\section{Non--Relativistic Molecular Hamiltonians: The Schr\"{o}dinger Equation}
\label{sec:NRH}

Fundamental to the description of any quantum molecular system, relativistic or non--relativistic, is the non--relativistic
molecular Hamiltonian, $\op{H}^\NR : \hilb{H}^\NR \mapsto \hilb{H}^\NR$, where $\hilb{H}^\NR$ describes a quantum
system containing $N_\mathrm{el}$ electrons and $N_\mathrm{nuc}$ nuclei in the absence of relativistic effects. 
%As such,
%it is an approximation to the true Hilbert space which describes the physical state of affairs exactly, but it often
%provides a sufficient model for systems when effects such as those which arise from special relativity are unimportant.
$\op{H}^\NR$ is so fundamental to the description
of quantum molecular systems in that it is the linear operator on $\hilb{H}^\NR$ which represents the non--relativistic 
total energy of the system.  In the absence of external fields, $\op{H}^\NR$ takes the form
\begin{equation}
  \label{eq:FullHNR}
  \op{H}^\NR = \op{H}_{el}^\NR + \op{H}_{nuc}^\NR + \op{H}_{mix}^\NR
\end{equation}
where 
\begin{subequations}
\begin{align}
  &\op{H}_{el}^\NR  = \sum_i^{N_\mathrm{el}}\op{T}^\NR(i) + \frac{1}{2}\sum_{i \neq j}^{N_\mathrm{el}} \op{g}^C(i,j),\\ 
  &\op{H}_{nuc}^\NR = \sum_A^{N_\mathrm{nuc}}\op{T}^\NR(A) + \frac{1}{2}\sum_{A \neq B}^{N_\mathrm{nuc}} \op{g}^C(A,B),\\ 
  &\op{H}^\NR_{mix} = \sum_i^{N_\mathrm{el}} \sum_A^{N_\mathrm{nuc}} \op{g}^C(i,A).
\end{align}
\end{subequations}
Here we have denoted operator action onto the electronic degrees of freedom as $i,j$ and $A,B$ for the nuclear degrees of
freedom. For a general (electronic or nuclear) coordinate, $\xi$, the non--relativistic kinetic energy operator, 
$\op{T}^\NR$, is given by
\begin{align}
  \op{T}^\NR(\xi) = \frac{1}{2m_\xi} \op{\vc{p}}(\xi) \cdot \op{\vc{p}}(\xi),
\end{align}
where $\op{\vc{p}}$ is the linear momentum operator and $m_\xi$ is the mass of the $\xi$-th particle.
$\op{g}^C$ is the two body Coulomb operator which describes the electrostatic interaction between charged particles.

In the spinor representation, these operators take the general form (in atomic units)
\begin{align}
  \label{eq:SpinorKinMom}
  &\spop{\vc{p}}(\vc{x}_\xi) = -\ii \nabla_\xi \otimes \op{I}_S(\xi) \quad \Longrightarrow \quad 
  \spop{T}^\NR(\vc{x}_\xi) = - \frac{1}{2m_\xi} \Delta_\xi \otimes \op{I}_S(\xi),
\end{align}
\begin{align}
  \label{eq:SpinorCoulomb}
  &\spop{g}^C(\vc{x}_\xi,\vc{x}_\zeta) = 
    \frac{Z_\xi Z_\zeta}{r_{\xi\zeta}} \otimes \op{I}_S(\xi) \otimes \op{I}_S(\zeta), \qquad r_{\xi\zeta} = \vert \vc{r}_\xi - \vc{r}_\zeta \vert
\end{align}
where $\nabla_\xi$ and $\Delta_\xi$ are the gradient and Laplacian operators acting on the $\xi$-th spatial coordinate,
respectively. $Z_\xi$ is the charge of the $\xi$-th particle, which in atomic units is given by $-1$ for electrons and 
the number of protons for a particular nucleus, respectively . 
$\op{I}_S(\xi)$  is the identity spin operator for particle $\xi$, which has been introduced to make a
careful distinction between the spinor basis of electrons, which are spin--1/2 fermions, and nuclei, which are in general
\emph{not} spin--1/2 fermions and thus carry a much more complicated spin structure \todo{cite}. For electrons, $\op{I}_S=\pauli{0}$.
The presence of $\op{I}_S$ in \cref{eq:SpinorKinMom,eq:SpinorCoulomb} may be interpreted as the action of these operators 
\emph{do not} manipulate the spin degrees of freedom of the total wave function.

As has been previously stated on numerous occasions, the primary focus of this work is to treat the many body
electronic problem for molecular systems, not the quantum nature of the nuclei. The combined quantum treatment
of \cref{eq:FullHNR} obfuscates this point in that the presence of $\op{H}^\NR_{mix}$ intimately couples the
electronic and nuclear degrees of freedom. Thus it would be of practical utility to, in some way, decouple
the quantum treatment of the electrons and the nuclei such that they may be treated separately. 
%To this end,
%we will work within the Born--Oppenheimer approximation \todo{cite} for the molecular Hamiltonian such that
%we will consider a single tensor product ansatz for the total molecular wave function $\ket{\Psi_{tot}}$,
To this end, we will work within the Born--Oppenheimer ansatz for the molecular wave function \cite{Oppenheimer27_457,Tully98_407} such
that it may be written as a single tensor product of an electronic and nuclear wave function,
\begin{equation}
  \ket{\Psi_{tot}} \approx \ket{\Psi_{el}} \otimes \ket{\Theta_{nuc}}.
\end{equation}
Due to a large disparity in mass between electrons and nuclei, the energetic regimes which describe their respective
dynamics are typically well separated. Namely, from the inertial frame of the electrons, one might approximate
the nuclear kinetic energy to be negligible, i.e.
\begin{equation}
\innerop{\Psi_{tot}}{\sum_A^{N_\mathrm{nuc}} \op{T}^\NR(A) }{\Psi_{tot}} \approx 0.
\end{equation}
This assumption is referred to as the Born--Oppenheimer approximation (a concept distinct from from the 
Born--Oppenheimer ansatz for the molecular wave function). This is typically a safe assumption for reasonably
heavy nuclei as the ratio of the electronic and nuclear kinetic energies are of the same order as
the ratio of the electronic and proton masses, $\approx 10^{4}$. Thus, from the electronic perspective,
the nuclear configuration is approximately static, and the electrons only ,,feel" the electrostatic
potential of a fixed nuclear wave function at any given time. This approach is therefore analogous to
the mean field treatment of the Hartree--Fock wave function in \cref{eq:HFEq}. Denoting a particular fixed nuclear wave
function $\ket{\Theta_{nuc}^{fix}}$, we may write down a Hamiltonian which acts on the electronic 
component of the wave function and is valid in the inertial frame of the electrons,
\begin{equation}
\label{eq:NRBOH}
\op{H}^\BO_{el} = \innerop{\Theta_{nuc}^{fix}}{\op{H}^\NR - \sum_A^{N_\mathrm{nuc}}\op{T}^\NR(A)}{\Theta_{nuc}^{fix}}
=E_{nn} + \sum_i^{N_\mathrm{el}} \op{T}^\NR(i) + \op{V}_{ne}(i) + \sum_{i\neq j}^{N_\mathrm{el}} \op{g}^C(i,j),
\end{equation}
where
\begin{align}
  &\op{V}_{ne}(\vc{x}_i) = 
    -\sum^{N_\mathrm{nuc}}_A \left(\int_{\mathbb R^3} \frac{Z_A\rho^{1,0}_\mathrm{nuc,A}(\vc{R})}{\vert \vc{r}_i - \vc{R} \vert} \dd^3\vc{R}\right) \otimes \pauli{0}, \label{eq:Vne}\\
&E_{nn} = \frac{1}{2} \sum^{N_\mathrm{nuc}}_{A\neq B} \iint_{\mathbb R^3} \frac{Z_AZ_B\rho^{1,0}_\mathrm{nuc,A}(\vc{R})\rho^{1,0}_\mathrm{nuc,B}(\vc{R}')}{\vert \vc{R} - \vc{R}' \vert} \dd^3\vc{R} \dd^3\vc{R}'.
\end{align}
$\rho^{1,0}_{nuc,A}(\vc{R})$ is the one particle scalar density of the $A$--the nucleus defined through a proper generalization of
\cref{eq:SpinorDensity} for non spin--1/2 fermions. The explicit form of $\rho^{1,0}_{nuc,A}(\vc{R})$ immaterial
to this work except for the property,
\begin{equation}
\sum_A^{N_\mathrm{nuc}} \int_{\mathbb R^3} \rho_{nuc,A}^{1,0}(\vc{R}) \dd^3\vc{R} = N_\mathrm{nuc}.
\end{equation}
This criteria is clearly met if $\rho^{1,0}_{nuc,A}(\vc{R})$ is a normalized function. In this work, we will approximate
$\rho^{1,0}_{nuc,A}(\vc{R})$ as a classical charge distribution described by a single Gaussian function \todo{cite},
\begin{equation}
\rho^{1,0}_{nuc,A}(\vc{R}) = \mathcal{N} e^{-\gamma_A(\vc{R} - \vc{R}_A)^2},
\end{equation}
where $\vc{R}_A$ is the classical nuclear position and \todo{give expressions for exponent and normalization condition}


The wave equation governed by \cref{eq:NRBOH} as a specialization of \cref{eq:QWave} is given by
\begin{equation}
\label{eq:NRSCH}
\op{H}^\BO_{el} \ket{\Psi_{el} (t)} = \ii \partial_t \ket{\Psi_{el} (t)}
\end{equation}
and will be referred to as the electronic Schr\"{o}dinger equation within the Born--Oppenheimer approximation \cite{Tully98_407}.
In this work, all non--relativistic specializations of \cref{eq:QWave} will referred to as some variant of the Schr\"{o}dinger
equation due to the explicit presence of the kinetic energy operator in the Hamiltonian. As such, it is 
a specialization of the original Schr\"{o}dinger Hamiltonian of the form
\begin{equation}
\label{eq:SchHam}
\op{H}(t) = \op{T}^\NR + \op{V}(t).
\end{equation}
\todo{Restricted / Unrestricted} \todo{Fock Operator}

There are a number of problems inherent in the form of \cref{eq:SchHam,eq:NRBOH}. The first problem that is of interest to this work,
and perhaps the most glaring in the context of molecular calculations involving the quantum treatment of electrons,
is that in the limit of electrostatics in the absence of external fields there is only trivial action of the
Hamiltonian onto the spin components of the electronic  wave function. To be clear, electronic spin is inherent in any quantum
treatment of electrons as it manifests naturally through the irreducible representations of the Galilean group
which governs non--relativistic mechanics \cite{Levy67_286}.  Phenomenologically, one may introduce non--trivial spin manipulation into the non--relativistic
Hamiltonian through the interaction with an external magnetic field, $\vc{B}$, via the spin--Zeeman term,
\begin{equation}
\op{H}^\BO_{el} \mapsto  \op{H}^\BO_{el} + \op{H}^\mathrm{Zeeman}, \qquad \op{H}^\mathrm{Zeeman}  = 
  \frac{1}{2} \sum_{i = 1}^{N_\mathrm{el}}\sum_{k=1}^3 B^k \left( \pauli{k}(i) + \op{L}_k(i) \right),
\end{equation}
due to observation of the linear relationship between the magnetic field strength and energy level splittings in the Stern--Gerlach 
experiment\cite{Napolitano17_book}. Here, $\op{\vc L}$ is the one body orbital angular momentum operator. 
It is well known, however, that $\op{H}^\mathrm{Zeeman}$ only yields a proper physical description
of the electronic spin degrees of freedom in the strong field limit \todo{cite}, whereas in the weak field limit, intrinsic magnetic
effects which arise from special relativity, such as spin--orbit coupling, become energetically competitive.

The second, perhaps more subtle problem relating specifically to the forms of \cref{eq:SchHam,eq:NRSCH} is that they are manifestly
incapable of adhering to the laws of special relatively, i.e. it is impossible to write down a Hamiltonian of the form in
\cref{eq:SchHam} which is Lorentz covariant. This problem is easily identified through recognizing that the Schr\"{o}dinger
equation is quadratic in spatial coordinates through $\op{T}^\NR$ and linear in time, and thus incapable of being compatible with Lorentz boosts. At first glance,
one might be tempted to think in terms of a classical analogue for the quantum picture, where for velocities much lower than the speed of light,
the dynamics of classical bodies is approximately governed by Newtonian mechanics: the classical analogue of the Schr\"{o}dinger equation.  Within such a mindset, one might consider
relativistic effects as only being important for heavy elements, such as Gold or Uranium, due to the fact that their core electrons
move at velocities which are a considerable fraction of the speed of light. However, as is often the case with the quantization
of classical mechanics, such a simplistic assumption yields qualitatively incorrect model physics even for light elements such as Carbon and Oxygen\todo{cite Pyyko}.
The realization of relativistic effects in light elements typically manifests in context of the \emph{ab initio} introduction of spin couplings
into the Hamiltonian, which not surprisingly also solves the aforementioned problem with treating electronic spin non--relativistically. 
The following section provides a brief overview of the treatment of relativistic effects in molecular quantum mechanics.



% Relativistic Hamiltonian
\section{Approximate Relativistic Hamiltonians: The Dirac Equation}
\label{sec:RELH}

To be clear, although it is rarely outright stated in the context of the electronic structure theory
literature, there is no \emph{truly} relativistically covariant formulation of many--body quantum 
mechanics. This is due to the fact that the Coulomb interaction only treats the instantaneous interactions
of charged particles and is thus manifestly incapable of adhering to the principles of special relativity.
One must venture into the realm of quantum field theory (QFT) in order to develop truly covariant quantized 
descriptions of the electromagnetic force; however, QFT has proven to be rather difficult to exploit in
the context of practical calculations on molecular systems \todo{cite Liu}. Here, we will work with
\emph{approximate} relativistic descriptions of molecular quantum mechanics within a Hamiltonian
formulation, which have been shown to have good agreement with experiment \todo{cite}.
For simplicity in the subsequent developments, we will posit \emph{a priori} the validity of the 
Born--Oppenheimer ansatz and approximations of the previous section in the context of relativistic
theory \co{justify?}. Further, while no approximation of the quantum nature of the nuclei was required in the 
context of the Sch\"{o}dinger equation, in the context of relativistic theory we will approximate
nuclei to be classical charge distributions, i.e. no intrinsic angular momentum (spin) which
yields an absence of magnetic interactions into the Hamiltonian within the fixed nuclei approximation.
This approximation is justified \co{because?}, and will drastically simplify the resulting 
quantum mechanical treatment.

Within the context of the electronic problem, relativistic quantum mechanics is approximately governed
by the Breit equation,
\begin{equation}
\label{eq:BreitEq}
\op{H}^\DCB \ket{\Psi^\FrC_{el} (t)} = \ii \partial_t \ket{\Psi^\FrC_{el} (t)},
\end{equation}
where $\op{H}^\DCB : \hilb{H}^\FrC \rightarrow \hilb{H}^\FrC$ is the Dirac--Coulomb--Breit (DCB) Hamiltonian.
In the absense of external fields , the DCB Hamiltonian for $N$ electrons is given in atomic units by \todo{cite}
\begin{equation}
  \label{eq:DCBHam}
  \op{H}^\DCB = \sum_i^N  \op{h}^D(i) + \op{V}_{ne}(i) \otimes \vc{I}_2 + \frac{1}{2} \sum_{i \neq j}\op{g}^C(i,j) \otimes \vc{I}_2 
  - \op{g}^B(i,j),
\end{equation}
where $\op{V}_{ne}$ and $\op{g}^C$ are the external potentials and classical Coulomb interactions of \cref{eq:Vne} and 
\cref{eq:SpinorCoulomb}, respectively, and
\begin{alignat}{2}
&\op{h}^D(i) = 
c \vc{a}(i) \cdot \op{\vc {p}}(i) + (\vc{b} - \vc{I}_4) c^2, \qquad &&\text{Dirac Hamiltonian}, \label{eq:DiracHam}\\
%
  &\op{g}^B(\vc{x}_i,\vc{x}_j) = \frac{1}{2r_{ij}} \left( \vc{a}(i)\cdot\vc{a}(j) - \frac{(\vc{a}(i) \cdot \vc{r}_{ij}) (\vc{a}(j) \cdot \vc{r}_{ij})}{r_{ij}^2}\right), 
\qquad && \text{Breit interaction}. \label{eq:BreitInt}
\end{alignat}
Here we have denoted the the Dirac matrices as
\begin{align}
  \vc{a}_k &= \begin{bmatrix} \vc{0}_2 & \pauli{k} \\ \pauli{k} & \vc{0}_2 \end{bmatrix}, \quad k \in \{1,2,3\},\\
  \vc{b} &= \begin{bmatrix} \vc{I}_2 & \vc{0}_2 \\ \vc{0}_2 & -\vc{I}_2 \end{bmatrix}.
\end{align}
$\vc{I}_2$ and $\vc{I}_4$ are the 2-by-2 and 4-by-4 identity matrices, respectively, and have used the shorthand notations
\begin{align}
  \vc{a} \cdot \op{\vc{p}}  &= -\ii \left( \frac{\partial}{\partial x} \otimes \pauli{1} + \frac{\partial}{\partial y} \otimes \pauli{2} + \frac{\partial}{\partial z} \otimes \pauli{3} 
    \right) \otimes \begin{bmatrix} \vc{0}_2 & \vc{I}_2 \\ \vc{I}_2 & \vc{0}_2 \end{bmatrix}, \\
  \vc{a}(i) \cdot \vc{a}(j) &= \sum_{k = 1}^3 \vc{a}_k(i) \vc{a}_k(j), \\
  \vc{a}(i) \cdot \vc{r}_{ij} &= (x_i - x_j) \otimes \vc{a}_1(i) + (y_i - y_j) \otimes \vc{a}_2(i) + (z_i - z_j) \otimes \vc{a}_3(i).
\end{align}
From the structure of $\op{H}^\DCB$, or more specifically  $\op{h}^D$ and $\op{g}^B$, it is clear that $\hilb{H}^\FrC$
consists of higher dimensional wave functions than in the non--relativistic case. 
The formal structure of $\ket{\Psi_{el}^\FrC}$ in the spinor basis is given by \todo{cite}
\begin{equation}
  \label{eq:Bispinor}
  \Psi_{el}^\FrC(\vc{x}_1,\vc{x}_2,\ldots,\vc{x}_N) = 
    \begin{bmatrix}
       \Psi_{el}^\mathrm{L}(\vc{x}_1,\vc{x}_2,\ldots,\vc{x}_N) \\
       \Psi_{el}^\mathrm{S}(\vc{x}_1,\vc{x}_2,\ldots,\vc{x}_N) 
    \end{bmatrix},
\end{equation}
where the so--called ``large" (superscript $L$) and ``small" (superscript $S$) components exhibit the same spinor
structure as non--relativistic spinor wave functions. As such, wave functions of the form \cref{eq:Bispinor} are often
referred to as bispinors, or four--component wave functions due to their four complex spin components.

The formal treatment of relativistic quantum mechanics is beyond the scope of this work, and we shall default to far
more rigorus treatments of its development in some of the standard texts on the subject \todo{cite}. There are, however,
several aspects of \cref{eq:BreitEq} which will play an important role in the following developments. The block structure
of \cref{eq:DiracHam,eq:BreitInt} indicate that the eigenspectrum of \cref{eq:DCBHam} consists of a set of energy
eigenstates (in the sense of \cref{eq:TIQWave}) containing both positive and negative energy solutions separated
by an energy gap of $2c^2$.
The negative energy eigenstates (relative to $2c^2$)  of \cref{eq:DCBHam} will be referred to as positronic
solutions and the positive energy solutions will be referred to as electronic solutions. As the names might suggest,
in the context of treating the electronic problem molecular physics we will be interested in the electronic solutions to \cref{eq:BreitEq}, not
the positronic solutions. However, this does not negate the presence of these negative energy states, and indeed their presence is
crucial to a rigorus treatment of relativistic effects. 

In the context of calculations involving the DCB Hamiltonian, explicit treatment of the positronic states causes a number of practical
problems which render its straightforward application nontrivial. Due to these negative energy states, the energy functional of
\cref{eq:EnergyFunctional} is unbounded from below. Thus minimization of \cref{eq:EnergyFunctional} over purely \emph{electronic}
states is not possible with standard optimization techniques and requires rather opaque projection schemes to avoid variational
collapse \todo{cite}. Further, due to the kinetic balance relationship between the large and small components of the 
electronic wave function \todo{cite}...,





\todo{Dirac Hamiltonian + why it fixes things}
\todo{DCB Hamiltonian}
\todo{BP hamiltonian from DCB}
\todo{NR limit of BP}
\todo{General Two component}
\todo{X2C (why its better), possibly appendix on implementation?}
\todo{Uncontracted basis?}





% Semi--Classical Light Matter
\section{Semi--Classical Light--Matter Interaction}
\label{sec:SCLMI}

% Polarization Propagator
\subsection{The Resonant Convergent Polarization Propagator}
\label{sec:PolarProp}

% Absorption Cross Section
\subsection{The Linear Absorption Cross Section}
\label{sec:AbsorptionTheory}
